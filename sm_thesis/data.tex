\chapter{Data and Processing}

In this chapter we will discuss the data used to build our recommender system
and the processing for the data.

\section{Clothing Item Data and Processing}

The primary data for our system is data about clothing items. This data has been
scraped from retail websites, specifically Topshop\footnote{www.topshop.com} and
Abercrombie\footnote{www.abercrombie.com}. The scraped data includes product
images, type, description, prices and any auxilary information such as color,
size etc. 

For instance, image XXX shows an example of clothing item data obtained from
Topshop. Our database includes 5000 clothing items from the above websites.
Along with the core features obtained form the scraped data (e.g.
type of item such as shorts, tops etc), we further process this data to extract
features from the images and text as described below. Note that scraping and
post-processing of scraped data varies with the retail store.

\subsection{Derived Information}

We process the scraped data to obtain features such as material, weather index,
casual index, price index etc. We also process the image to obtain features such
as colors, texture etc.

Text-based Features:
\begin{itemize}
  \item Material: Item descriptions are processed to identify materials for the
  item (e.g. silk, cotton, leather etc)
  \item Weather index: Type of the item (e.g. shorts, tees) and material is used
  to determine the weather index for a particular item (e.g. shorts = Hot,
  Cardigan = Cool)
  \item Casual index: Similar to weather, item type and material are used to
  determine the casual index for an item (0 = not casual, 1 = any, 2 = not
  casual)
  \item Price index: We aggregate the price of certain type of items in a
  store to determine the min and max price for that item type and use it to
  determine the relative price of that item as 0 = Low, 1 = Medium, and 2 =
  High.
  \item Type specific features: depending on the item type, we process the text
  to define type-specific features such as sleeve length for tops and heel
  height for shoes
\end{itemize}

Image-based Features:
\begin{itemize}
  \item Color: We use MATLAB to binarize the product image to remove background
  and then process the image to determine the colors present in the image and
  the \% of the image occupied by the color. As in \cite{}, we record the
  primary, secondary and decorative colors, in addition to the total number of
  colors.
  \item Texture: We use Gabor filters to determine if the item has texture
  \item Pattern: We combine color and texture information to determine if there
  is a pattern on the clothing item.
\end{itemize}

\section{Combination Data and Processing}

The second component of our dataset is the data about combinations. Using
information about ``Collections'' available on retail websites (examples of
combinations put together by retailers) and examples of combinations from
fashion blogs, we obtain examples of good combinations. Our dataset includes
examples of 200 good combinations of clothing items. Note that our dataset
includes examples only of ``positive training examples.'' The reasons for this
are two-fold: (a) most readily available, real-world examples are of good
fashion and (b) fashion is extremely subjective and hence we would like to
minimize subjective biases in defining good fashion.
