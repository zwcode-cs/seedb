\section{Problem Statement}
\label{sec:problem_statement}

In this paper, we describe our prototype of \SeeDB, a system to
automatically identify and visually highlight interesting aspects of a dataset.
Given a database $D$ and a query $Q$, \SeeDB\ finds and visualizes the most
interesting aspects of $Q$ with respect to the underlying dataset. 

To identify the most interesting aspects of $Q$, \SeeDB\ considers various
``views'' of the query, where a view can be some way of slicing or aggregating
the data so that it may be visualized in terms of histograms, time series charts
etc. Currently \SeeDB\ limits views to those that can be generated by adding one
aggregate and one group-by clause. To illustrate, consider the query $Q$
for ``Stapler'' product sales in 2013 from the SuperStore dataset. A possible
``view'' of the input query can be constructed by {\it aggregating the sales
data by region}. The corresponding SQL query, which we call the {\it input view
query}, is shown below.

\noindent 
\begin{small}
\begin{verbatim}
Q = SELECT Geography.Region, AVG(Sales.sales) FROM 
Sales JOIN Geography JOIN Product WHERE 
Products.name = "Staplers" AND Sales.year = 2013
GROUP BY Geography.Region
\end{verbatim}
\end{small}

The above query produces the result table shown in
Figure~\ref{}.a that can then be visualized as a histogram shown in
Figure~\ref{}.b. To determine if this view is interesting or useful, \SeeDB\
applies the same view, i.e., the same aggregate and group-by, to the
entire dataset, $D$. We call this query the {\it data view query} (shown below). This
query generates the result table shown in Figure~\ref{}.c and histogram from
Figure~\ref{}.d.

\noindent 
\begin{small}
\begin{verbatim}
Q = SELECT Geography.Region, AVG(Sales.sales) FROM 
Sales JOIN Geography JOIN Product
GROUP BY Geography.Region
\end{verbatim}
\end{small}

Results of {\it input view query} and {\it data view query} are comparable since
they have the same values in column 1. \SeeDB\ computes the utility of this view
as the difference in the distribution of results (after normalization) of these
two view queries. Using one potential metric, the Earth-Mover-Distance, we can
compute the difference between these two queries to be XXX. To get the most interesting
views, \SeeDB\ must compute and evaluate the utility of a large number of such
views and pick the views with largest utility. 

\subsection{Definitions}
\label{subsec:definitions}
We classify table attributes into two types for use by \SeeDB.
\begin{denselist}
\item {\bf Dimension Attribute}: An attribute is considered a dimension
attribute if the attribute has categorical or ordinal values. Dimension attributes are those
that users can insert into a group-by clause.
\item {\bf Measure Attribute}: An attribute is considered a measure
attribute if the attribute is numeric and can be aggregated. Measure attributes are usually
metrics that the user cares about (e.g. sales, cost etc.) and can be inserted
into aggregate clauses.
\end{denselist}

We denote the subset of data selected by query $Q$ as $D(Q)$ and a view
$\mathcal{V}$ with group-by attribute $d_i$ and aggregate attribute $m_j$ as
$V(Q, D, d_i, m_j)$. The normalized results of the {\it input view query} are
denoted by $P_\mathcal{V}(Q)$ and those of the {\it data view query} by
$P_\mathcal{V}{D}$ (the results are normalized to values $\in [0, 1]$ so that
the sum of the aggregates = 1). 

\subsection{Utility Function}
\SeeDB\ uses a utility function or distance metric, $\mathcal{U}:
\mathcal{R}^n \times \mathcal{R}^n \rightarrow {R}$, to determine the difference
between the two results, $P_\mathcal{V}(Q)$ and $P_\mathcal{V}(D)$.
Higher the difference, larger the utility. \SeeDB\ can use a variety of existing
metrics, a few of which are discussed below.

\begin{denselist}
  \item {\bf Earth Movers Distance}: Commonly used to measure differences
  between color histograms from images, EMD is a popular metric for comparing
  discrete distributions.
  \item {\bf Euclidean Distance}: The L2 norm or Euclidean distance considers
  the two distributions are points in a high dimensional space and measures the
  distance between them.
  \item {\bf Kullback-Leibler Divergence}: K-L divergence measures the
  information lost when one probability distribution is used to approximate
  another.
  \item {\bf Jenson-Shannon Distance}: Based on the K-L divergence, this
  distance measures the similarity between two probability distributions.
\end{denselist}

\subsection{Formal Definition}
Given an input query $Q$, a dataset $D$ with
dimension attributes $\mathcal{D}=\{d_1, d_2\ldots d_m\}$ and measure attributes
$\mathcal{M}=\{m_1, m_2\ldots m_n\}$, a utility function $\mathcal{U}:
\mathcal{R}^n \times \mathcal{R}^n \rightarrow {R}$, and a positive integer $k$,
\SeeDB\ computes and returns the top-$k$ views of query $Q$ generated by adding
a group-by dimension attribute $d_i \in \mathcal{D}$ and an aggregate measure
attribute $m_j \in \mathcal{M}$ that have the highest utility as defined by
utility function $\mathcal{U}$.

\subsection{Extensions}
There are two important variations of the problem that can be addressed exactly
by the same techniques discussed in the paper. These are:
\begin{denselist}
\item {\bf Group-by clauses with multiple dimension attributes}: It is
straightforward to extend the \SeeDB\ techniques  to group-by clauses with multiple attributes.
However, for ease of exposition and visualization, we limit the number of
attributes in the group-by clause to one.
\item {\bf Comparison of two-queries}: Instead of comparing the views of
the input query to the entire underlying dataset, it may be more appropriate to
compare them to another subset of the data (e.g. sales of ``Staplers'' vs.
sales of ``Printers''). This simply involves replacing the dataset parameter $D$
with a second query $Q'$. This variation is important since it can help users
find interesting differences in data. Our techniques apply to this variation
unchanged and we show experimental results for this variation.
\end{denselist} 


%Trend in the subset of the data that deviates from the corresponding trend in
%the overall data.