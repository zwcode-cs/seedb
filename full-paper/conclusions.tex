%!TEX root=document.tex

\section{Conclusions}
\label{sec:conc}

Finding the right visualization given a query of interest is a
laborious and time-consuming task.
In this paper, we presented \SeeDB, a visualization recommendation
engine to help data scientists 
rapidly identify interesting and useful visualizations of their data.
Our implementation of \SeeDB on top of traditional DBMSs led
to several inefficiencies that we remedied using two suites of optimization 
techniques, scanning-based, and pruning-based techniques. 
These techniques help us reduce latency from XXX to YYY to ZZZ,
and the gains from the two suites are in fact additive. 
Our user study indicated that \SeeDB provides {\em statistically significantly} useful
visualizations, and helped us identify some directions for future improvement.
\SeeDB is an important first step in our exploration of 
automated visualization recommendation tools, 
paving the way toward automating all the tedium of data analysis.



% We presented two implementations of \SeeDB, one that runs on top of existing
% DBMSs and another that is a custom execution engine that supports shared table scans
% and aggressive pruning of low-quality views.
% Our experimental evaluation on a range of real and synthetic datasets shows that
% our optimizations reduce latency to just a few seconds to evaluate hundreds of different
% visualizations.
% In addition, our experiments on our custom execution engine show that our pruning
% heuristics can reduce latency by 10-fold by aggresively pruning low-utility views.
% This provides us a means to rapdily surface the top few views and then
% gradually return additional views.
% Finally, we demonstrated that our pruning
% strategies do not adversely affect accuracy of views returned.