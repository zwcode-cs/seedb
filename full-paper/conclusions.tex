%!TEX root=document.tex

\section{Conclusions}
\label{sec:conc}

Finding the right visualization given a query of interest is a
laborious and time-consuming task.
In this paper, we presented \SeeDB, a visualization recommendation
engine to help users 
rapidly identify interesting and useful visualizations of their data using a deviation-based 
metric to highlight attributes with unusual variation.
Our implementation of \SeeDB runs on top of a relational engine, and employs two types of optimization 
techniques, sharing-based, and pruning-based techniques, to obtain near-interactive performance.
These techniques help us reduce by a factor of 40X--100X, with the optimizations combining in an additive way. 
Furthermore, our user study shows that \SeeDB provides useful
visualizations, that both help users find interesting visualizations with fewer iterations, and that
users find help as an augmentation to a visualization system.
In conclusion, \SeeDB is an important first step in our exploration of 
automated visualization recommendation tools, 
paving the way toward automating the tedium of data analysis.



% We presented two implementations of \SeeDB, one that runs on top of existing
% DBMSs and another that is a custom execution engine that supports shared table scans
% and aggressive pruning of low-quality views.
% Our experimental evaluation on a range of real and synthetic datasets shows that
% our optimizations reduce latency to just a few seconds to evaluate hundreds of different
% visualizations.
% In addition, our experiments on our custom execution engine show that our pruning
% heuristics can reduce latency by 10-fold by aggresively pruning low-utility views.
% This provides us a means to rapdily surface the top few views and then
% gradually return additional views.
% Finally, we demonstrated that our pruning
% strategies do not adversely affect accuracy of views returned.