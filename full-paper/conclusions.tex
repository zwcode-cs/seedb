%!TEX root=document.tex

\section{Conclusions}
\label{sec:conc}

Finding the right visualization given a query of interest is a
laborious and time-consuming task.
In this paper, we presented \SeeDB, a visualization recommendation
engine to help users 
rapidly identify interesting and useful visualizations using a deviation-based 
metric.
Our implementation of \SeeDB runs on top of a relational engine, and employs two types of optimization 
techniques, sharing-based, and pruning-based techniques, to obtain near-interactive performance.
These techniques reduce latency by over 100X, with the optimizations combining in a multiplicative way. 
Furthermore, our user study shows that our deviation-based metric can, in fact, capture interestingness of a 
visualization, and that \SeeDB enables users to find interesting visualizations faster.
In conclusion, \SeeDB is an important first step in our exploration of 
visualization recommendation tools, 
paving the way towards rapid visual data analysis.

\stitle{Acknowledgements.}
We thank the anonymous reviewers for their valuable feedback. We acknowledge support from grants IIS-1513407 and IIS-1513443 awarded by the National Science Foundation, funds from the Intel Science and Technology Center in Big Data, 
grant 1U54GM114838 awarded by NIGMS through funds provided by the trans-NIH Big Data to Knowledge (BD2K) initiative \break 
(www.bd2k.nih.gov), and funds provided by Google and Intel. The content is solely the responsibility of the authors and does not represent the official views of the funding agencies and organizations.

% We presented two implementations of \SeeDB, one that runs on top of existing
% DBMSs and another that is a custom execution engine that supports shared table scans
% and aggressive pruning of low-quality views.
% Our experimental evaluation on a range of real and synthetic datasets shows that
% our optimizations reduce latency to just a few seconds to evaluate hundreds of different
% visualizations.
% In addition, our experiments on our custom execution engine show that our pruning
% heuristics can reduce latency by 10-fold by aggresively pruning low-utility views.
% This provides us a means to rapdily surface the top few views and then
% gradually return additional views.
% Finally, we demonstrated that our pruning
% strategies do not adversely affect accuracy of views returned.