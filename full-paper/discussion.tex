%!TEX root=document.tex


\section{Discussion}
\label{sec:discussion}
The previous sections discussed how our current implementation of \VizRecDB can
efficiently find top views by pruning and aggressive query optimization.
In this section, we discuss some ongoing work and possible extensions of
\VizRecDB.

\stitle{User Study}
In this paper, we focus mainly on the implementation details of \VizRecDB and
ways to enable real-time interaction.
In parallel, we are also in the process of running a user study that evaluates
the \SeeDB\ frontend and end-to-end functionality. 
Our user study has two components: an MTurk\cite{} based component that
evaluates the quality of our deviation metrics and an in-person component that
involves data analytics experts and evaluates user interaction on the
frontend.
As in Section \ref{sec:experiments}, we use the BANK and DIAB dataset for our
evaluation.
\stitle{Improving Interactivity}
Our current work focuses on finding the most interesting views of a dataset and
presenting them to the user.
While our frontend currently allows basic interaction with the views presented
to the user, it is also essential to allow the user to interrogate our views
directly and further manipulate the data iteratively in the style of
\cite{2013-immens}.

\stitle{Incorporating User Preferences}
Although data ditributions and deviations in distributions can be used to
measure the interesting-ness of a view, it is currently difficult to take
context into account while recommending views.
For instance, in a sales dataset, an analyst may only be interested in grouping
data by store as compared to manufacturer, and our system should be learn these
preferences and take them into account.
One way to do so is to use traces of visualizations that analysts have generated
in the past or interacted with in \SeeDB and weight views based on prior
interest.
 In the future, we could also attempt to learn a distance metric based on user's
 feedback. 
 
 \stitle{Extending the Definition of Interesting}
 In this paper, we find views that are ``interesting'' using our definition of
 interesting measured as deviation from the expected distribution.
 Clearly, there are other definitions of interesting that are also valid and can
 be incorporated into \SeeDB.
 For instance, we can give the user the ability to specify a trend that they
 want to explore and ask \SeeDB to find views that either closely match the
 trend or show a large deviation from it.
 Similarly, we can chose the opposite definition of interesting and find the
 user views that are extremely similar to each other.
 We note that our techniques can be used essentially unchanged to solve this
 problem: the only thing that changes is our distance metric.
 
 \stitle{Binning and Joins}
 We currently assume that dimension attributes are either categorical or are
 numeric with a small cardinality.
 However, we can turn continous attributes into dimension attributes by
 performing binning (e.g. diving an age range into age groups).
 Binning and different granularities of binning can add an interesting dimension
 to our problem.
 
 Furthermore, our work currently assumes that we are querying a denormalized
 table.
 However, this is not a strict requriement. 
 Suppose that we are querying multiple tables in a star schema and want to find
 interesting views of these tables.
 One feasible and efficient technique is to first query the fact table in the
 schema, find interesting views and only for those attributes query the
 linked dimension tables.
 We can prove that exploring views in this manner does not lead to loss of
 high-utility views.
