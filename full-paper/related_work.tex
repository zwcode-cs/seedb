%!TEX root=document.tex


\section{Related Work}
\label{sec:related_work}
\mpv{have to rewrite}

\SeeDB\ is related to work from many fields:

\subsubsection*{Interactive Data Visualization Tools}
Over the past few years, the research community has introduced a number of
interactive data analytics tools such as ShowMe, Polaris, and 
Tableau~\cite{DBLP:journals/cacm/StolteTH08, DBLP:journals/tvcg/MackinlayHS07}.
Unlike \SeeDB, which recommends visualizations automatically, these tools place
the onus on the analyst to specify the visualization to be generated.
For datasets with a large number of attributes, it is not possible
for the analyst to manually study all the attributes; hence, interactive
visualization needs to be augmented with automated techniques of visualization.
Profiler is one such automated tool that allows analysts to detect anomalies in
data \cite{DBLP:conf/AVI/KandelPPHH12}.

Similar visualization specification tools have also been introduced by the
database community, including Fusion
Tables~\cite{DBLP:conf/sigmod/GonzalezHJLMSSG10} and the
Devise~\cite{DBLP:conf/sigmod/LivnyRBCDLMW97} toolkit.

Our work is also similar to VizDeck \agp{say something here.}

\subsubsection*{Data Cubes}
The work done in \SeeDB\ is of a flavor similar to previous literature in
building and browsing OLAP data cubes. Data cubes have been very well studied in
the literature \cite{DBLP:conf/SIGMOD/HarinarayanRU96,
DBLP:jounral/DMKD/GrayCBLR97}, and work such as
~\cite{DBLP:conf/vldb/Sarawagi99, DBLP:conf/vldb/SatheS01,
DBLP:conf/vldb/Sarawagi00, DBLP:conf/SIGKDD/OrdonezC09} has explored the
questions of allowing analysts to find explanations for trends, get suggest for
cubes to visist, identify generalizations or patterns starting from a single
cube. While we can reuse some of the similarlity metrics proposed in these
papers, the exact techniques are different because of the specific problem
setup.

\subsubsection*{General Purpose Data Analysis Tools}
Our work is also related to work on building general purpose data analysis tools
on top of databases. For example, MADLib \cite{DBLP:conf/VLDB/HellersteinRSWF12}
implements various analytic functions inside the database. MLBase
\cite{DBLP:conf/CIDR/KraskaTDGFJ2013} provides a platform that allows users to
run various machine learning algorithms on top of the Spark system
\cite{DBLP:conf/SCC/ZahariaCFSS10}.
Similarly, statistical analysis packages such as R, SAS and Matlab could also be
used to perform analysis similar to \SeeDB.


\subsubsection*{Multi-query optimization}
Since \SeeDB\ must execute a large number of queries, there are several
opportunities for performing multi-query optimization and we explore some of
these strategies in Section \ref{subsec:seedb_backend} and build an analytical model of
\SeeDB\ performance. For this, we draw upon work in the areas of multi-query
optimization and modeling parallel query execution \cite{DBLP:conf/VLDB/WuCHN13,
DBLP:journal/TODS/Sellis1988, DBLP:conf/VLDB/ZukowskiHNB07}.



