\section{\SeeDB Execution Engine}
\label{sec:optimizer}

As mentioned in the previous section, the \SeeDB execution engine is responsible for
evaluating the entire collection of potential visualizations and returning the top-$k$
visualizations in real-time.
To do so, it must apply aggressive optimization to minimize the number of queries to the
DBMS and avoid evaluation of low-utility views.

\subsection{Basic Implementation}
\label{sec:basic_implementation}

In the basic implementation of \SeeDB, for each aggregate view, \SeeDB generates
a SQL query corresponding to the target
and reference view, and issues
the two queries, one at a time, to the underlying DBMS.
It repeats this process for each aggregate view.
As the results are received, \SeeDB can compute the
distance between the target and reference view
distributions, and identify the $k$ visualizations
with highest utility. 

Naturally, this basic implementation has many inefficiencies.
In a table with $a$ dimensions, $m$ measures, and $f$ aggregation functions, 
$2\times f \times a \times  m$ queries must be executed independently.  
As we show in Section~\ref{sec:experiments}, this can take >100s for
large data sets (with hundreds of attributes and millions of rows).
These latencies are unacceptable for interactive use.
Next, we detail our {\em sharing-based} and {\em pruning-based} optimizations
that enable \SeeDB to respond to users in real-time.
\mpv{Remove:
In this paper, we propose two different suites of optimizations to deal with these
inefficiencies.
The first type of optimizations, discussed in Section~\ref{sec:sharing_opt}, involve work {\em sharing},
i.e., combining view-computation queries as much as possible.
The second type of optimizations, discussed in Section~\ref{sec:pruning_opt}, involve {\em pruning}, where some aggregate views are not completely evaluated over the whole data set.
 The second suite of optimizations
are approximate, in that they, since they use utility estimates, there is a small likelihood that the returned visualizations may not have the highest utility.}

%!TEX root=document.tex

\section{{\large \VizRecDB\ } Execution Engine}\label{sec:dbms-exec-engine}
In this section, we discuss the design and implementation of the \VizRecDB\
execution engine. 
The input to the \VizRecDB\ engine is a set of view stubs (triples of the form
$(a, m, f)$, where $a$ is the dimension attribute, $m$ is the measure,
and $f$ is the aggregation function) and the output is the top-$k$ views with the highest utility.
The goals of the \VizRecDB\ engine are two fold:
(1) to efficiently compute the utility of a large number of views, and 
(2) to accurately rank views in order of
utility to find the $k$ views with highest utility.


As discussed in the architecture overview, we explore two distinct
implementations of the \VizRecDB\ execution engine.
Our first design implements the \VizRecDB\ engine as a wrapper on top of a
traditional database system.
This design enables \VizRecDB\ to be used unchanged with a variety of
existing DBMSs.
% Our goal is to study how far we can push existing systems to
% support a \VizRecDB-type workload 
% without making any changes to the underlying DBMS.
Implemented as a wrapper over a DBMS, \VizRecDB\ is limited to using the API
exposed by the DBMS; essentially, 
\VizRecDB\ is limited to opening/closing connections to the
database and executing one or more SQL queries. 
Since we have no control over how queries are executed or access to intermediate
results, {\it our optimizations minimize total execution time by minimizing the
total number of scans of the underlying table}.
% We now discuss the basic framework used by \VizRecDB\ to query the DBMS and various
% optimizations supported by \VizRecDB.

\subsection{Execution in an RDBMS}
\label{sec:basic_framework}
Given a set of view stubs provided to the execution engine, conceptually,
the basic approach proceeds as follows:
(1) for each view, we generate SQL queries for the target and
comparison view---recall that these are aggregate queries, where
the target view is the aggregate query corresponding 
to the visualization being considered, and the comparison
view is the aggregate query that is being compared against,
for the purpose of utility computation;
see Section \ref{sec:problem_statement}, 
(2) we execute each view query independently on the DBMS, 
(3) the results of each query (i.e. the aggregated values) are processed and
normalized to compute the target and comparison distributions, 
(4) we compute utility for each target view 
and select the top-$k$ views with the largest utility,
which are then sent to the frontend.
If the underlying table has $a$ dimension attributes, $m$ measure attributes,
and $f$ aggregation functions, $2\times f \times a \times  m$ queries must be separately executed and their results
processed. Even for modest size tables (1M tuples, $f = 1, a = 50, m=5$), this
technique takes prohibitively long (500s on a row store; Section
\ref{sec:experiments}).
The basic approach is clearly inefficient since it examines every possible view and executes each view
query independently.

\subsection{RDBMS Optimizations} 
\label{sec:dbms_optimizations}
The basic approach discussed above performs poorly because it performs {\it 2
$\times$ number of views} scans of the underlying table. We can do much better
if we can use a single scan to evaluate multiple views simultaneously.
In fact, our ideal operator $\mathcal{O}$ would perform a single scan of the
data, compute all $a \times m \times f$ query results in one pass, and return
only the top-$k$ views.
Current database systems do not support this functionality.
A handful of systems have recently introduced the SQL GROUPING
SETS\footnote{GROUPING SETS allow the simultaneous grouping of query results by
multiple sets of attributes.} functionality that could be used to support this
operation. However, there are a few issues with this: the grouping sets
operator will not scale to tables with a large number of attributes because the 
size of the eventual result maintained in memory will be $2 \times a \times f
\times m$ attributes, multiplied by size of an aggregate distribution.
Furthermore, as we show in Section \ref{sec:in_memory_execution_engine},
evaluating all views for the entire dataset is unnecessary and inefficient. 
We can achieve much better performance by aggressively pruning low-utility
views on the fly.

Without this ideal operator $\mathcal{O}$,
our optimizations minimize table scans in two ways: (1) minimize the total number of
queries by intelligently combining queries, and (2) reduce the total
execution time by running queries in parallel. 
Our DBMS-backed execution engine for \VizRecDB\ is agnostic towards the
particular DBMS used to execute queries.
However, because of significant differences in the way that row stores and
column stores organize data, some of the optimizations described below
will be more powerful in row stores and may actually hurt performance in column
stores. In our experimental evaluation (Section \ref{sec:experiments}), we study
the relative advantages of each of our optimizations for row and column stores.

The optimizations supported by \VizRecDB\ are described below.

\subsubsection {Combine Multiple Aggregates} 
A large number of view queries have the same group-by attribute but different
aggregation attributes. 
Therefore, \VizRecDB\ combines all view queries with the same
group-by attribute into a single, combined view query. For instance, instead of executing
queries for views $(a_1$, $m_1$, $f_1)$, $(a_1$, $m_2$, $f_2)$ \ldots $(a_1$, $m_k$, $f_k)$
independently, we can combine the $n$ views into a single view represented by
$(a_1, \{m_1, m_2\ldots m_k\}$, $\{f_1, f_2\ldots f_k\})$. We demonstrate later
on (Section \ref{sec:expts_dbms_execution_engine})  that this optimization
offers a speed-up roughly proportional to the number of measure attributes.

\subsubsection {Combine Multiple Group-bys}
\label{subsec:mult_gb}
  Similar to the multiple aggregates optimization, another optimization
  supported by \VizRecDB\ is to combine queries with different group-by attributes
  into a single query with multiple group-by attributes.
  For instance, instead of executing queries for views $(a_1$, $m_1$, $f_1)$,
  $(a_2$, $m_1$, $f_1)$ \ldots $(a_n$, $m_1$, $f_1)$ independently, we can
  combine the $n$ views into a single view represented by $(\{a_1, a_2\ldots
  a_n\}$, $m_1$, $f_1)$ and post-process results.

Unlike the previous optimization, where the speed-up is proportional to the
number of view queries combined, in this case, the situation is not as straightforward. 
The reason for this is the following:
since we now store the aggregate for every combination
of $a_1, a_2, \ldots, a_n$, 
the number of aggregates that need to be recorded for 
$n$ views is the number of distinct combinations
of $(a_1, \ldots, a_n)$, which, in the worst case, 
is proportional to $\prod_i |a_i|$.
Thus, the number of aggregates that need to be recorded 
grows exponentially (in the worst case) in the
number of group-by attributes. 
We will show (Section \ref{sec:experiments}) that 
the time to execute a query with multiple group-by attributes does indeed
depend on the number of distinct values present in the resulting 
combination of dimension attributes. 
From a DBMS perspective, this is expected because keeping track of a large
number of aggregates impacts computational time (e.g. for sorting in sort-based aggregate)
as well as temporary storage requirements (e.g. for hashing in hash-based
aggregate) making this technique ineffective for large number of values.

Consequently, when we combine group-by attributes, we must ensure that the
number of distinct values remains {\it small enough}, below a specific 
threshold $\tau$, that we determine based on system parameters.
For simplicity, we ignore the correlations between dimension attribute values,
and work with worst-case estimates. 
The upper limit on the number of distinct values for a given combination of
group-by attributes is given by the product of the number of distinct values
for each attribute.
For example, if we combine three dimension attributes $a_i$, $a_j$ and $a_k$
with $|a_i|$, $|a_j|$ and $|a_k|$ distinct values respectively, the maximum number of
distinct groups is $|a_i|\times |a_j| \times |a_k|$.
 % The number of distinct groups in turn depends on the correlation between
 % values of attributes that are being grouped together. 
 % For instance, if two
 % dimension attributes $a_i$ and $a_j$ have $n_i$ and $n_j$ distinct values
 % respectively and a correlation coefficient of $c$, the number of distinct
 % groups when grouping by both $a_i$ and $a_j$ can be approximated by
 % $n_i$$\ast$$n_j$$\ast$(1-$c$) for $c$$\neq$1 and $n_i$ for $c$=1 ($n_i$ must
 % be equal to $n_j$ in this case).
  
Therefore, our problem can be stated as follows:
\begin{problem}[Optimal Grouping Optimization]
{\em Given a set of dimension attributes $A$ = \{$a_1$\ldots$a_n$\}, divide the
dimension attributes in $A$ into groups $A_1, \ldots, A_l$ (where $A_i$ is some
subset of $A$ and $\bigcap A_i$=$A$) such that the worst-case number of distinct values
for each group is below $\tau$.}
\end{problem}
Notice that the problem as stated above is isomorphic to the NP-Complete
{\em bin-packing} problem~\cite{garey}: to see this, we let each dimension attribute
$a_i$ correspond to an item in the bin-packing problem with weight $\log (|a_i|)$,
and we set the threshold on the bin size to be $\log \tau$,
then packing items into bins is identical to finding groups $A_1, \ldots, A_l$,
such that the worst-case number of distinct values is below $\tau$.
Thus, the problem as stated above is NP-Hard.

We use the standard first-fit algorithm~\cite{first-fit} to find the optimal
grouping of dimension attributes.
The first-fit algorithm works as follows:
on initialization, all groups $A_1, \ldots, $ $A_l$ are empty.
  For each dimension attribute, considered in an arbitrary order, place it in the first group
  $A_i$ that it can ``fit into'', i.e., the worst case
  number of distinct values for that $A_i < \tau$.
  For bin-packing, this algorithm has a guarantee of using up to 1.7X more 
  bins than necessary: here, this translates to up to 1.7X more groups than necessary.
Notice that since this grouping is independent of the input query, we can
perform grouping offline using more computationally expensive techniques as
well.

In Section \ref{XXX} we show that this optimization gives us a speedup of XXX.

% Given that the problem is NP-Hard, we use two strategies to group dimension attributes;
% the first strategy is adapted from the standard first-fit approximation algorithm~\cite{first-fit}
% for bin-packing; the second strategy is adapted from Huffman coding~\cite{huffman-codes}.
  
% \squishlist
% \item {\bf First-Fit}: The algorithm is simple;
% at first, all groups $A_1, \ldots, $ $A_l$ are empty.
% For each dimension attribute, considered in an arbitrary order, place it in the first group
% $A_i$ that it can ``fit into'', i.e., the worst case
% number of distinct values for that $A_i < \tau$.
% For bin-packing, this algorithm has a guarantee of using up to 1.7X more
% bins than necessary: here, this translates to up to 1.7X more groups than necessary.
  % In this trategy, we set an upper limit $V$ on the
  % number of distinct groups that any combination of dimension attributes can
  % produce.
  % Now, suppose that each dimension attribute $a_i$ $\in A$ has $n_i$ distinct
  % values.
  % We cast the problem of finding the optimal grouping of dimension attributes as
  % a version of bin-packing.
  % Let each dimension attribute $a_i$ be an item with volume $log(n_i)$ and
  % suppose that we are given bins of volume $log(V)$.
  % Then finding the optimal grouping of attributes is exactly equivalent to
  % finding the optimal packing of attribute items into the minimum number of
  % bins.
%   \item {\bf Huffman-Grouping}: Here, the algorithm works as follows:
%   we start by assigning each dimension attribute its own group,
%   and maintain these groups in sorted ascending order.
%   At each step, we take the two smallest groups out of this sorted list, combine
%   the dimension attributes in both of them into one new group, and then
%   add this new group to the mix and re-sort.
%   We keep doing this until we cannot combine the two smallest groups any more
%   without violating the threshold $\tau$.
% \squishend
% If this was vanilla bin-packing, the first-fit algorithm would be sufficient to
% find near-optimal groupings. 
% The advantage of having the huffman-grouping algorithm is the following: Instead
% of having a hard threshold $\tau$ on the worst-case number of distinct groups,
% the huffman-grouping algorithm can also adapt to the case where we
% are instead provided a black-box function that, given a particular
% grouping of attributes, returns a value for the ``goodness'' of the combination.
% As we will see in Section~\ref{sec:analytical_model}, 
% we develop an analytical model for the performance (i.e., the execution time)
% of the DBMS-backed execution engine. 
% This model can be used to derive this black-box function.
% Given a black-box function, the huffman-grouping algorithm operates
% as follows: until the ``goodness'' of the grouping keeps increasing 
% (i.e., the predicted execution time keeps decreasing),
% repeatedly combine the two smallest groups into a single group.
% We experimentally evaluate both strategies in Section \ref{sec:experiments}.
% 
% \srm{In 6.2.2., we mention the model.    Are we keeping this?  I think it could be ok to put the details in the appendix if we can show one graph that illustrates that the model can be used to predict how many groups to choose, etc.}

\subsubsection{Combine target and comparison view query}
\label{subsec:target_comparison_view}
Since the target view and comparison views only differ in the subset of data
that the query is executed on, we can easily rewrite these two view queries as
one. For instance, for the target and comparison view queries $Q1$ and $Q2$
shown below, we can add a group by clause to combine the two queries into $Q3$.
\begin{align*} 
Q1 = &{\tt SELECT \ } a, f(m) \ \ {\tt FROM} \  D\  {\tt WHERE \ \ x\ <\ 10\
GROUP \ \ BY} \ a \\
Q2 = &{\tt SELECT \ } a, f(m) \ \ {\tt FROM} \  D\  {\tt GROUP \ \ BY} \ a \\
Q3 = &{\tt SELECT \ } a, f(m), {\tt CASE\ IF\ x\ <\ 10\ THEN\ 1\ ELSE\ 0\
END}\\ 
&as\ group1,\ 1\ as\ group2\\ 
&{\tt FROM} \ D\ {\tt GROUP \ \ BY} \ a,\ group1,\ group2
\end{align*}
% This rewriting allows us to obtain results for both queries in a single table
% scan. The impact of this optimization depends on the selectivity of the
% input query and the presence of indexes. When the input query is less selective,
% the query executor must do more work if the two queries are run separately. In
% contrast, in the presence of an index, running selective queries independently
% may be faster.
% \srm{In 4.2.3., I don't understand why the combined query wouldn't always be faster.  It seems like you are replacing two scans with one, which should just be better.  How could it not be (unless there are very selective filter predicates that aren't show in the example???)}
% \agp{I agree.}

  \subsubsection {Parallel Query Execution}
  \label{subsec:parallel_exec}
  While the above optimizations reduce the number of queries executed, we can
  further speedup \VizRecDB\ processing by executing view queries in parallel. When
  executing queries in parallel, we expect co-executing queries to share pages in the
  buffer pool for scans of the same table, thus reducing disk accesses and
  therefore the total execution time. 
  However, a large number of parallel queries can lead to poor performance for
  several reasons including buffer pool contention, locking and cache line
  contention \cite{Postgres_wiki}. 
  As a result, we must identify the optimal number of parallel queries for our workload.
  
  % We do observe a reduction in the
  %overall latency when a small number of queries are executing in parallel;
  % however, the advantages disappear for larger number of queries running in
  % parallel. We discuss this further in the evaluation subsection.
\subsubsection{Other Optimizations}
To further speedup \VizRecDB we can pre-compute various partial results.
For example, in the case where our comparison view is constructed from the
  entire underlying table (Example 1 in Section \ref{sec:introduction}),
  comparison views are the same irrespective of the input query.
Therefore, we can precompute all possible comparison views once and store
  them for future use. 
%   comparisons. If the dataset has $a$ dimension and
%   $m$ measure attributes, 
%   pre-computing comparison views would add $a \times m$
%   tables. This corresponds to an extra storage of $O(a\times m \times n \times f)$ where $n$
%   is the maximum number of distinct values in any of the $a$ attributes,
%   and $f$ is the number of aggregation functions. 
%   Note that pre-computation cannot be used in situations where the comparison
%   view depends on the target view (Example 2) or is directly specified by the
%   user (Example 3).
%   
Similarly, if we precompute a sample of the table being queried, we can run
all view queries against this table to pick the top-$k$ views and only evaluate
those views on the entire dataset.
We do not experimentally evaluate these optimizations because while these
problems reduce the computation required by \VizRecDB, the
challenges related to evaluating a large number of views (whether limited to
target views or limited to a sample) remain unaddressed.


%!TEX root=document.tex


\subsection{Pruning-Based Optimizations}
\label{sec:in_memory_execution_engine}
The previous section described techniques to
batch and share computations across queries---however,
if the number of queries is large, this could still 
lead to high execution times, especially
because a lot of resources
are wasted on computing low-utility views. 
In this section, we describe mechanisms to prune away
some aggregate views without complete evaluation. 
% Naturally, pruning schemes are {\em approximate},
% in that they sometimes incorrectly prune away some
% high utility visualizations. 
% That said, the visualizations that are displayed
% may still be close to high utility 
% (which may be good enough for analysts) 
% and will certainly be
% correct; thus there is still merit to considering approximate schemes.



% The DBMS-backed execution engine from the previous section provides reasonable performance for small datasets.
% However, we find that for medium and large size datasets, the optimized engine take tens to hundreds of produce the top visualizations.
% There are a few reasons for this. (a) Since our engine runs a large number (50-200) queries for each \SeeDB invocation, a large number of scans of the data are performed needlessly; (b) Far too many resources are wasted on low-utility views; and (3) Results of \SeeDB are not available until the entire dataset has been processed.

% To mitigate these drawbacks, we explored a set of pruning strategies to identify and then discard low utility views.
% Pruning low-utility views can allow \SeeDB to reduce the number of queries run on the database, reduce resources spent on eventually-discarded views, and give \SeeDB the ability to return as soon as it has identified the top views.
% In order to evaluate various pruning strategies, we built a simple framework to process records sequentially and perform pruning on the fly.
% We now decribe our test framework and two specific pruning strategies adopted in \SeeDB.

\stitle{Basic Pruning Framework.}
\label{subsec:basic_framework}
\SeeDB employs two kinds of optimizations to eliminate low-utility views
and redundant views respectively.
The pruning of low-utility views is performed {\em online}, i.e., after the
query is provided,
while the pruning of redundant views is performed {\em offline},
before any queries have been provided.
Our primary focus in this section is the online view pruning;
we will only briefly describe offline view pruning.

Our online view pruning approach involves querying the underlying tables in
multiple {\em phases};
at the end of each phase, we evaluate each aggregate view,
and optionally prune a number of views, and then keep the rest.
This evaluation and pruning is certainly easy to implement if we were to push these
decisions ``within'' the database layer; however, since
our goal is to operate outside the database, 
% without making
% any significant changes to the database, 
we need some pre-processing
of the underlying tables.
Specifically, we partition our table horizontally into a number of partitions
such that processing each partition corresponds to one of our phases. 
Since we have a collection of these partitions, we can 
choose to operate on them in any order; if further randomness
is needed, we can preprocess each partition such that the records are internally
in random order.
After each partition is processed, we update our candidate and target
views for each aggregate view, 
compute the resulting utility, and then 
prune (i.e., discard) or accept (i.e., select to be in those to be displayed) some aggregate views.
When only $k$ aggregate views remain (or $k$ are accepted), we can choose to display them
to the analyst (if approximate visualizations are permissible),
or we can completely compute the visualizations for those $k$ aggregate views.
\techreport{Algorithm~\ref{algo:custom_exec_engine} shows the outline of our framework.}
We now describe two different schemes used by \SeeDB for pruning low-utility views.

\reviewer {
	D2.5 The horizontal partitioning is never detailed: how is it done? How is the
number of fragments decided?
}
%We evaluate both of these schemes in our experiments.

Note that the two pruning schemes described below have guarantees
in other settings that do not directly carry over to our setting.
In our evaluation, we show that in spite of this limitation, the pruning schemes
work rather well in practice. 
We can, however, show that as we sample more and more, the estimated utility
$\hat{U}$ can be made to be arbitrarily close to $U$ for all aggregate views.
We can state our claim formally in the following lemma. 
\papertext{The proof based on 
Hoeffding's inequality can be found in the extended
technical report ~\cite{seedb-tr}.}
\techreport{The proof of this lemma is presented in Section \ref{sec:convergence}.}
% At a high level, the proof
% involves repeated applications of Hoeffding's inequality to
% upper and lower-bound $\hat{U}$ within $U$ along with terms 
% that tend to $0$ as the number of samples increases.

\begin{lemma}
Let the target and comparison visualizations
both have $m$ groups.
Let $\hat{U}$ denote our estimate of $U$ based on a uniformly random sample 
across all $m$ groups. 
Then, as the number of samples tends to $\infty$, $\hat{U} \rightarrow U$
with probability $1-\delta$, for as small $\delta$ as needed.
\end{lemma}

\techreport{
\begin{algorithm}[h]
\caption{Pruning Framework}
\label{algo:custom_exec_engine}
\begin{algorithmic}[1]
\State viewsInRunning $\gets$ \{all views\}
\State currPhase $\gets$ 0
\While {currPhase.hasNext()}
\State processNextPartition()
%\State updateUtilityEstimates()
\If {currPhase.End()}
\State pruneViews(viewsInRunning)
\State currPhase.Next()
\EndIf
% \If {stoppingCondition.True()}
% \State break
% \EndIf
\EndWhile
\State return viewsInRunning.sort().getTopK()
\end{algorithmic}
\end{algorithm}
}

% Note that we have two alternatives for computing the visualizations for the top-$k$
% views, once $k$ views have been accepted: 
% we can either compute the top-$k$ views to completion (i.e.,
% on the remaining unprocessed records),
% or we can display approximate visualizations for each view, as soon as we ``accept'' them
% as being in the top-$k$. 
% We use the former option as the default, 
% and the latter as an additional optimization that may be employed.
% If at the end of a given phase, \SeeDB\ finds that the views in running satisfy
% the given stopping criteria, e.g., that \SeeDB\ has already identified the
% top-$k$ views, the \SeeDB\ engine can stop processing early.

% \begin{algorithm}
% \caption {ComputeAggregate(Query
% {\it currQuery}, int $d$)}
% %\small 
% \begin{algorithmic}[1] 
% \State int[$d+1$] $currCard$\ \ //All arrays are
% indexed from 1 
% \STATE $currCard[1]$ = ExecuteCellQuery({\it currQuery})  
% \FOR {$i=2$ to $d+1$} 
% \STATE {\it prevQuery} $\leftarrow$ GetPreviousNeighbour($i$-1)\ \ //decrement
% the $(i-1)^{th}$ dimension of {\it currQuery} by stepsize 
% \STATE int[] $prevCard$ = GetAllAggregates({\it prevQuery})
% \STATE $currCard[i]$ = $currCard[i-1]$ + $prevCard[i]$ 
% \ENDFOR 
% \STATE StoreAllAggregates({\it currQuery}, $currCard$) 
% \STATE {\bf return} $currCard[d+1]$
% \end{algorithmic}
% \label{algo:aggregatecomputation}
% % \end {algorithm}
% \mpv{should we remove this?}
% \stitle{Pruning Techniques}: The pruning techniques we discuss next enable \SeeDB\ to discard low-utility views
% early, thus saving computational resources and potentially allowing \SeeDB\ recommend
% views even before all records have been processed.
% To understand the intuition behind the pruning techniques, recall that we define utility of
% view $V_j$ as $ U (V_i) = S ( P[V_i (D_Q)], P[V_i (D)] )$ where $S$ is a distance function measuring
% the distance between the target and comparison distributions, $P[V_i (D_Q)]$ and $P[V_i (D)]$.
% Let the total number of records in the input be denoted by $R$ and the total records
% read by phase $j$ as $R_j$. 
% Once $R_j$ records have been read in, we can estimate target and comparison 
% distributions, $\widetilde{P}^j[V_i (D_Q)]$ and $\widetilde{P}^j[V_i (D)]$.
% As \SeeDB\ processes larger portions of the file, these estimates become more accurate. 
% As one might expect:

% $$\lim_{R_j \rightarrow R}\widetilde{P}^j[V_i (D_Q)] \rightarrow P[V_i (D_Q)] \cap 
% \widetilde{P}^j[V_i (D)] \rightarrow P[V_i (D)] $$

% metric discussed in Section \ref{} are of the form $\mathcal{F}:
% (\mathcal{R}^p, \mathcal{R}^p) \rightarrow \mathcal{R}$, i.e. they are functions
% measuring the distance between the target and comparison distributions of a view.
% For view $V_j$ let $\mathcal{P}^j_T$ and $\mathcal{P}^j_C$ respectively denote the 
% target and comparison view of $V_j$.

% Therefore, as we process records in every phase $j$, we can keep running estimates of the
% view distributions as well as view utilities, $\widetilde{U}^j(V_i)$.
% Moreover, {\it we can use these estimates to perform the pruning of low utility views}.
% In this paper, we propose and evaluate two different strategies
% to perform the pruning of views based on utility estimates. 
% The first strategy is based on top-$k$ algorithms that use confidence intervals 
% and the second is based on an adaptation of the Multi-Armed Bandit problem.

% of the view distributions $\widetilde{\mathcal{P}}^{i,j}_T$ and $\widetilde{\mathcal{P}}^{i,j}_C$
% and estimates of the utility $\mathcal{F}(\widetilde{\mathcal{P}}^{i,j}_T, 
% \widetilde{\mathcal{P}}^{i,j}_C)$.

% The general idea is to keep running estimates of utility for each view, and
% perform pruning of low utility views based on these estimates.
% To implement a pruning strategy, we merely specify two things: (1)
% statistics to track for each view, and 
% (2) the rule used to prune views at the end of a phase.
% An important side effect of our implementation is that
% as we scan more data from the file, our estimates of utilities become more
% accurate.

% 
% \begin{itemize}
% \item The first two strategies are based on confidence interval-based top-$k$ algorithms.
% That is, at the end of each phase, confidence intervals are updated for all
% the views still in the running and we prune views based on overlap
% with the top-$k$ confidence intervals.
% % and if the upper-bound for any of them
% % is lower than the lower-bound for confidence intervals of $k$ or more other views,
% % then those views are discarded.
% \begin{denselist}
% \item The first uses worst-case confidence intervals based on the
% Hoeff\-ding-Serfling inequality~\cite{serfling1974probability}, 
% which is a generalization of Hoeffding's inequality~\cite{hoeffding1963probability}
% when a number of samples are randomly chosen without replacement
% from the same set.
% These confidence intervals are necessarily more conservative.
% \item The second assumes that the underlying probability distribution is
% Gaussian and uses 95\% confidence intervals~\cite{all-of-statistics}.
% These confidence intervals are more ``aggressive'' compared to those derived
% from the Hoeffding inequality.
% (For experiments evaluating this assumption, see Section~\ref{sec:evaluating_normal}.) 
% \end{denselist}
% \item The last strategy is based on an adaptation of Multi-Armed Bandit (MAB)
% algorithms for top-$k$ arm identification~\cite{}. 
% Multi-Armed Bandit algorithms are well-studied in the field of stochastic
% control; in our scenario, the arms are the views, and pulling an arm corresponds
% to updating a view's utility.
% \end{itemize}


\stitle{Confidence Interval-Based Online Pruning.}
\label{sec:confidence_interval}
Our first pruning scheme uses worst-case statistical confidence intervals.
This technique is similar to top-k based pruning algorithms developed 
in other contexts as described in ~\cite{DBLP:conf/pods/FaginLN01, 
DBLP:conf/vldb/IlyasAE04, DBLP:conf/ICDE/ReDS07}.
Our scheme works as follows: during processing each partition,
we keep an estimate of the mean utility for every aggregate view $V_i$ and a
confidence interval around that mean.
That is, for every view $V_i$ we track its mean utility $u_i$, and a
confidence interval around the mean, $u_i \pm c_i$. 
At the end of a phase, we use the following rule to prune low-utility
views:
{\em If the upper bound of the utility of view $V_i$ is lesser
than the lower bound of the utility of $k$ or more views, then $V_i$ is discarded.}

\begin{figure}[h]
\vspace{-10pt}
\centerline{
\hbox{\resizebox{8cm}{!}{\includegraphics[trim=10mm 100mm 55mm 35mm, 
clip=true]{Images/confidence_pruning.pdf}}}}
\vspace{-20pt}
\caption{Confidence Interval based Pruning}
\label{fig:conf_interval}
\vspace{-15pt}
\end{figure}

\reviewer {
	V5 is extraneous
}

To illustrate, suppose a dataset has 4 views $V_1$ to $V_4$ and we want to find the top-$2$ views.
Further suppose that at the end of phase $p$,
$V_1$-$V_4$ have confidence intervals as shown in Figure \ref{fig:conf_interval}.
Views $V_1$ and $V_2$ have the highest estimates for utility so far.
Consider view $V_3$; we see that its confidence interval overlaps with the
confidence intervals of the current top views, making it possible
that $V_3$ will be in the final top views. On the other hand, the confidence
interval for $V_4$ lies entirely below the lowerbounds of $V_1$ and $V_2$.
Since we can claim with high probability
that the utility of $V_4$ lies within its confidence interval, it follows that
with high probability, $V_4$'s utility will be lower than that of both $V_1$ and
$V_2$, and it will not appear in the top-$2$ views.
\papertext{Pseudocode for our pruning scheme can be found in our technical report~\cite{seedb-tr}.}
\techreport{We state the algorithm formally in
Algorithm~\ref{algo:ci_based_pruning}.}

\techreport{
\begin{algorithm}
\caption{Confidence Interval Based Pruning}
\label{algo:ci_based_pruning}
\begin{algorithmic}[1]
\State viewsInRunning.sortByUpperbound()
\State topViews $\gets$ viewsInRunning.getTopK()
\State lowestLowerbound $\gets$ min(lowerbound(topViews))
\For {view $\not \in$ topViews}
\If {view.upperbound < lowestLowerbound}
\State viewsInRunning.remove(view)
\EndIf
\EndFor
\end{algorithmic}
\end{algorithm}
}

We use {\it worst case} confidence intervals as derived from
the Hoeffding-Serfling inequality~\cite{serfling1974probability}.
The inequality states that if we are given $N$ values $y_1, \ldots, y_N$ in 
$[0, 1]$ with average $\mu$, and we have have drawn $m$ values without replacement, $Y_1, \ldots, Y_m$, 
then we can calculate a running confidence interval around the current mean 
of the $m$ values such that the actual mean of the $N$
is always within this confidence interval with a probability of $1 - \delta$:
\begin{theorem}
\label{thm:hs}
% Let $\calY = y_1,$ $\ldots,$ $y_N$ be a set of $N$ 
% values in $[0,1]$ with average value
% $\frac1N \sum_{i=1}^N y_i = \mu$.
% Let $Y_1,\ldots,Y_N$ be a 
% sequence of random variables drawn from $\calY$ without
% replacement.
Fix any $\delta > 0$. For $1 \le m \le N-1$, define
{\small $$
\varepsilon_m = \sqrt{\frac{(1-\frac{m-1}N)(2\log \log (m) + \log(\pi^2/3\delta))}{2m}}.
$$
$$
\textrm{Then:} \ \   \Pr\left[ \exists m, 1 \le m \le N : 
  \left|\frac{\sum_{i=1}^m Y_i}{m} - \mu\right| > \varepsilon_m \right] 
\le \delta.
$$
}

\end{theorem}
In our setting, each $Y_i$ corresponds to the an estimate of utility computed based on the
records seen so far. 
% estimates of utility that we 
% have obtained based on the set of records seen thus far. 
% Therefore, to apply this pruning strategy, we track the current estimates of
% view utilities at each step and use the above confidence interval calculation
% to perform pruning at the end of every phase.

% Note that in this setting, we are assuming that since the
% utility estimate at any stage of processing is in $[0, 1]$, 
% the $Y_i$ values, i.e., the incremental contributions to the utility
% that come from reading each record, are also between $[0, 1]$,
% and are independent of the current value of the utility. 
% This is is not true in our setting, 
% because the utility function could be arbitrary.
% Thus, the theoretical guarantees do not directly apply to our setting. 

% \stitle{Normal Confidence Intervals.} In this scheme, we assume that the utility
% distributions for each view are Gaussian and apply the standard confidence
% intervals to our utility measurements.
% % We describe the equations first assuming that
% % when every time a record is read, for every view,
% % a utility value is ``sampled''
% % from a normal distribution. (This assumption is not
% % quite correct; we will discuss this  below.)

% Consider a specific view $V_i$. 
% If the mean utility across the sampled records 
% (i.e., the records read thus far) is $\mu$,
% and the variance in the utility of the sampled records
% is $\sigma$, then, we have:
% \begin{align}
% CI & = \mu \pm z \times \frac{\sigma}{\sqrt{m}}
% \end{align}
% Thus, the CI (or confidence interval) is 
% a confidence interval centered around $\mu$, 
% and depends on $\sigma$. 
% It additionally depends on the number of records
% read thus far, $m$,
% and $z$, the factor that depends on our confidence interval threshold.
% For instance, for a 95\% confidence interval, $z = 1.96$.

% We note that the assumption that we are drawing from a normal distribution is
% not quite accurate since our samples vary in size and are not independent.
% As a result, we make two simple adjustments to the confidence interval
% calculations that are described in Appendix~\ref{sec:ci_pruning}.
% In Section \ref{sec:experiments}, we show experimentally on multiple datasets
% that our confidence interval calculations accurately capture utility and can be used to
% perform pruning with high confidence.






% \stitle{Normal Confidence Intervals.}
% As described above, we must specify
% a set of statistics to track for each view and a rule that is used to
% prune views based on the statistic.
% For confidence interval based pruning, the statistics we track are the mean, 
% variance, and confidence intervals of the view utility.
% As \SeeDB\ reads each record, it updates the data
% distributions for all views and calculates the current utility of each view. 
% Using past measures of utility, \SeeDB\ also tracks the mean,
% variance and confidence intervals for the utility of each view.
% % At the end of a phase, \SeeDB\ uses the following rule for pruning low-utility
% % views (stated more formally below): {\it if the upperbound on the utility
% % of view $v_i$ is lesser than the least lowerbound on the utility of the
% % top-$k$ views, view $v_i$ is discarded.}

% % Let us dive deeper into this pruning rule.
% Note that as we sequentially read records from a file, we are
% approximating a sampling process (remember that the records are in random order).
% For instance, suppose that we have read 10K records from a 1M record file.
% In this case, the records 1 -- 10K constitute a 1\% sample of the entire file.
% When we read the next say 10 records, the records 1 -- 10,010 constitute an
% incrementally larger sample of the underlying file.
% Thus, as we read more data from the file, we obtaining a large
% number of samples from the underlying data (notice however, that these samples
% are not independent).

% Since we are generating a large number of samples from a population, we can
% invoke a well-studied concept in statistics called the ``sampling distribution.'' 
% A sampling distribution for a statistic $S$ is the distribution of
% $S$ generated by taking a large number of samples of a fixed size and computing
% the statistic $S$ on each sample.
% In our case, the population we draw from is the set of all records in the file
% and our samples are the increasingly larger sets of records that we are reading in.
% The statistic $S$ that we are computing is the view utility (we
% compute a utility value for each view).
% Now, the sampling distribution of the {\it mean} has been well studied and it
% has been proven that the mean of the sampling distribution is equal to the mean of the
% population and the standard error of the sampling distribution is equal to the
% standard error of the population divided by the square root of the sample size. 
% These two formulas are shown in Equations \ref{eq:mean} and \ref{eq:variance}.
% Similarly, if we know the mean and standard error of the sampling distribution,
% we can compute a confidence interval around the population mean. This is shown
% in Equation \ref{eq:confidence_interval} where $z$ is the factor that depends on the
% confidence threshold we are aiming for and $N$ is the number of items
% in each sample.

% \begin{eqnarray}
% \label{eqnarray:mean_and_variance}
% \mu_M = \mu \label{eq:mean}\\
% \sigma_{M} = \frac{\sigma}{\sqrt{N}} \label{eq:variance}\\
% CI = \mu_M \pm z \ast \frac{\sigma_M}{\sqrt{N}}\label{eq:confidence_interval}
% \end{eqnarray}

% If we were modeling the mean of our samples instead of the utility, we could use
% the above result directly.
% However, we find that with a few minor modifications, we can use the confidence
% interval bounds shown above.
% The first modification we make has to do with how we define utility.
% Remember from Section \ref{sec:problem_definition} that the utility of a view is
% defined as the distance between two distributions: the distribution of aggregate values for the
% target view and the distribution of aggregate values for the comparison view.
% These distributions are in turn tied to the number of distinct groups present in
% each dimension attribute.
% For our purposes, it means that if a dimension attribute has $n$ distinct
% groups, then a sample with $x$ rows gives us approximately $\frac{x}{n}$ values
% for each group (assuming uniform distribution).
% Said another way, a sample with $x$ rows for the purpose of computing utility is
% really only a sample of $\frac{x}{n}$ rows.
% So the first modification we make to Equation \ref{eq:confidence_interval} is to
% replace $N$ by $\frac{N}{G_{max}}$ where $G_{max}$ is the maximum number of
% distinct groups present in any dimension attribute.
% Second, we observe that the sampling distribution applies to the case where
% samples are of the same size and are independently generated.
% This is not true in our algorithm; therefore, to compensate, make two
% conservative modifications: we set $N$ to the number of rows that
% have been read in the previous phase (remember that pruning happens at the end
% of every phase) and we set the $z$ parameter to a value $\geq$ 1.96 (the normal
% 95\% confidence interval value). These modifications ensure (as we will show
% empirically in Section \ref{sec:experiments}) that the confidence intervals
% always contain the mean and continually shrink as we read in more data.

% As shown in Line 12 of Algorithm \ref{algo:custom_exec_engine},
% when a phase ends, we clear all statistics collected in that phase; we do not
% want less accurate estimates from previous phases to contaminate the more
% accurate estimates from subsequent phases. \agp{deal with this.}






% Now that we have a way of finding confidence intervals, we elaborate on how we
% use them to perfom pruning.
% Suppose at the end of phase $p$ the confidence intervals for the views in
% running have values shown in Figure \ref{fig:conf_interval} and we want to
% identify the two views with the highest utility.
% Consider view $V_3$, we see that its confidence interval overlaps with the
% confidence intervals of the current top views $V_1$ and $V_2$, making it likely
% that $V_3$ will be in the final top views. On the other hand, the confidence
% interval for $V_4$ lies entirely below the lowest bound of the top two
% intervals.
% Since we can claim with high probability (depending on the confidence threshold)
% that the utility of $V_4$ lies within its confidence interval, it follows that
% with high probability, $V_4$ will not appear in the top-$2$ views.
% This is essentially our pruning rule. 
% 



\stitle{Multi-Armed Bandit Online Pruning.}
\label{sec:multi_armed_bandit}
Our second pruning scheme uses Multi-Armed Bandit (MAB)~\cite{bandits, AuerCF02, LaiR85} policies.
\techreport{The setting of MAB is as follows: a gambler is faced with several slot
machines (``one-armed bandit''s), each of which has an underlying 
(unknown) reward distribution. 
Every play results in a reward from the corresponding machine's
reward distribution.
The goal is to devise a {\it strategy} of which machine to play
at each turn in order to maximize the reward~\cite{bandits}.}
A recently-studied variation of MAB focuses on finding the arms with the highest
mean rewards~\cite{BubeckWV13, audibert2010best}.
This variation is similar to the problem addressed by \SeeDB: each possible view 
can be thought of as a one-armed bandit and our goal is find the views with the 
highest reward (i.e. utility). \reviewer {something werid here}.
We therefore adapt the Successive Accepts and Rejects algorithm from \cite{BubeckWV13} 
to find arms with the highest mean reward. 
\techreport{Algorithm~\ref{algo:mab_based_pruning} shows the pseudocode for our pruning technique.}
% As before, the processing of the input table is divided into phases.
% In every phase, \SeeDB\ reads in new records and updates the distributions and utilities
% for every view in the running.
At the end of every phase, all active views are ranked in order of their utility means. 
We then compute two special differences between the utility means: $\Delta_1$
is the difference between the highest mean and the $k+1$st highest mean, and
$\Delta_n$ is the difference between the lowest mean and the $k$th highest mean.
If $\Delta_1$ is greater than $\Delta_n$, the view with the highest mean is
``accepted'' as being part of the the top-$k$ (and it no longer participates
in pruning computations).
On the other hand, if $\Delta_n$ is higher, the view with the lowest mean is discarded
from the set of views in the running.
\cite{BubeckWV13} proves that under certain assumptions about reward distributions,
the above technique identifies the top-$k$ arms with high probability.

\reviewer{
	For the MultiArmed
Bandit pruning, it would be good to discuss whether
the assumptions of reward distributions would necessarily in this context.
}

% In MAB, each pull of an arm corresponds to a drawing from a sample
% the underlying probability distribution of that arm.
% In our case, each new record updates the utilities for all views and
% each resulting updated utility can be thought of as a sample from the
% utility distribution of that view.

% In applying MAB techniques to our problem setup, we make two assumptions:
% (1) although the utility of a
% view is ultimately a single value, we can approximate it as a probability
% distribution that is normally distributed around the true utility, and 
% (2) our running estimate of utility after reading $i$
% records is a sample derived from the above utility distribution.

\techreport{
\begin{algorithm}
\caption{MAB Based Pruning}
\label{algo:mab_based_pruning}
\begin{algorithmic}[1]
\State viewsInRunning.sortByUtilityMean()
\State \{$\bar{u}_{i}$\} $\gets$ sorted utility means
\State $\Delta_1$ $\gets$ $\bar{u}_{1}$ - $\bar{u}_{k+1}$
\State $\Delta_n$ $\gets$ $\bar{u}_{k}$ - $\bar{u}_{n}$
\If {$\Delta_1$ < $\Delta_n$}
\State viewsInRunning.acceptTop()
\Else
\State viewsInRunning.discardBottom()
\EndIf
\end{algorithmic}
\end{algorithm}
}





% \techreport{\cite{BubeckWV13} provides bounds on the optimality of this heuristic for the
% MAB setting.
% Since our problem setup isn't exactly the same, the optimality bounds don't
% transfer directly.
% However, as we show in the experimental section, the MAB heuristic performs well
% on real datasets.}