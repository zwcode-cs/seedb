\documentclass{sig-alternate}
%\documentclass{vldb}
%\documentstyle{article}
%\documentstyle[amsmath,amsthm,amssymb,twocolumn]{article}
\usepackage{times}

\usepackage{graphicx, array, blindtext}


\usepackage{MnSymbol} 
%\usepackage{MinionPro}
%\usepackage[mathlf,textlf,minionint]{MinionPro}
%\usepackage[T1]{fontenc}
%\usepackage{textcomp}
\usepackage{multirow}
%\usepackage{times}
\usepackage{pgfplots}
\usepackage{subfigure}
%\usepackage{amsmath,amssymb}
\usepackage{graphicx,color}
%\usepackage{verbatim}
%\usepackage{framed}
%\usepackage[ruled,vlined]{algorithm2e}

\usepackage[font={scriptsize,it}]{caption}
\usepackage{floatrow}

\begin{document}



\newcommand{\agp}[1]{\textcolor{red}{Aditya: #1}}
\newcommand{\mpv}[1]{\textcolor{blue}{Manasi: #1}}
%\newcommand{\alkis}[1]{\smallskip\noindent \textcolor{red}{\it $\Rightarrow$ Alkis: #1}}
\newcommand{\SeeDB}{{\sc SeeDB}}
\newcommand{\calQ}{\mathcal{Q}}
\newcommand{\calR}{\mathcal{R}}
\newcommand{\att}[1]{{\text{#1}}}

\newtheorem{definition}{Definition}[section]
\newtheorem{example}[definition]{Example}
\newtheorem{problem}{Problem}[section]
\renewcommand{\baselinestretch}{0.995}

%\DeclareMathOperator*{\argmax}{arg\!\max}




%\newcommand{\histvis}{\mbox{\sc HistVis}}
\newcommand{\squishlist}{
   \begin{list}{$\bullet$}
    { \setlength{\itemsep}{0pt}
      \setlength{\parsep}{2pt}
      \setlength{\topsep}{0pt}
      \setlength{\partopsep}{0pt}
      \leftmargin=25pt
\rightmargin=0pt
\labelsep=5pt
\labelwidth=10pt
\itemindent=0pt
\listparindent=0pt
\itemsep=\parsep
    }
}
\newcommand{\squishend}{\end{list}}

\newenvironment{denselist}{
    \begin{list}{\tiny{$\bullet$}}%
    {\setlength{\itemsep}{0ex} \setlength{\topsep}{0ex}
    \setlength{\parsep}{0pt} \setlength{\itemindent}{0pt}
    \setlength{\leftmargin}{0.5em}
    \setlength{\partopsep}{0pt}}}%
    {\end{list}}

\newcommand{\eat}[1]{}
\newcommand{\papertext}[1]{#1}
\newcommand{\techreport}[1]{}

\newcommand{\techreporttext}[1]{}
\newcommand{\stitle}[1]{\vspace{0.25em}\noindent\textbf{#1}}




\title{{\LARGE \sc SeeDB}: Automatically Generating Query Visualizations}
%\subtitle{\vspace{-10pt}[Vision Paper]\vspace{5pt}}



\numberofauthors{4} 
\author{
\alignauthor
\hspace{-30pt}Manasi Vartak  \\ 
\affaddr{\hspace{-30pt}MIT} \\
\affaddr{\hspace{-30pt}mvartak@mit.edu} 
\alignauthor 
\hspace{-80pt}Angela Zhang\\ 
\affaddr{\hspace{-80pt}MIT} \\ 
\affaddr{\hspace{-80pt}angelaz@mit.edu}
\alignauthor
\hspace{-105pt}Aditya Parameswaran \\ 
\affaddr{\hspace{-105pt}MIT \& U. Illinois (UIUC)} \\
\affaddr{\hspace{-105pt}adityagp@illinois.edu} 
\alignauthor 
\hspace{-130pt}Samuel Madden \\ 
\affaddr{\hspace{-130pt}MIT} \\ 
\affaddr{\hspace{-130pt}madden@csail.mit.edu} 
}

\maketitle
\begin{abstract}
%!TEX root=document.tex


Data analysts operating on large volumes of data 
often rely on visualizations to interpret the results of queries. 
However, finding the right visualization for a query is 
a laborious and time-consuming task. 
We propose \SeeDB, a system that partially automates 
this task: 
given a query, \SeeDB\ explores the space of all possible visualizations,
and automatically identifies and recommends to the analyst those visualizations
it finds to be most ``interesting'' or ``useful''.
\agp{Say something about speedup of optimizations, and also about utility}
\end{abstract}
%!TEX root=document.tex

\section{Introduction}
\label{sec:introduction}


Data visualization is one of the most common techniques for identifying 
trends and finding anomalies in data.
However, with high-dimensional datasets, identifying visualizations that 
effectively present interesting variations or patterns in the data is a non-trivial task:  a
user typically builds a large number of visualizations optimizing for a range of visualization 
types, aesthetic features, and more
%may have to plot thousands of pairs of attributes against each other 
before arriving at one that shows something valuable.

%The hope is that these recommendations, combined with the current interactivity and ease of use of visualization 
%software, can help analysts quickly glean insights form the data.

In this paper, we tackle the problem of automatically recommending such valuable visualizations.
The problem of recommending visualizations is complicated because a ``good'' visualization needs to take into account several different dimensions, including:
\begin{inparaenum}
\item types and properties of the different attributes in the dataset ({\it metadata}), 
\item visual qualities of the visualization ({\it aesthetics}). 
\item the particular subset of data the user is interested in ({\it query}), 
\item variation in and distribution of the data itself ({\it data distribution}), 
\item meaning and relative importance of attributes in the data ({\it semantics}),  and
\item past history of interactions with this user and other users ({\it user preferences})
\end{inparaenum}

Current visualization systems like Spotfire and Tableau have limited capabilities to recommend visualizations --- they only focus on metadata and aesthetics dimensions, following standard visualization best-practices, e.g., choice of appropriate colors,
rules defining when a bar chart is more appropriate than a trend line, etc.  
No existing system that we are aware of 
incorporates insights about the underlying data, or for that matter, any of the other dimensions into its recommendations.

%There is much room for research into leveraging information along the remaining dimensions to make more holistic
%recommendations.



%Data analysts must sift through very large volumes of data 
%to identify trends, insights, or anomalies.  This task often involves the visual inspection of data, 
%Given the scale of data, and the relative ease and intuitiveness of examining data visually,
%analysts often use visualizations as a tool to identify patterns of interest.
%Consequently, visualization software such as Tableau~\cite{}, Spotfire~\cite{} and Many Eyes~\cite{} 
%has seen unprecedented user adoption~\cite{kristi-tech-report}.
%In spite of user friendly software, however, selecting the ``right'' visualization still remains a laborious and 
%challenging task, particularly for novices unfamiliar with the data\mpv{citation?}. 
%As demonstrated by the paper on the Tableau use, majority of an analyst's time is spent in exploring
%different visualizations.
%To alleviate this problem, major visualization software vendors are attempting to incorporate automatic visualization
%recommendations into their software.
%For instance, Spotfire recently launched a ``Recommendations'' module that uses attribute metadata (e.g. type, number of distinct
%values) to suggest some ``best-practice'' visualizations~\footnote{http://spotfire.tibco.com/recommendations}.
%Similarly, Tableau's Show Me capability recommends a chart type (bar chart, map etc.) that is appropriate
%for the particular view of the data~\cite{DBLP:journals/tvcg/MackinlayHS07}.
%Both of these recommendations are straightforward applications of the best practices of visual design~\cite(); 
%neither incorporates insights about the underlying data or prior user history into its recommendations.
% These recommendations use basic metadata about attributes (categorical or numeric, number of distinct values etc)
% to produce simple visualizations that follow best practices of visual design.
%The ultimate vision for such a recommendation module is that given a dataset or a query, the visualization 
%software can study trends in data and previous interactions on that dataset to present to the user the 
%visualizations it deems most valuable.


% \mpv{Should we add a paragraph and schematic of an ideal system, that has an ensemble of models along one or 
% more of the above dimensions to produce a holistic list of recommendations?}

%In this paper, we develop a system that uses a combination of metadata, query, and statistics to 
%provide high quality, data-driven visualization recommendations.
%We envision our model to be used in conjunction with models currently used for recommendation and models that
%leverage context and user history to produce holisitc recommendations.

\stitle{Data-driven Recommendations in \SeeDB.}
To this end, in this work, we describe a new visualization recommendation engine we are building called {\it \SeeDB}.
Eventually, we aim to address all six dimensions of the above list of dimensions by developing a suite of optimization techniques designed to explore
the entire range of visualizations of a user-supplied data set or query result.  
As a first step in this paper, we develop a general data-driven deviation-based metric that can capture all six dimensions to a limited extent (see Section~\ref{sec:problem_statement} and Section~\ref{sec:discussion}).  
We focus on the query and data distribution aspects as a special case:
these add significant utility, and at the same time significant complexity to 
the visualization recommendation process,
and are particularly relevant when an analyst is approaching a dataset 
for the first time. 
We leverage lessons from metadata and aesthetics-based recommendation
generation from prior work.

% However, because metadata and aesthetics based recommendations have been addressed in prior work, in this paper we focus on {\it data-driven} recommendations that use information about data distribution, metadata and the user query to recommend visualizations.
% Such visualizations are particularly appropriate when
% approaching a dataset for the first time, with limited domain knowledge or historical context.

% that identify unusual variations in the result of the user-supplied query as compared to the underlying data.
% consider the inherent variations in the data from a user-supplied query. 
%   that focuses less on the aesthetics of good visualizations, and more
% on developing general techniques that explore the space
%of possible visualizations and recommend those that highlight data variability in the subset of data specified by a user's query.
%Metadata about a dataset, the user's query and information about data distributions can together provide powerful
%information to identify potentially interesting visualizations for the user.
%We call these recommendations {\it data-driven} since they do not draw upon any semantic information about the dataset;
%they are guided by statistics alone.
%We envision these techniques being incorporated into existing visualization systems that allow users to supply context (via, e.g.,
%interaction), and that focus more on the aesthetics of good visualization.
%As mentioned before, since we envision these recommendations to be augmented with context information (along with
%information about aesthetics), we can attack the problem piecemeal.

%Of course,  however, we note upfront
%that {\it there is a variety of ways to quantifying utility and these merit further exploration}.
%\mpv{In fact, we hope that the techniques and framework we describe in subsequent sections may be a general framework 
%into which we can plug in a variety of metrics.}

% deviation-based 
\stitle{Illustrative Example.}
Before presenting our specific model for evaluating the quality of a visualization, we provide a brief illustrative example.
Consider a smartphone app analytics team that is tasked with studying the metrics for BadApp, a smartphone app that has
poor performance and has received a lot of consumer complaints. 
Suppose that the team uses the AppMetrics database containing metrics such as network usage, 
power consumption, load times etc.
Given the large size of the database (millions of records), an analyst will 
overwhelmingly use visualization software to glean insights into the behavior of BadApp.

In a typical workflow, an analyst would begin by using the program's GUI or a custom query language to execute the equivalent
of the following SQL query and pull all BadApp metrics from the database. 
\noindent 
\begin{align*}
& \tt Q \ \ = \ \ SELECT \ * \ FROM \ \  AppMetrics \ \ WHERE  \ Name=``BadApp"
\end{align*}
Next, the analyst would use an interactive drag-and-drop GUI interface to visualize various metrics of BadApp.
For instance, the analyst may visualize average network usage for BadApp, total crashes grouped by session time,
average load times by carrier, distribution of mobile operating systems, and so on.
Under the hood, these visualization operations are essentially queries to the underlying data store and subsequent graphing of 
the results.
For example, the visualization for average load times by carrier is generated by running an operation equivalent to the
SQL query (Q') shown below.
%The result of this query is a two-column table that is very likely going to be viewed as a bar-char~\cite{vql, kristi}.
Table \ref{tab:staplerX} and Figure \ref{fig:staplerX} respectively show an example of the results of Q' and a potential
visualization.

\noindent
\begin{align*}
& \tt Q' = SELECT \ \ carrier,\ AVG(load\_time) \ \ FROM \ \  AppMetrics \\
& \tt \hspace{20pt} WHERE\ Name=``App\text{-}A" \ \ GROUP  \ \ BY \ \ carrier
\end{align*}

\begin{figure}[h]
\vspace{-10pt}
	\centering
	\begin{subfigure}{0.49\linewidth}
	   \begin{tabular}{cc} \hline
		  Carrier & Load Times (ms) \\ \hline
		  AT\&T & 180.55 \\ \hline
		  Sprint & 90.13 \\ \hline
		  T-Mobile & 122.00 \\ \hline
		  Verizon &  145.50\\ \hline
		  \end{tabular}
		  \caption{Data: Average Load Times by Carrier for BadApp} \label{tab:staplerX}
	\end{subfigure}
	\begin{subfigure}{0.49\linewidth}
		\centering
		{\includegraphics[width=4cm] {Images/dist1.pdf}}
		\caption{Visualization: Average \\ Load Times by Carrier
		 for BadApp}
		\label{fig:staplerX}
	\end{subfigure}
	
	\centering
	\begin{subfigure}{0.49\linewidth}
		{\includegraphics[width=4cm] {Images/dist2.pdf}}
		\caption{Scenario A: Average Load Times by Carrier}
		\label{fig:staplerX-a}
	\end{subfigure}
	\begin{subfigure}{0.49\linewidth}
		\centering
		{\includegraphics[width=4cm] {Images/dist3.pdf}}
		\caption{Scenario B: Average Load Times by Carrier}
		\label{fig:staplerX-b}
	\end{subfigure}
	\vspace{-10pt}
	\caption{Motivating Example}
	\label{fig:intro}
	\vspace{-15pt}
\end{figure}

\noindent As discussed previously, what might make one of these visualizations valuable depends on a host of factors.
In this work, we primarily focus on a data-driven utility metric that is based on deviation.
Specifically, we posit that a visualization is {\em potentially ``interesting'' if it shows 
a trend in the subset of data selected by the analyst
(i.e., metrics about BadApp)
that deviates from the equivalent trend in the overall dataset (i.e., metrics
about all apps)}.
In our example, the visualization of load times by carrier, as shown in Figure
\ref{fig:staplerX}, would be valuable if the global trend for load times of all
apps showed the {\it opposite} trend (e.g. Figure \ref{fig:staplerX-a}).
However, the same visualization may be uninteresting if the load times of all apps
follow a similar trend (Figure \ref{fig:staplerX-b}).
Due to the scale of data, most common visualizations show aggregate summaries of data
as opposed to individual records (e.g. average sales by state vs. sales of each store 
in every state).
Consequently, our current implementation of \SeeDB\ recommends visualizations that show aggregate 
summaries of data.

Of course, there are a variety of other possible data-driven metrics for quality or utility
of a visualization.
For instance, one might focus on visualizations that show order statistics or anomalies~\cite{DBLP:conf/avi/KandelPPHH12}. 
These would be important for spotting, for example, unusual spikes in machine load.
Similarly, drawing upon the literature on data cubes, one might choose visualizations that highlight aggregates that
are unusual given the remaining values in the cube~\cite{DBLP:conf/vldb/Sarawagi00}.
Finally, one might choose to not aggregate any values but show correlations between attributes by plotting a random
sampling of data-points.  Incorporating these other types of visualizations into our framework is an interesting
direction for future work.

Finally, we note that of the vision for \SeeDB\ has been described in a previous vision paper~\cite{DBLP:conf/vldb/Parameswaran2013} and demo paper~\cite{DBLP:journals/pvldb/VartakMPP14}, but
neither papers presented the specific visualization search techniques or evaluation.


% The majority of visualizations that are generated in visualization systems are based on aggregate summaries of the
% underlying data.
% As a result, the {\it trends} we study are the results of grouping and aggregation applied to a given dataset.


% Thus, the recommendation algorithm used by \SeeDB\ works as follows: given a dataset $D$ and a query $Q$
% indicating the subset of data of interest to the analyst, \SeeDB finds the visualizations of $Q$ that 
% show the highest deviation between trends in $Q$ and trends in $D$. 
% Specifically, \SeeDB considers visualizations that can be constructed via a combinations of grouping and 
% aggregation applied to $Q$.



\stitle{Contributions.}
In this paper, describe \SeeDB,
a visualization recommendation engine that efficiently manages the search
for data-driven visualization recommendations at interactive time scales.
We show that \SeeDB can be implemented on top of a traditional relational database,
allowing it to take advantage of the benefits of the interfaces and features
relation engines expose.
Doing this, however,
also leads to inefficiencies that arise due to the requirement that data be
accessed via SQL.   
We manage these inefficiencies by introducing two classes of optimizations:
{\em sharing-based} optimizations, that try to batch
and share as much computation as possible to minimize the number of SQL queries,
and {\em pruning-based optimizations}, that try to avoid
as much unnecessary work as possible, and discarding candidate aggregate views
that are of low utility.

In summary, the contributions of this paper are:
\begin{denselist}
%  \item We design \SeeDB as a system for data-driven visualization recommendations.
%  We explore and evaluate two distinct implementations of the system, one as a
%  wrapper around a database and another a custom solution (Section~\ref{sec:system_architecture}).
\item We describe a general deviation-based framework 
for evaluating the utility  of a visualization,
and present and evaluate a specific metric that identifies 
aggregate dimensions in a dataset that
show maximal variation (Section~\ref{sec:problem_statement}).
We develop two classes of optimizations to make such aggregates run fast 
in a conventional DBMS.
  \item The first set of techniques
  combine queries and aggregates to minimize the number of queries executed and 
  maximize the sharing of scans between queries, 
  including bin-packing~\cite{garey} algorithms and parallelization
  (Section~\ref{sec:sharing_opt}).
  \item The second set of techniques further optimize the process by adapting techniques 
  from both traditional confidence-interval-based~\cite{hoeffding1963probability} top-$k$ ranking and the
   multi-armed bandit problem~\cite{bandits} 
   to the problem finding the top-$k$ visualizations (Section~\ref{sec:in_memory_execution_engine}).
  % \item We explore visualization pruning techniques based on data distribution
  % to prune visualizations even before they are evaluated by the \SeeDB\ system 
  % (Section~\ref{sec:pruning}).
  \item We evaluate the performance of our optimizations on a range of
  real and synthetic datasets and demonstrate the resulting 40-100X speedup 
  (Section~\ref{sec:experiments}). We also demonstrate that our algorithms
  improve performance while still finding very high or maximum utility visualizations.
  \item We present the results of a user study evaluating the recommendations produced by \SeeDB.
\end{denselist}








\section{Problem Statement}
\label{sec:problem_statement}

Given a database $D$ and a query $Q$, \SeeDB\ finds and visualizes the most
interesting aspects of $Q$ with respect to the underlying dataset. For this
purpose, \SeeDB\ considers various ``views'' of the query, where a view can be
some way of slicing or aggregating the data so that it may be visualized in
terms of histograms, time series charts etc. Currently \SeeDB\ limits views to
those that can be generated by adding one aggregate and one group-by clause
(group-by can have one or more attributes). Given an input query $Q$, a view
$Q_V$ may be described as $V(Q, A_i, (G_k\ldots Gl))$ where $Q$ is the original
query, $A_i$ is the aggregate attribute and $(G_k\ldots G_l)$ are attributes in
the group-by clause. The result of a view can be thought of as a two-column
table (e.g. sum of expenses for all patients, grouped by doctor and hospital)
that can be normalized to a probability distribution (e.g.
fraction of state's medical expenses per doctor and hospital). We denote the
distribution generated by $Q_V$ as $Pr(Q_V)$.

Since a large number of views are possible, \SeeDB\ must identify the iteresting
ones. To decide if a view is interesting, \SeeDB\ runs an equivalent view query
on the entire underlying dataset, i.e. $V(D, A_i, (G_k\ldots G_l))$, and obtains
a distribution for this view too. Let's call this distribution $Pr(D_V)$. We
posit that the utility or interesting-ness of a view depends on two things: (1)
how much $Pr(Q_V)$ deviates from $Pr(D_V)$ and (2) how complex is the view
$Q_V$. We prefer views with high deviation and low complexity. In the demo
system, \SeeDB\ measures view complexity simply as the number of attributes
added to the query. For measuring deviation among the query and dataset
distributions, \SeeDB\ can use a variety of existing metrics, a few of which are
discussed below. A user is able to select any of these metrics during his/her
analysis.  

\begin{itemize}
  \item {\bf Earth Movers Distance}: Commonly used to measure differences
  between color histograms from images, EMD is a popular metric for comparing
  discrete distributions.
  \item {\bf Euclidean Distance}: The L2 norm or Euclidean distance considers
  the two distributions are points in a high dimensional space and measures the
  distance between them.
  \item {\bf Kullback-Leibler Divergence}: K-L divergence measures the
  information lost when one probability distribution is used to approximate
  another.
  \item {\bf Jenson-Shannon Distance}: Based on the K-L divergence, this
  distance measures the similarity between two probability distributions.
\end{itemize}


Using the above notion of utility, \SeeDB\ produces the top-k views having the
high utility. Thus, formally, given database $D$, query $Q$ and positive integer
$k$, \SeeDB\ finds $k$ views of $Q$ that have the largest utility. 

%Trend in the subset of the data that deviates from the corresponding trend in
%the overall data.

%!TEX root=document.tex


\section{Related Work}
\label{sec:related_work}
\VizRecDB\ is related to work from multiple areas;
we review the major papers in each of the areas, and how they relate to
\VizRecDB below. \\

\noindent{\it Interactive Data Visualization Tools}:
Over the past few years, the research community has introduced a number of
interactive data analytics tools such as ShowMe, Polaris, and 
Tableau~\cite{DBLP:journals/cacm/StolteTH08, DBLP:journals/tvcg/MackinlayHS07}.
Similar visualization specification tools have also been introduced by the
database community, including Fusion
Tables~\cite{DBLP:conf/sigmod/GonzalezHJLMSSG10} and the
Devise~\cite{DBLP:conf/sigmod/LivnyRBCDLMW97} toolkit. 
Unlike \VizRecDB, which recommends visualizations automatically, these tools place
the onus on the analyst to specify the visualization to be generated.
For datasets with a large number of attributes, it is not possible
for the analyst to manually study all the attributes; hence, interactive
visualization needs to be augmented with automated visualization techniques.

A few recent systems have attempted to automate some aspects of data analysis
and visualization. Profiler is one such automated tool that allows analysts to
detect anomalies in data \cite{DBLP:conf/AVI/KandelPPHH12}---this tool
maintains an in-memory data cube, which is infeasible for large datasets.
Our work is also similar to VizDeck which is a tool that given a dataset, uses a
set of pre-determined rules to create diverse visualizations and
allows the user to pick and choose the visualizations that seem relevant
\cite{DBLP:conf/sigmod/KeyHPA12}.
Thus, while powerful, VizDeck requires much more manual input than \VizRecDB. 
In addition, the visualizations generated by VizDeck do not leverage the
context of the underlying dataset, making the visualizations generated by
both systems very different in flavor. 
It would be instructive to augment
VizDeck visualizations with \VizRecDB\ visualizations to study their relative
utility.
\agp{Suggested edit: Replace the last three lines with: Another related tool 
is VizDeck~\cite{DBLP:conf/sigmod/KeyHPA12}, which, given a dataset,
depicts all visualizations on a dashboard that the user can manipulate
by reordering or pinning visualizations.
Given that VizDeck generates all visualizations, it is only meant for 
small datasets; additionally, the VizDeck does not discuss techniques
to speed-up the generation of these visualizations.}
 \\

\noindent {\it Data Cubes}:
The work done in \VizRecDB\ is similar to previous literature in
browsing OLAP data cubes. 
Instead of building complete data cubes,
one can think of \VizRecDB\ views as projections of the cube along various
dimensions.
 Data cubes have been very well studied in the literature
\cite{DBLP:conf/SIGMOD/HarinarayanRU96, DBLP:jounral/DMKD/GrayCBLR97}, and work such as
~\cite{DBLP:conf/vldb/Sarawagi99, DBLP:conf/vldb/SatheS01,
DBLP:conf/vldb/Sarawagi00, DBLP:conf/SIGKDD/OrdonezC09} has explored the
questions of allowing analysts to find explanations for trends, get suggest for
cubes to visit, identify generalizations or patterns starting from a single
cube. 
This literature is not directly applicable to our problem since the cubes we
are considering have 10s to 100s of dimensions, making traditional cube
algorithms inefficient. \\

\noindent {\it General Purpose Data Analysis Tools}:
Our work is also related to data mining and the work on building general purpose
data analysis tools on top of databases. 
For example, MADLib \cite{DBLP:conf/VLDB/HellersteinRSWF12}
implements various analytic functions inside the database. 
MLBase similarly
\cite{DBLP:conf/CIDR/KraskaTDGFJ2013} provides a platform that allows users to
run various machine learning algorithms on top of the Spark system
\cite{DBLP:conf/SCC/ZahariaCFSS10}.
Statistical analysis and graphing packages such as R, SAS and Matlab could also
be used generate visualizations, but they lack the ability to filter and
recommend ``interesting'' visualizations. \\


\noindent {\it Multi-query optimization}
Since \VizRecDB\ must execute a large number of queries, there are several
opportunities for performing multi-query optimization and we explore some of
these strategies in Section \ref{sec:dbms_optimizations} and build an analytical model of
\VizRecDB\ performance. For this, we draw upon work in the areas of multi-query
optimization and modeling parallel query execution \cite{DBLP:conf/VLDB/WuCHN13,
DBLP:journal/TODS/Sellis1988, DBLP:conf/VLDB/ZukowskiHNB07}. \\

\noindent {\it Top-k Ranking and Multi-Armed Bandit Problems}
The techniques we use in our custom implementation of \VizRecDB\ draw upon work
from top-k ranking, statistical sampling and the multi-armed bandit strategies. 
In particular, the confidence interval technique discussed in Section
\ref{sec:confident_interval} draws inspiration from the seminal top-k ranking work by Fagin and others in
\cite{DBLP:conf/pods/FaginLN01, DBLP:conf/vldb/IlyasAE04}.
Similarly, multi-armed bandits (referenced in Section
\ref{sec:multi_armed_bandit}) form 
a rich area of research having applications from ad auctions to reinforcement learning. 
Our technique is related to the original UCB algorithm \cite{AuerCF02, LaiR85}
as well as recent work related to the top-$k$ MAB variant \cite{BubeckWV13,
audibert2010best}.

\agp{Refer to my email about the related work and add additional ones.}

 

\begin{figure*}[ht]
  \centering
  \includegraphics[width=\textwidth]{images/seedb-architecture.pdf}
  \caption{SeeDB Architecture}
  \label{fig:sys-arch}
\end{figure*}

\section{System architecure}

\subsection{\SeeDB overiew}
\label{overview}

Figure \ref{XXX} shows the architecture of our system. Currently, \SeeDB\ is a
wrapper around a database (PostgreSQL in this case). While optimization
opportunities are restricted by virtue of being outside the DBMS, we believe
that it allows quick iteration and permits \SeeDB\ to be used with different
backends. 

Once a user issues a query $Q$ to \SeeDB\, the system generates potential views
by rewriting $Q$ with various group-bys and aggregates. Each such rewritten
query is termed as a ``view.'' These views are then grouped according to
optimization opportunities discussed in \ref{optimizations} and sent to the
DBMS. The results of these view queries are then processed by \SeeDB to compute
the utility of views. The top-k views with highest utility are picked,
appropriate visualization techniques chosen and the top views are visualized at
the \SeeDB front end. The analyst can then examine these views and perform
further processing.

\subsection{Basic Framework}
\label{basic_framework}

Given a user query $Q$, the basic version of \SeeDB\ computes all possible view
queries by adding a single aggregate and group-by operator. Each of these view
queries is executed at the backend along with an equivalent aggregate+group-by
query on the complete underlying dataset. These two query results produce a
``distribution'' for the attribute that has been aggregated. These two
distributions are compared using the chosen distance metric (default: earth
movers distance, other: L2, Jensen-Shannon distance etc). All the views are
ranked by the distance between the query and dataset distribution and the
top k views with the largest distance are chosen. Appropriate visualizations are
chosen for these distributions (heuristics in Section \ref{user_interface}) and
are displayed to the user for further interaction.

{\bf Utility Metric}:

One of the key challenges behind \SeeDB\ 
is formalizing the utility function $U(R)$ for a discriminating view $R$. 
There are many choices for $U$ and we expect \SeeDB\ 
to recommend views that score high on several metrics. 
As discussed previously, the proposed metric tries to capture the idea of
``deviation'' between distributions, i.e., a view has high utility if its
contents show a trend that deviates from the corresponding trend in the original
database.

We first define some notation. For any discriminating view $R_i$ 
in the class defined above, we note that $R_i(D)$ and $R_i(Q(D))$ 
are both two column tables. 
A two-column table can be represented using a weight vector.
We let the weight vector $W_{a, f(m)}$ represent the 
result of $R_i(D) = \gamma_{a, f(m)}(D)$, i.e., 
distribution of the aggregate function $f$ on the measure quantity $m$ 
across various values of the attribute $a$. 

The utility $U$ of a discriminating view $\gamma_{a, f(m)}$ is defined to be the
distance between $W_{a, f(m)}^Q$, and $W_{a, f(m)}$:
$U(\gamma_{a, f(m)}) = S(W_{a, f(m)}^Q, $ $W_{a, f(m)})$ where $S$ is a distance
metric. The higher $S$ is, the more useful a discriminating view is.
Common distance metrics used in visualization literature include K-L
divergence~\cite{wikipedia-KL}, Jenson-Shannon
distance~\cite{wikipedia-JS,entropy-vis}, and earth mover
distance~\cite{wikipedia-prob-dist}.
Wang~\cite{entropy-vis} provides a good overview of the metrics used in
scientific visualizations, while \cite{wikipedia-prob-dist} provides a summary
of probability-based distance metrics.
As discussed earlier, we do not prescribe any specific distance metrics,
instead, we plan to support a whole range of distance metrics, which can be
overridden by the data analyst.

\subsection{Optimizations}
\label{optimizations}

The fact that \SeeDB\ evaluates a large number of possible views presents
various opportunities for optimization. The strategies include:

\begin{enumerate}
  \item {\it Rewrite view query}: Since similar group-by and aggregate queries
  are executes on the results of user query $Q$ and the underlying dataset, we
  can combine these queries into one query. As shown in Figure \ref{}.a this
  achieves a speed-up of Y\%.
  \item {\it Single Group-by Multiple Aggregates}: A large number of view
  queries have the same group-by clause but aggregates on different attributes.
  A straightforward optimization combines all view queries with the same
  group-by clause into a single view query. This rewriting provides a speed up
  linear in the number of aggregate attributes. (Figure \ref{}. b).
  \item {\it Multiple Aggregate Computation}: Similar to data cubes,
  \SeeDB\ seeks to compute group-bys for a large set of attributes. One
  optimization is to combine queries with different group-by attributes into a
  single query with mulitple group-bys. For instance, consider view queries
  $VQ(Q, A_1, GB_1)$, $VQ(Q, A_1, GB_2)$ \ldots $VQ(Q, A_1, GB_n)$. Instead of
  executing them individually, we can rewrite them into a single view query
  $VQ(Q, A_1, (GB_1, GB_2\ldots GB_n))$. While this strategy reduces query
  execution time, \SeeDB\ must spend more time combining the results and
  obtaining separate aggregates for individual $GB_i$'s. This is reminiscent of
  data cube algorithms. As shown in Figure \ref{}.c the speed up depends closely
  on the number of distinct values for each of the group-by attributes and the
  memory constraints of the DBMS.
  \item {\it Sampling}: The final optimization we study for the purpose of this
  demo is sampling. Instead of running queries on the entire dataset, we
  run queries on subsets of the data at the expense of reduced accuracy.
  Figure \ref{}.d shows the effects of this optimization. << write about
  accuracy >>
  
  Although other optimizations are possible, particularly related to
  pre-computing aggregate results, we discuss them in the full paper currently
  in preparation.
\end{enumerate}

\subsection{User Interface}
\label{user_interface}

<< write about user interface, add pictures>>



\begin{figure*}[ht]
  \centering
  \includegraphics[width=\textwidth]{images/frontend}
  \caption{SeeDB Frontend: Query Builder (left) and Example Visualizations
(right)}
\label{fig:frontend1}
\end{figure*}


\section{SeeDB Frontend}
\label{subsec:seedb_frontend}

TThe \SeeDB\ frontend, designed as a thin client, performs two main functions: it
allows the analyst to issue a query to \SeeDB, 
and it visualizes the results (views) produced by the \SeeDB\
backend.
To provide the analyst maximum flexibility in issuing queries, \SeeDB\
provides the analyst with three
mechanisms for specifying an input query: 
\vspace{5 mm}

 \squishlist
   \item directly filling in SQL into a text box
   \item sing a query builder tool that allows analysts
unfamiliar with SQL to formulate queries through a form-based interface
   \item using pre-defined query templates which encode commonly performed operations,
e.g., selecting outliers in a particular column. 
 \squishend

\vspace{5 mm}


We find that pre-defined query templates are particularly useful since analysts are often interested in anomalous data points. Additionally, \SeeDB\
's pre-defined templates and query building tool enable analysts who are relatively unfamilar with the SQL syntax or the dataset being considered are still able to gain valuable insights into the dataset in one glance. This way, we have both simplified the process of insight identification as well as making it more accessible to less experienced data analysts.


Once the analyst issues a query via the \SeeDB\ frontend, the backend
evaluates various views and delivers the most interesting ones (based on
utility) to the frontend.
For each view delivered by the backend, the frontend creates a visualization
based on parameters such as the data
type (e.g. ordinal, numeric), number of distinct values, and semantics (e.g.
geography vs. time series).
The resulting set of visualizations is displayed to the analyst who can then
easily examine these ``most interesting'' views at a glance, explore specific views in
detail via drill-downs, 
%by hovering and clicking on various portions of the view, 
and study metadata for each view (e.g. size of result, sample data, value with
maximum change and other statistics). 
Figure~\ref{fig:frontend1} shows a screenshot of the \SeeDB\ frontend (showing
the query builder) in action.

After an analyst sees the resulting graphs returned by \SeeDB\ frontend, they can also slice-and-dice views further by performing drill-downs on specific attributes in the view by (a) interactively selecting sections of the graph or (b) selecting a different group by or aggregate value. As such, an analyst is able to actively interact with the system to get to the most interesting results. \SeeDB\ reduces the latency of these interactions by preemptively sending the results of all possible group-bys and aggregates to the frontend. When the analyst changes the value they want to aggregate or group by, the new graphs are rendered instantaneously using the data that has already been sent to frontend, as opposed to making another round trip to the backend and database server.

\SeeDB\'s frontend is designed intentionally to be simple and intuitive, such that an analyst who has limited SQL syntax knowledge and is not previously familiar with the dataset can easily identify insights in the database that they might not be able to find manually. For the advanced analysts, \SeeDB\ allows them to directly enter queries, reduce manual labor, and speed up the insight identification process.



\chapter{SeeDB Backend}
\label{subsec:seedb_backend}

The \SeeDB\ backend is responsible for all the computations for 
generating and selecting views. 
%\agp{next line can be deleted if needing space, repetitive.}
% As shown in Figure~\ref{fig:sys-arch}, the \SeeDB\ backend is composed of four
% modules that are respectively responsible for collecting metadata (Metadata Collector), pruning
% the view space and generating view queries (Query Generator), optimizing view
% queries (Optimizer), and processing query results to identify the top-$k$
% interesting views (View Processor). 
To achieve its goal of finding the most
interesting views accurately and efficiently, the \SeeDB\ backend must not only accurately
estimate the quality of a large number of views but also minimize total processing time.
We first describe the basic \SeeDB\ backend framework and then discuss our optimizations.

% One of the chief challenges in \SeeDB\ is producing the most interesting views
% of the query result in the least possible time. For achieve the above
% performance goal, \SeeDB\ must perform optimizations at two stages: first, using
% prior knowledge such as statistics to prune out uninteresting views without examining table data; and second, minimizing the
% execution time for queries that are issued to the database. 

% \subsubsection{Basic Framework} \label{subsubsec:basic_framework}
\section{Basic Framework}
\label{sec:basic_framework}
Given a user query $Q$, the basic approach computes all possible two-column
views obtained by adding a single-attribute aggregate and group-by clause to
$Q$. (Remember from \ref{sec:problem_statement} that $Q$ is any query that
selects one or more rows from the underlying table.) The target and comparison
views corresponding to each view are then computed and each view query is
executed independently on the DBMS. The query results for each view are
normalized, and utility is computed as the distance between these two
distributions (Section \ref{sec:problem_statement}).
Finally, the top-$k$ views with the largest utility are chosen to be displayed.
If the underlying table has $d$ dimension attributes and $m$ measure attributes,
$2\ast d \ast m$ queries must be separately executed and their results
processed. Even for modest size tables (1M tuples, $d$=50, $m$=5), this
technique takes prohibitively long (700s on Postgres). The basic approach is
clearly inefficient since it examines every possible view and executes each view
query independently. We next discuss how our optimizations fix these problems. 

%\subsubsection{View Space Pruning}
%\label{subsubsec:view_space_pruning}

%\mpv{put microbenchmark pictures here}

% \section{View Space Pruning:}
% In practice, most views for any query $Q$ have low utility since the target view
% distribution is very similar to the comparison view distribution. 
% \SeeDB\ uses this property to aggressively prune 
% view queries that are unlikely to have high utility. 
% This pruning is based on metadata about the table including data
% distributions and access patterns. 
% \mpv{fix following. taken from previous full paper}
% Specifically, no expensive scans of the underlying
% tables are performed. In addition, we order the execution of view queries so
% that higher utility views can be computed before those with lower utility,
% thus permitting early stopping. For each table in the DBMS, we assume that
% statistics from Table~\ref{tab:statistics} are available or can be computed
% cheaply. The data type for each column is numeric, categorical, ordinal, geographic, or
% date\_or\_time. The data type for column $C_i$, $T(C_i)$, along with the number
% of distinct values $|C_i|$ is used to determine whether the column will be
% treated as a {\it dimension} attribute or a {\it measure}
% attribute (Section~\ref{sec:definitions}). As before, we denote the set of
% dimension attributes by $\mathcal{D}$ and measure by $\mathcal{M}$.
% 
% We employ the following heuristics for pruning and ordering views based on the
% statistics above.
% 
% \begin{table}
% {\scriptsize \center
% \vspace{-10pt}
% \begin{tabular}{|c|c|c|c|}
% \hline
% $T(C_i)$ & Data type for column $C_i$ \\ \hline
% $|C_i|$ & Number of distinct values in $C_i$ \\
% \hline $Var(C_i)$ & Variance of values in $C_i$ \\ \hline
% $Corr(C_i, C_j)$ & Correlation measure for all pairs of columns \\ \hline
% $\mathcal{H}_{i\ldots k}$ & Hierarchies between columns $C_i$ to $C_k$ \\ \hline
% $f_{C_i}, f_{C_i, C_j}$ & Frequency of access for each column and column pair \\
% \hline
% \end{tabular} 
% \vspace{-10pt}
% \caption{Statistics and Table Metadata \label{tab:statistics}}
% }
% \end{table}
% 
% \begin{itemize}
% \item {\it Variance-based pruning}: Dimension attributes with low variance are
% likely to produce views having low utility (e.g. consider the extreme case where
% an attribute only takes a single value); \SeeDB\ therefore prunes views
% with grouping attributes with low variance.
% \item {\it Correlated attributes}: If two dimension attributes $a_i$ and $a_j$ have
% a high degree of correlation (e.g. full name of airport and abbreviated name of
% airport), the views generated by grouping the table on $a_i$ and $a_j$ will be
% very similar (and have almost equal utility). We can therefore generate and
% evaluate a single view representing both $a_i$ and $a_j$. \SeeDB\ clusters
% attributes based on correlation and evaluates a representative view per
% cluster.
% \item {\it Bottom-up hierarchy traversal}: \mpv{fix. from full paper draft}
% We observe that for a set of
%   dimension attribute with a hierarchial structure, $H_{C_{i\ldots k}}$, if a
%   view $V$ at hierarchy level $h$ has utility $u$, then views at hierarchy level
%   $h-1$ will have utility $\leq$ $u$. XXX: is this true for all utility
%   functions?
% \item {\it Access frequency-based pruning}: In tables with a large number of
% attributes, only a small subset of attributes are relevant to the analyst and
% are therefore frequently accessed for data analysis. \SeeDB\ tracks access patterns
% for each table to identify the most frequently accessed columns and combinations of
% columns. While creating views, \SeeDB\ uses this information to prune attributes
% that are rarely accessed and are thus likely to be unimportant.
% \end{itemize}

% \subsubsection{View Query Optimizations} \label{subsubsec:optimizations}
% \section{View Query Optimizations} The second set of optimizations used by
% \SeeDB\ minimizes the execution time for view queries that haven't been pruned
% using the techniques described above.

\section{Optimizations} \label{sec:optimizations}
Since view queries
tend to be very similar in structure (they differ in the aggregation attribute,
grouping attribute or subset of data queried), \SeeDB\ uses multiple techniques
to intelligently combine view queries. In addition, \SeeDB\ leverages
parallelism and partitioning to further reduce query execution time. The
ultimate goal of these optimizations is to minimize scans of the underlying
dataset by sharing as many table scans as possible. SeeDB supports the following
optimizations as well as their combinations.

\subsection{Combine target and comparison view query}
\label{subsec:target_comparison_view}
Since the target view and comparison views only differ in the subset of data
that the query is executed on, we can easily rewrite these two view queries as
one. For instance, for the target and comparison view queries $Q1$ and $Q2$
shown below, we can add a group by clause to combine the two queries into $Q3$.
\begin{align*} 
Q1 = &{\tt SELECT \ } a, f(m) \ \ {\tt FROM} \  D\  {\tt WHERE \ \ x\ <\ 10\
GROUP \ \ BY} \ a \\
Q2 = &{\tt SELECT \ } a, f(m) \ \ {\tt FROM} \  D\  {\tt GROUP \ \ BY} \ a \\
Q3 = &{\tt SELECT \ } a, f(m), {\tt CASE\ IF\ x\ <\ 10\ THEN\ 1\ ELSE\ 0\
END}\ as\ group1,\ 1\ as\ group2\\ 
&{\tt FROM} \ D\ {\tt GROUP \ \ BY} \ a,\ group1,\ group2
\end{align*}
This rewriting allows us to obtain results for both queries in a single table
scan. The impact of this optimization will depend on the selectivity of the
input query and the presence of indexes. When the input query is less selective,
the query executor must do more work in running the two queries separately.
In contrast, if the target and comparison views are both selective, and an index
is present on their selection attributes, individual queries can run much
faster than the combined query which must scan the entire table.
  
\subsection {Combine Multiple Aggregates} 
A large number of view queries have the same group-by attribute but different
aggregation attributes. In addition, the majority of real-world datasets,
tables have few measure attributes but a large number of dimension attributes
(e.g. the Super Store dataset has 5 measure attributes but tens of dimension
attributes). Therefore, \SeeDB\ combines all view queries with the same group-by
attribute into a single, combined view query. For instance, instead of executing
queries for views $(a_1$, $m_1$, $f_1)$, $(a_1$, $m_2$, $f_2)$ \ldots $(a_1$, $m_k$, $f_k)$
independently, we can combine the $n$ views into a single view represented by
$(a_1, \{m_1, m_2\ldots m_k\}$, $\{f_1, f_2\ldots f_k\})$. We expect this
optimization to offer a speed-up roughly linear in the number of measure
attributes.

\subsection {Combine Multiple Group-bys}
\label{subsec:mult_gb}
  Since \SeeDB\ computes a large number of group-bys, one significant
  optimization is to combine queries with different group-by attributes into a
  single query with multiple group-bys attributes.
  For instance, instead of executing queries for views $(a_1$, $m_1$, $f_1)$,
  $(a_2$, $m_1$, $f_1)$ \ldots $(a_n$, $m_1$, $f_1)$ independently, we can
  combine the $n$ views into a single view represented by $(\{a_1, a_2\ldots
  a_n\}$, $m_1$, $f_1)$ and post-process results at the backend. Alternatively,
  if the SQL GROUPING SETS\footnote{GROUPING SETS allow the simultaneous
  grouping of query results by multiple sets of attributes.} functionality is
  available in the underlying DBMS, \SeeDB\ can leverage that as well.
  While this optimization has the potential to significantly reduce query
  execution time, the number of views that can be combined will depend on the
  number of distinct groups present for the given combination of grouping
  attributes. For a large number of distinct groups, the query executor must
  keep track of a large number of aggregates. This increases computational time
  as well as temporary storage requirements, making this technique ineffective.
  The number of distinct groups in turn depends on the correlation between
  values of attributes that are being grouped together. For instance, if two
  dimension attributes $a_i$ and $a_j$ have $n_i$ and $n_j$ distinct values
  respectively and a correlation coefficient of $c$, the number of distinct
  groups when grouping by both $a_i$ and $a_j$ can be approximated by
  $n_i$$\ast$$n_j$$\ast$(1-$c$) for $c$$\neq$1 and $n_i$ for $c$=1 ($n_i$ must
  be equal to $n_j$ in this case).
  As a result, we must combine group-by attributes such that the number of
  distinct groups remains small enough. In Section \ref{sec:experiments}, we
  characterize the performance of this optimization and devise strategies to
  choose dimension attributes that can be grouped together.
  
  %Given a set
  %of candidate views, we model the problem of finding the optimal combinations
  %of views as a variant of bin-packing and apply ILP techniques to obtain the
  %best solution. 
  
  %\mpv{Bin packing forumation details}
  
  If we choose a set of grouping attributes that creates a large number of
  distinct groups, not only does the query executor need to do more work, the result
  returned to the client is large and the client takes longer to process the
  result. Since this process can be very inefficient, we choose to store
  the intermediate results as temporary tables and then subsequently query the
  temp tables to obtain the final results. For ease of further analysis, we
  denote these two phases as {\it Temp Table Creation} (where the
  intermediate results are created and stored) and {\it Temp Table Querying}
  (where the temp tables are queried for final results) respectively.
  
%   (We discuss our model and
%   algorithm in our full paper~\ref{}).\agp{if full paper is not available
%   by the time of the demo, we can't cite it unfortunately..}
%   A variation of this approach also implemented
%   on \SeeDB\ is to send the results of the multiple group-by query to the front
%   end and ask the \SeeDB\ frontend to compute utility and select views. The
%   advantage of this approach is that it allows for more efficient interactive
%   exploration of the views.

  \subsection {Parallel Query Execution}
  \label{subsec:parallel_exec}
  While the above optimizations reduce the number of queries executed, we can
  further speedup \SeeDB\ processing by executing view queries in parallel. When
  executing queries in parallel, we expect co-executing queries to share pages in the
  buffer pool for scans of the same table, thus reducing the total execution
  time. However, a large number of parallel queries can lead to poor
  performance for several reasons including buffer pool contention, locking and
  cache line contention \cite{Postgres_wiki}. As a result, we must identify the
  optimal number of parallel queries for our workload.
  
  % We do observe a reduction in the
  %overall latency when a small number of queries are executing in parallel;
  % however, the advantages disappear for larger number of queries running in
  % parallel. We discuss this further in the evaluation section.
  
  \subsection {Sampling}
  For large datasets, sampling can be used to significantly improve
  performance. To use sampling with \SeeDB, we precompute a sample of the
  entire dataset (size of sample depends on desired accuracy). When a query is
  issued to \SeeDB, we run all view queries against the sample and pick the
  top-k views. Only these high-utility views are then computed on the entire
  dataset. As expected, the accuracy of views depends on the size of the sample;
  a larger sample generally produces more accurate results and we can develop
  bounds on the accuracy of aggregates computed on samples.
  There are two ways to employ sampling in the \SeeDB\ setting:
  (1) depending on the response time required, choose a sample size that will
  provide the required response time and accordingly return to the user the
  estimated accuracy of the results; or (2) given a user-specific threshold for
  accuracy, determine the correct size of the sample and apply the above
  technique. 
  
% We next describe a scheme that allows us to associate upper and lower bounds for
% views by evaluating them on a small sample of the dataset.
% We describe the use of the scheme on a simple view where AVG(Y) for a given
% attribute Y is being computed for each group in attribute X.
% We can then depict this view using a bar chart or a histogram.
% 
% For this derivation, we assume that the AVG(Y) for any X = $x_i$, is normally
% distributed around a certain mean $p$.
% Given a number of samples for Y for X = $x_i$, we can employ the following
% theorem \cite{stats_book} to bound $p$ within a confidence interval with
% probability $1 - \delta$:
% \begin{theorem}~\label{thm:confint}
% If $\hat{p}$ and $s$ are the mean and standard deviation 
% of a random sample of size $n$ from a normal distribution with unknown 
% variance, a $1 - \delta$ probability confidence interval
% on $p$ is given by:
% $$\hat{p} - \frac{t_{\delta/2, n-1} s}{\sqrt{n}} \leq p \leq \hat{p} + \frac{t_{\delta/2, n-1} s}{\sqrt{n}}$$
% where $t_{\delta/2, n-1}$ is the upper 100$\alpha/2$ percentage point
% of the $t$-distribution with $n-1$ degrees of freedom.
% \end{theorem}
% 
% Now, we demonstrate how we can use this theorem to establish an upper 
% and lower bound for the utility of a view, with probability $1 - \delta$.
% 
% Let the distance vector corresponding to the target view be:
% $\bar{a} = [a_1, a_2, \ldots, a_k]$ while the distance vector corresponding to
% the comparison view is:
% $\bar{b} = [b_1, b_2, \ldots, b_k]$.
% Notice that on very large datasets, it may be beneficial to precompute the
% distance vectors corresponding to the comparison views, so we assume that the
% vector $\bar{b}$ is computed exactly and known in advance.
% We let $a = \sum_i a_i$, and $ b = \sum_i b_i$.
% 
% Our goal is to use the sample to bound the values of the $a_i$ around $\ha_i$
% such that we can establish upper and lower bounds for the utilities.
% By applying Theorem~\ref{thm:confint}, we
% can get values $c_i$ for which $a_i \in [\ha_i - c_i, \ha_i + c_i]$
% with probability greater than $1 - \delta/k$.
% (By union bound, we will be able to ensure that all $a_i$'s
% are in their intervals with probability $1 - \delta$.)
% 
% Now, given these values $c_i$, we can establish an upper bound for the
% EMD (and also similarly for other distance metrics) in the following manner:
% We let $q_1(\bar{a}) = \sum_i \ha_i - c_i$, and $q_2(\bar{a}) = \sum_i \ha_i + c_i$.
% 
% 
% \begin{align*}
% EMD(\bar{a}, \bar{b}) & = \sum_i |a_i / a - b_i / b|\\
% 					& = \sum_i |a_i / a - b_i / b|\\
% 					& = 1/ab \sum_i \max (a_ib  - b_ia, b_ia - a_ib)\\
% \end{align*}
% Thus, we have:
% \begin{align}
% \frac{1}{b q_1(\bar{a})} \sum_i \max (a_ib  - b_ia, b_ia - a_ib) \leq & EMD(\bar{a}, \bar{b}) \leq \frac{1}{b q_2(\bar{a})} \sum_i \max (a_ib  - b_ia, b_ia - a_ib)\label{eq:emd}
% \end{align}
% 
% Note that: 
% \begin{align*}
% (\ha_i - c_i)b  - b_i (\sum_i (\ha_i + c_i)) & \leq a_ib  - b_ia  \leq (\ha_i + c_i)b  - b_i (\sum_i (\ha_i - c_i)), \textrm{\ and} \\
% b_i (\sum_i (\ha_i - c_i)) - (\ha_i + c_i) b & \leq b_i a  - a_i b  \leq  b_i (\sum_i (\ha_i + c_i)) - (\ha_i - c_i) b
% \end{align*}
% By plugging these quantities back into Eq~\ref{eq:emd},
% we have upper and lower bounds on the EMD metric.
% Similar mechanisms may be used to derive upper and lower bounds for other metrics.
% 
% Now that we have upper and lower bounds for the utility of each target view
% by evaluating the query on a sample,
% we can easily use it to prune away a number of views that are definitely not likely to be part 
% of the top-K,
% and instead focus on views that may be part of the top-K.

%   \subsection {Partitioning Tables}
%   The increase in the total execution time when a large number of queries are
%   executed in parallel suggests that there is a ``sweet spot'' with respect to
%   the maximum number of queries that can be run in parallel on a given table.
%   Therefore, we uniformly partition large tables into smaller ones and run
%   subsets of queries against each of the partitions. Note that the views
%   returned are nor approximate because we are now executing views against
%   subsets of the data. As a result, bounds developed in sampling now apply. We

  \subsection {Pre-computing Comparison Views}
  We notice that in the case where our comparison view is constructed from the
  entire underlying table (Example 1 in Chapter \ref{sec:introduction}),
  comparison views are the same irrespective of the input query.
  In this case, we can precompute all possible comparison views once and store
  them for use in all future comparisons. If the dataset has $d$ dimension and
  $m$ measure attributes, pre-computing comparison views would add $d$$\ast$$m$
  tables. This corresponds to an extra storage of $O(d\ast m \ast n)$ where $n$
  is the maximum number of distinct values in any of the $d$ attributes. In this
  case, we still need to evaluate each target view, and we can leverage previous
  optimizations to speed up target view generation.
  
  Note that pre-computation cannot be used in situations where the comparison
  view depends on the target view (Example 2) or is directly specified by the
  user (Example 3).


%If a dimension attribute $\mathcal{d}$ is highly correlated with measure
  %attribute $\mathcal{m}$, then?

% \mpv{also from full paper draft}
% It is possible to collect the above statistics at the dataset level too, as
% opposed to the entire table level. The advantage of table level statistics is
% that they have to be computed only once per table; however, dataset-level
% statistics are more accurate since they only consider the specific parts of the
% table. XXX: we use dataset-level statistics with table statistics do not result
% in aggressive pruning. 

% 
% 
% \section{Handling attributes of different kinds}
% \mpv{Numerical, Temporal, Categorical, Gepgraphical}\\
% \mpv{Binning}
%\section{Handling distribution of single attributes}
%!TEX root=document.tex

\section{User Study}
\label{sec:user_study}

The previous section evaluated \SeeDB 
and our optimizations in terms of performance.
In this section, we assess the utility of \SeeDB's recommendations 
with real users.
First, we perform a study to validate our deviation-based distance metric.
We show that although simple, our deviation-based
metric can find visualizations users feel are interesting.
Second, we compare \SeeDB
to a manual charting tool without visualization recommendations.
We show that \SeeDB can enable users to find interesting visualizations
faster and can surface unexpected trends.
We also find that users overwhelmingly prefer \SeeDB over a manual charting 
tool.

\subsection{Validating Deviation-based Utility}
\label{sec:validating_metric}

\SeeDB uses deviation between the target and reference dataset as a measure
of interestingness of a visualization.

\stitle{Ground Truth.}
To validate deviation as a utility metric, we obtained ground truth data about
interestingness of visualizations and evaluated \SeeDB against it.
To obtain ground truth, we presented 5 data analysis experts with the Census 
dataset (Section \ref{sec:introduction}) the analysis task of
studying the effect of marital status on socio-economic indicators.
We presented experts with the full set of potential aggregate visualizations 
and asked them to classify each visualization as interesting or
not interesting {\em in the context of the task}.
Of the 48 visualizations, on average, experts classified 4.5 visualizations
(sd = 2.3) as being interesting for the task.
The small number indicates that of the entire set of potential visualizations, 
only a small fraction (\textasciitilde10\%) show interesting trends.
To obtain consensus on ground truth, we labeled
any visualization chosen by a majority of participants as 
interesting; the rest were not. 
This process identified 6 interesting and 42 uninteresting visualizations.
In addition to Figures~\ref{fig:interesting_viz} (interesting, recommended by \SeeDB) 
and Figure \ref{fig:uninteresting_viz} (not interesting, not recommended by \SeeDB), 
Figure~\ref{fig:huhi}, a
visualization recommended by \SeeDB, 
was labeled as interesting (according to a expert: ``\ldots it
shows a big difference in earning for self-inc adults'') while 
Figure~\ref{fig:luli} was labeled as not interesting (notice the lack of deviation).
While some classifications can be explained using deviation, some cannot: 
Figure \ref{fig:huli} shows high deviation and is recommended by \SeeDB, 
but was deemed uninteresting, while Figure \ref{fig:luhi} shows small 
deviation but was deemed interesting (``\ldots hours-per-week seems like a 
measure worth exploring''). 

% Figure \ref{fig:gt_examples} shows ground truth for four other visualizations: 
% Figure \ref{fig:huhi} shows another visualization that was chosen as interesting by 4 of 5 
% participants.
% According to participants, this visualization was interesting because ``\ldots it
% showed a big difference in earning for self-inc adults'' and indicated a trend to 
% be examined further.
% Figure \ref{fig:luli} in contrast shows a visualization that participants
% did not select as relevant.
% Notice that the two distributions in this chart do not show significant difference. 
% However, this is not to say that deviation was the only factor relevant for interestingness.
% Figure \ref{fig:huli}, for example, shows a visualization that in fact has high deviation, but was 
% not classified as interesting by any participants.
% Similarly, Figure \ref{fig:luhi} shows a visualization that shows less deviation but was 
% classified as interesting by 2 participants because the .

% The gold standard for evaluating a recommendation system is to obtain ground 
% truth about user preferences about
% various (ideally all) items and examine whether the system's recommendations 
% can correctly classify each item\cite{??}.
% We adopt the same evaluation strategy.
% For the census dataset and associated analytical task (discussed in Section 
% \ref{sec:introduction}), we obtain ground truth about the interesting-ness of 
% each potential visualizations of the dataset.
% We then evaluate whether \SeeDB can correctly classify visualizations as
% interesting or not interesting.

% \stitle{Obtaining Ground Truth}.
% To obtain ground truth, we recruited 5 participants with significant data analysis 
% experience (3 female, 2 male).
% We presented each participant with the Census dataset and the task of finding visualizations
% that showed interesting trends related to marital status (Section \ref{sec:introduction}).
% Participants were presented with the entire set of aggregate visualizations \mpv{how many}
% for this dataset and asked to classify visualizations as being interesting or 
% not-interesting for the task.
% Participants were also asked to explain verbally why they thought a visualization
% was interesting.
% We capped the study at 10 minutes.

\begin{figure}[t]
	\centering
	\begin{subfigure}{0.45\linewidth}
		{\includegraphics[width=4cm, trim=0 0 3cm 0, clip=true] {Images/HUHI_work_avg_cap_gain.pdf}}
		\caption{High deviation, interesting}
		\label{fig:huhi}  
	\end{subfigure}
	\begin{subfigure}{0.54\linewidth}
		{\includegraphics[width=4.5cm] {Images/LULI_race_avg_age.pdf}}
		\caption{Low deviation, not interesting}
		\label{fig:luli}
	\end{subfigure}
	\begin{subfigure}{0.45\linewidth}
		{\includegraphics[width=4cm, trim=0 0 3cm 0, clip=true] {Images/HULI_work_avg_cap_loss.pdf}}
		\caption{High deviation, not interesting}
		\label{fig:huli}  
	\end{subfigure}
	\begin{subfigure}{0.54\linewidth}
		{\includegraphics[width=4.5cm] {Images/LUHI_inc_avg_hours.pdf}}
		\caption{Low deviation, interesting}
		\label{fig:luhi}
	\end{subfigure}
	\vspace{-10pt}
	\caption{Examples of ground truth for visualizations}
	\vspace{-10pt}
	\label{fig:gt_examples}
\end{figure} 


% Figure \ref{fig:interesting_viz} in Section \ref{sec:introduction} (showing variation in
% average capital gain across sex for the single and married adults) was a visualization 
% classified as interesting by 4 or 5 participants. 
% In contrast, Figure \ref{fig:uninteresting_viz} was a visualization unanimously classified
% as uninteresting.

% Once we obtained ground truth, we evaluated whether \SeeDB could correctly
% classify visualizations with respect to ground truth.
% In all, participants selected 23 unique visualizations as being interesting.
% To obtain a consensus on ground truth, we used a simple voting system.
% Any visualization that was chosen by majority of participants (3 or more)
% was considered to be interesting; the rest were not.
% Of the 23 unique visualizations classified as interesting, majority of participants 
% agreed on 6 visualizations as being interesting.
% Observe that while there are subjective differences in the criteria for interesting-ness,
% it is possible to distill general criteria for interesting-ness of visualizations.

\begin{figure}[t]
	\centering
	\begin{subfigure}{0.32\linewidth}
		{\includegraphics[trim={0 1.3cm 0 0}, clip, width=2.5cm]{Images/census_gt_distribution.pdf}}
		\caption{Utility Distribution}
		\label{fig:gt_dist}
	\end{subfigure}
	\begin{subfigure}{0.65\linewidth}
		\centering 
		{\includegraphics[width=4cm] {Images/seedb_roc.pdf}} 
		\caption{ROC of SeeDB (AUROC = 0.903)}
		\label{fig:roc}
	\end{subfigure}
	\vspace{-10pt}
	\caption{Performance of Deviation metric for Census data}
	\vspace{-20pt}
	\label{fig:census_gt}
\end{figure}

\stitle{Efficacy of Deviation-based Metric}.
Figure \ref{fig:gt_dist} shows a heatmap of the number of times a
visualization was classified as interesting 
({\em yellow} = popular, {\em blue} = not popular), sorted
in {\em descending order} of our utility metric.
We notice that the majority of yellow bands fall at the top of the
heatmap, indicating, qualitatively, that popular visualizations have higher utility.
To evaluate the accuracy of \SeeDB's recommendations over the Census data, 
we ran \SeeDB for the study task, varying $k$ between 0 \ldots 48, and measured
the agreement between \SeeDB recommendations and ground truth.
As is common in data mining, 
we computed the ``receiver operating curve'' or ROC curve for \SeeDB, 
Figure \ref{fig:roc}, depicting 
the relationship between the true positive rate (TPR) on the 
x-axis and false positive rate (FPR) on the y-axis for different values of a
parameter ($k$ in this case). 
TPR is the number of interesting visualizations returned as a fraction of the 
total number of interesting visualizations, 
while FPR is the fraction of recommendations
that were incorrectly returned as interesting, as a fraction of the number
of interesting visualizations returned. 
ROC curves for highly accurate classifiers are skewed towards the upper left
corner of the graph. The red line indicates the random baseline (every example
is classified randomly).
As can be seen in the figure, \SeeDB performs significantly better than the baseline.
For example, for $k$=3, all 3 visualizations recommended by \SeeDB are interesting,
giving TPR = 0.5 and FPR = 0; for $k$=5, 
four of the 5 recommended visualizations are interesting, 
giving TPR = 4/6 = 0.667 and FPR = 0.05. 
The area under ROC (AUROC) for \SeeDB---the typical measure of 
classifier quality---is 0.903.
This indicates that the accuracy of \SeeDB recommendations is very high.\footnote{\small AUROC's 
above 0.8 are considered very good, while those above 0.9 are
excellent}
%\mpv{add points on chart for k=3, 5}

While ROC curves on different datasets and tasks will vary,
this user study shows that \SeeDB recommendations have high quality
and coverage,
despite focusing on a simple deviation-based utility metric.
We expect that taking into account other aspects (apart from deviation),
would improve \SeeDB's recommendations even more.

% We varied the number of visualizations $k$ recommended by \SeeDB between 0 and
% max \mpv{fill in max}.
% For each value of $k$ we computed the number of true positives (\SeeDB recommended
% visualizations that were classified as ``interesting'' by majority), false
% positives, true negatives and false negatives.
% The true positive rate (recall) and false positive rate for \SeeDB are shown 
% in the `receiver operating curve'' (ROC)\cite{} in Figure \ref{fig:roc}.
% As expected, we observe that as $k$ increases, the true positive rate (TPR)
% increases (or stays constant), but false positive rate (FPR) also increases as 
% more visualizations are incorrectly classified as interesting.
% Clearly, \SeeDB performs significantly better than the baseline algorithm which 
% classifies every visualization as interesting with 50\% probability.
% For example, for $k$ = 3, TPR = 0.5 and FPR = 0;
% i.e., for $k$ = 3, all 3 visualizations recommended by \SeeDB are in fact interesting.
% However, \SeeDB recovers only 3 of the 6 visualizations.
% Likewise, for $k$ = 5, 4 of 5 of the recommended visualizations are interesting, giving
% TPR = 0.667 and FPR = 0.05.
% Finally, for $k$=14, \SeeDB recommends all 6 interesting visualizations, giving a 
% TPR of 1 but having an FPR of 0.421.
% The standard metric for computing quality of a recommender is to
% compute AUROC or area under the ROC curve.
% AUROC for \SeeDB in \ref{fig:roc} is 0.903.
% AUROC values above 0.8 are indicative of
% high quality of recommendations \cite{}, demonstrating that \SeeDB performs very well in
% making recommendations.

% While ROC curves for \SeeDB will vary with dataset and query, the above analysis, both
% qualitatively and via ROC, indicates that our deviation-based metric can in fact identify
% interesting visualizations with high accuracy.
% Although there are many other factors that determine interesting-ness, deviation seems
% to capture a significant part of the metric. 

% Now that we have validated our deviation-based metric, we examine how \SeeDB, a visualization tool
% with deviation-based recommendations, compares to a manual chart construction tool in performing
% visual analysis.

\subsection{{\large \SeeDB} vs. Manual Visualization Tool}
\label{sec:seedb_vs_manual}

%The motivation behind \SeeDB is to build a tool that can support fast
%visual analysis by automatically recommending interesting visualizations.
In this section, we describe results from a controlled user study comparing 
\SeeDB to a manual visualization tool for performing visual analysis. 
We hypothesized that: (i) when using \SeeDB, analysts would find interesting 
visualizations {\em faster} than when using the manual tool, (ii) analysts
would find {\it more} interesting visualizations when using \SeeDB vs. the 
manual tool, and (iii) analysts would {\em prefer} using \SeeDB to a manual tool.

% Next, we assess the efficacy of \SeeDB in enabling fast visual analysis.
% Towards this, we 
% To assess the efficacy of \SeeDB in enabling faster visual analysis,
% we conducted a user study where participants performed visual analysis
% using both \SeeDB and a manual chart construction tool.
% We hypothesized that: (i) When using \SeeDB, participants would find 
% interesting visualizations {\em faster} than when using manual chart
% construction, (ii) Participants would find more interesting visualizations
% when using \SeeDB vs. when using manual chart construction, (iii) 
% Participants would prefer using a tool with recommendations vs. a manual
% construction tool.

\stitle{Participants and Datasets}. We recruited 16 participants (5 female, 11
 male) all graduate students with prior data analysis experience and visualization
 experience (e.g. R, matplotlib or Excel).
 None of the participants had previously worked with the study datasets.

 Our study used the Housing and Movies datasets from 
 Table \ref{tab:datasets}.
 These  datasets were chosen because they were easy to understand and 
  comparable in size and number of potential visualizations.
 
\stitle{Study Protocol}.
Our study used a 2 (visualization tool) X 2 (dataset) within-subjects design.
The visualizations tools used were \SeeDB and {\em MANUAL}, a manual chart
construction-only version of \SeeDB (i.e., \SeeDB with the recommendations bar, 
component ``D'' in Figure \ref{fig:frontend1}, removed).
Using the same underlying tool in both modes allowed us to control for
tool functionality and user interface.
We used a within-subjects design to compensate for per-participant differences 
in data analysis expertise, and used counterbalancing to remove any effects 
related to order and the test dataset.

Our study began with a short tutorial on the two study tools.
Following the tutorial, participants were asked to perform two visual analysis 
tasks, one with \SeeDB in each mode.
For each mode, we introduced participants to the test dataset
and the analytical prompt using written instructions.
Each analytical task asked participants to use the specified tool to find 
visualizations supporting or disproving a specific hypothesis.
Participants were asked to use the bookmark button (in component ``C'' in Figure 
\ref{fig:frontend1}) to flag any visualizations they deemed interesting in
context of the task.
Participants were also encouraged to think aloud during the study.
Since the analytical tasks were open-ended, we capped each analysis session at 8 minutes.
Participants filled out a tool-specific survey at the end of each task and
an exit survey at the end of the study.
Most survey questions were answered on a 5-point Likert scale.
The study lasted \textasciitilde 45 minutes and participants were compensated 
 with a \$15 gift card.
All studies were conducted in a lab setting using Google Chrome on a 15-inch 
Macbook Pro.

\stitle{Methods and Metrics}.
Over the course of each study session, we collected data by three means: interaction logs 
from each tool, responses to surveys, and exit interview notes.
The interaction logs capture the number of visualizations
constructed, the number of visualizations bookmarked, bookmark rate, and interaction traces.
\SeeDB and MANUAL both support the construction of different types of charts such as bar 
charts, scatterplots etc.
Since \SeeDB can only recommend aggregate visualizations shown as bar charts,
we report results for aggregate visualizations.
We evaluate statistical significance of our results using paired t-tests and ANOVA,
and supplement interaction analysis with qualitative observations.


% Since users were asked to bookmark visualizations they found to be relevant to the task,
% bookmarking behavior from tool interaction logs provides a rich source
% of information about the analytical process.
% Specifically, we study a number of metrics including: (i) number of bookmarks ($num\_bookmarks$), 
% (ii) total number of visualizations viewed ($total\_viz$), 
% (iii) bookmarking rate ($bookmark\_rate$) defined as $num\_bookmarks$/$total\_viz$, and 
% (iv) the time between consecutive bookmarks ($bookmark\_time$).
% \SeeDB and MANUAL support construction of two kinds of charts: aggregate visualizations and 
% scatterplots (to replicate real visual analysis).
% Since \SeeDB can only recommend aggregate visualizations, we also analyze the above
% metrics for aggregate visualization only.
% We supplement bookmark analysis with qualitative data from surveys and study notes.
% We evaluate statistical signifance of our results using paired t-tests and ANOVAs.

\stitle{Results}.
Over the course of our study, participants built over 220 visualizations 
and bookmarked 70 visualizations (32\% bookmark rate).
We next describe our key findings and observations.

\stitle{1. \SeeDB enables fast visual analysis}.
Table \ref{tab:agg_bookmarks} shows an overview of the bookmarking behavior for each tool
focusing on total number of visualizations generated, number of bookmarks and bookmarking rate.
First, we observe that the total number of (aggregate) visualizations created in the \SeeDB
condition is higher than that for MANUAL. 
While not statistically significant, this difference suggests that analysts are exposed to more
{\em views} of the data with \SeeDB than MANUAL, possibly aiding in a more thorough exploration of
the data.
Next, we find that the number of aggregate visualizations bookmarked in \SeeDB is much higher (3X more)
than that for MANUAL.
In fact, the two-factor analysis of variance shows a significant effect of tool on the number of bookmarks,
F(1,1) = 18.609, p < 0.001. 
We find no significant effect of dataset, F(1, 1) = 4.16. p > 0.05, or
significant interaction between tool and dataset.
While this result indicates that analysts bookmark more visualizations in \SeeDB, we note that the number of 
bookmarks for a tool may be affected by the total number of visualizations built with the tool.
Therefore, to account for variance in the total number of visualizations, we also examine $bookmark\_rate$
for the two tools defined as the fraction of
created visualizations that are bookmarked ($\frac{num\_bookmarks}{total\_viz}$).
We find, once again, that the $bookmark\_rate$ for \SeeDB (0.42) is 3X larger than the $bookmark\_rate$ for 
MANUAL (0.14).
The two-factor analysis of variance shows a significant effect of tool on bookmark rate, F(1,1) = 10.034, p < 0.01.
As before, we find no significant effect of dataset on bookmark rate, F(1, 1) = 3.125. p > 0.05, or
significant interaction between tool and dataset.
Together the two results above indicate that there is a {\bf significant effect of tool on both the number of bookmarks as well as the
bookmark rate}.
\SeeDB-recommended visualizations are 3 times more likely to be interesting compared
to manually constructed visualizations.
%In other words, analysts are 3X more likely to arrive at interesting visualizations
%when using \SeeDB vs. MANUAL:
%\SeeDB can thus allow analysts to find analyze data faster. 
%\mpv{Remove? Finally, we note that the statistical results regarding faster analysis are supported by anecdotal evidence 
%and survey data from study participants.
Finally, 87\% of participants indicated that \SeeDB recommendations sped up their visual analysis, many alluding
to the ability of \SeeDB to ``\ldots quickly deciding what correlations are relevant'' and 
``[analyze]...a new dataset quickly''.


% Moreover, we once again find this difference in $bookmark\_rate$ is statistically significant
% within subjects as well as across subjects ({\em Paired t-test, t = -2.5599, df = 8, p-value = 0.03365}).
% This implies that {\bf a \SeeDB-recommended visualization is 3 times more likely to be
% interesting compared to a manually constructed aggregate visualization}.
% In other words, users are 3X more likely to arrive at interesting visualizations when using
% \SeeDB vs. MANUAL; i.e., \SeeDB can enable users to find interesting insights faster.
% Results of a 2-way ANOVA also indicate that \SeeDB has a significant impact on 
% aggregate bookmark rate ({\em df = 1, sum sq = 0.3681, mean sq = 0.3681, F value = 10.034, p = 0.00685}). 
% (We find that choice of dataset does not affect bookmark rate, and there are no interaction or order effects.)

% We also find that, on average, the $bookmark\_time$ for \SeeDB (92.91 $\pm$ 49.26) is twelve seconds 
% shorter than that for MANUAL (105.02 $\pm$ 58.24). 

% However, unlike in Table \ref{tab:bookmarks}, we observe that number of aggregate visualizations 
% is higher for \SeeDB compared to MANUAL suggesting that the larger number of bookmarks in \SeeDB 
% might be a consequence of a larger number of aggregate visualizations with \SeeDB (possibly because
% \SeeDB only recommends aggregate visualizations)\footnote{Although \SeeDB only recommends aggregate
% visualizations, users have the ability to plot or modify a recommendation to construct a scatterplot}.
% As a result, we find $bookmark\_rate$, which is the proportion of aggregate visualizations bookmarked
% to the number of aggregate visualizations viewed, to be an unbiased metric.
% We find that $bookmark\_rate$ for \SeeDB (0.42) is in fact 3X larger than the $bookmark\_rate$ for 
% MANUAL (0.14).
% This implies that \stitle{a \SeeDB recommended aggregate visualization is 3 times more likely to be
% interesting compared to a manually constructed aggregate visualization}.

% A 3X higher $bookmark\_rate$ implies that users are 3X more likely to find interesting insights with 
% recommended visualizations vs. if they create visualizations manually; i.e., \SeeDB can enable users 
% to find interesting insights faster.

\begin{table}[htb]
  \centering \scriptsize
   \vspace{-5pt}
  \begin{tabular}{|c|c|c|c|c|} \hline
   & total\_viz & num\_bookmarks & bookmark\_rate \\ \hline
  MANUAL & 6.3 $\pm$ 3.8 & 1.1 $\pm$ 1.45 & 0.14 $\pm$ 0.16 \\ \hline
  \SeeDB & 10.8 $\pm$ 4.41 & 3.5 $\pm$ 1.35 & 0.43 $\pm$ 0.23 \\ \hline
  \end{tabular}
  \vspace{-10pt}
  \caption{Aggregate Visualizations: Bookmarking Behavior Overview}
  \label{tab:agg_bookmarks} 
  \vspace{-5pt}
\end{table}



% We find that, in general, participants bookmarked slightly more visualizations with \SeeDB than 
% with MANUAL.
% On the other hand, we find that participants interacted with fewer visualizations in \SeeDB than
% in MANUAL.
% As a consequence of these two opposing forces, we find that participants using \SeeDB view fewer
% visualizations but bookmark more, i.e., the visualizations they interact with are, on average, 
% higher quality.
% This trend id reflected in the $bookmark\_rate$ for \SeeDB; the $bookmark\_rate$ for \SeeDB is 1.5X 
% higher than that for MANUAL.

% While these differences are not statistically significant, they point towards a trend: {\it \SeeDB
% enables participants to arrive at interesting visualizations faster than MANUAL}.

% \begin{table}[htb]
%   \centering \scriptsize
%   \begin{tabular}{|c|c|c|c|c|} \hline
%    & num\_bookmarks & total\_viz & bookmark\_rate \\ \hline
%   MANUAL & 3.3 $\pm$ 1.42 & 14.1 $\pm$ 5.4 & 0.24 $\pm$ 0.09 \\ \hline
%   \SeeDB & 3.5 $\pm$ 1.35 & 12.1 $\pm$ 4.7 & 0.36 $\pm$ 0.22 \\ \hline
%   \end{tabular}
%   \vspace{-10pt}
%   \caption{All Visualizations: Bookmarking behavior Overview}
%   \label{tab:bookmarks} 
%   \vspace{-10pt}
% \end{table}



% Recall that \SeeDB (currently) only supports recommendations for aggregate visualizations.
% The above results include data for both scatterplots as well as aggregate visualization, and
% therefore do not entirely reflect bookmark behavior for aggregate visualizations.
% Table \ref{tab:agg_bookmarks} shows the same bookmarking metrics as in Table \ref{tab:bookmarks}
% for aggregate visualizations.

\stitle{2. All participants preferred \SeeDB to MANUAL}. 
100\% of all users preferred \SeeDB to MANUAL for
visual analysis, i.e., all users preferred to have recommendation support during analysis.
78\% of participants found the recommendations ``Helpful'' or ``Very Helpful'' and thought that they
showed interesting trends.
In addition, a majority of users found \SeeDB a powerful means to get an overview of interesting trends
and starting points for further analysis. 
One participant noted that \SeeDB was ``\ldots great tool for proposing a set of initial queries for a dataset''.
%Due to space constraints, we explore this result further in the associated tech report~\cite{seedb-tr}.
77\% of participants also indicated that \SeeDB visualizations showed unexpected trends (e.g., the difference
in capital gain in Figure \ref{fig:huhi}), and indicated that \SeeDB suggested visualizations
they wouldn't have created (e.g. although users did not manually generate Figure \ref{fig:interesting_viz}, it was
bookmarked by majority of our experts).
%e.g., ``\ldots interesting aspects of data to compare. I don't think I would have checked those by myself.''.
% To illustrate, Figure \ref{blah} \mpv{making these} shows two visualizations that were not generated by participants using MANUAL,
% but were in fact recommended by \SeeDB\ {\it and} bookmarked as being interesting.
%\papertext{An intriguing observation we made was that while some analysts liked having recommendations, 
%they did not want to rely 
%too heavily on recommendations but instead let intuition guide their analysis.}
\techreport{An intriguing observation from two participants was that while they wanted recommendations to support them
in analysis, they did not want to rely too heavily on recommendations and ignore their creativity.
One participant noted {\em ``The only potential downside may be that it made 
me lazy so I didn't bother thinking as much about what I really could study or be interested in''}.
This observation suggests lines for future work that can find the right balance between automatically 
recommending insights and allowing the user to leverage their intuition and creativity.}

\techreport{
\begin{figure}
	\centering
	{\includegraphics[trim={0 0 0 0}, clip, width=9cm]{Images/traces.pdf}}
	\caption{Interaction trace examples: (R) = Recommendation, (M) = Manual, (B) = Bookmark}
	\vspace{-10pt}
	\label{fig:traces}
\end{figure}

\stitle{3. \SeeDB provides a starting point for analyses}. 
To our knowledge, \SeeDB is the first tool to provide recommendations for supporting visual
analysis.
As a result, we were interested in how recommendations could fit into the analytical workflow.
While a participant's exact workflow was unique, we repeatedly found specific patterns in the
interaction traces of \SeeDB.
Figure \ref{fig:traces} shows examples of three such traces.
Interaction traces show that participants often started with a recommended visualization, 
examined it, modified it one or more times (e.g. by changing to a different aggregate function 
or measure attribute) and bookmarked the resulting visualization.
%  {\em recommendation 
% $\rightarrow$ manual\_modify $rightarrow$ manual\_modify $rightarrow$ \ldots bookmark}.
% Specifically, participants would often start from one of the recommendations and explore other
% visualizations that were variations of it (e.g. different aggregation or measure attribute) 
% until they found an interesting visualization.
% \mpv{can I put any data here?}
Thus, even if participants did not bookmark recommendations directly, their often created
small variations of the visualization and bookmarked them.
In other words, along with providing recommendations that were interesting by themselves, \SeeDB
helped direct participants to other interesting visualizations by {\em seeding} their analysis.
This pattern was highlighted in user comments as well; e.g.,
``\ldots would be incredibly useful in the initial analysis of the data'', 
``\ldots quickly deciding what correlations are relevant and gives a quick peek'',
``\ldots great tool for proposing a set of initial queries for a dataset''.
In addition to understand the role recommendations played in analysis, these observations 
also serve to reinforce the design choice of \SeeDB as a complement to a traditional
visualization system vs. a standalong system; the mixed-initiative nature of the tool 
is essential for it to be functional in visual analysis.
}


\techreport{
\subsection{Limitations}
Given that both studies described above were conducted in the lab, the studies had limitations.
First, due to constraints on time and resources, the sample sizes for both studies were small.
A larger set of participants and spread of datasets could be used to further demonstrate the
efficacy of our system.
Second, our user studies were conducted with graduate students participants who, on one hand, 
likely have higher data analysis skills than typical users, while on the other hand, are 
not experts in the dataset being analyzed.
Consequently, our results represent results the perspective of capable data analysts who 
have limited familiarity with the data.
We find that \SeeDB is particularly well suited for this particular setting of initial data 
analysis when the user is not very familiar with the data (\~ coldstart).
It would be instructive to evaluate \SeeDB on datasets about which users have expert knowledge.
Finally, we note that being a research prototype, limited functionality of \SeeDB (e.g. in types of
charts) and potential issues with learnability and interactivity may have also had an impact on
our study.

\mpv{Also: The datasets we evaluated had a relatively small (< 100) number of potential visualizations;
it would be valuable to evaluate the performance of \SeeDB on datasets with thousands of potential
visualizations.
It is also possible that some datasets were easier to interpret than others.
}
}



% 86\% of participants indicated that the recommendations sped up their analysis.


% When asked to rate the recommendations provided by \SeeDB, 78\% participants indicated that the
% recommendations were either ``Helpful'' or ``Very Helpful''. 
% We also found that 90\% of participants found the comparative visualizations shown by \SeeDB 
% helpful in their analysis.
% 66\% of participants indicated that the recommendations needed improvement, in particular,
% participants were interested in seeing different types of charts (e.g. geographical, time series)
% and obtaining measures of statistical significance.

% \stitle{Qualitative Feedback}. In their qualitative feedback, participants highlighted the importance of 
% a tool like \SeeDB at the initial stages of analysis. 
% One partitipant said, {\em ``It's a great tool for proposing a set of initial queries for a dataset I have never seen. 
% And from these visualizationns, I can figure out which related patterns to dig into more.''}
% Others thought that the strength of the tool was in quickly finding relevant trends, {\em ``It's a good tool that helps 
% in quickly deciding what correlations are relevant and gives a quick peek''}. 
% Overall, participants indicated that \SeeDB was particularly suited for exploratory analysis of new datasets, 
% {\em ``I thought SeeDB was very helpful in helping me get more familiar with a new dataset quickly.''}.




% %!TEX root=document.tex

\section{Experimental Evaluation}
\label{sec:experiments}
 
In this section, we present our evaluation of \SeeDB in terms of its performance 
at exploring alternative visualizations.
Results of our user study are presented in the following section.
We begin with a study of the DBMS-backed execution engine and examine how far we can
push conventional relational engines to support a \SeeDB workload.
The results motivate empirically the need for our custom execution engine.
We then present our evaluation of the custom execution engine and pruning strategies.
The datasets used in our experiments are listed in Table~\ref{tab:datasets}.
We test our 
techniques on a variety of syntheic as well as real datasets to evaluate 
their performance and accuracy.
All experiments were run on a single machine with 8 GB RAM and a 16 core Intel 
Xeon E5530 processor. 
In these experiments, use the {\it earth mover distance (EMD)} as our utility function.
% Unless described otherwise, experiments were repeated 3 times and the latency measures 
% were averaged.
\srm{small data, attributes}
\begin{table}[htb]
  \centering \scriptsize
  \begin{tabular}{|c|c|c|c|c|c|} \hline
  Name & Description & Size & Dims & Measures & Views \\ \hline
  % SYN1 & Synthetic data & 1M & 50 & 5 & 250 \\
  % & Randomly distributed, & & & & \\ 
  % & varying \# distinct values & & & & \\ \hline
  SYN & Synthetic data & 1M & 50 & 20 & 1000 \\
  & Randomly distributed, & & & & \\ 
  & varying \# distinct values & & & & \\ \hline
  SYN*-10 & Synthetic data & 1M & 20 & 1 & 20 \\
  & Randomly distributed, & & & & \\ 
  & 10 distinct values/dim & & & & \\ \hline
  SYN*-100 & Synthetic data & 1M & 20 & 1 & 20 \\
  & Randomly distributed, & & & & \\ 
  & 100 distinct values/dim & & & & \\ \hline
  BANK  & Customer Loan dataset  \mpv{fix} & 40K & 10 & 8 & 80* \\ \hline
  DIAB  & Hospital data \mpv{fix} & 100K & 10 & 8 & 80* \\
  & about diabetic patients & & & & \\ \hline
  \end{tabular}
  \vspace{-10pt}
  \caption{Datasets used for testing}
  \label{tab:datasets} 
  \vspace{-10pt}
\end{table}





%!TEX root=document.tex


% \begin{figure*}[t]
% 	\centering
% 	\begin{subfigure}{0.33\linewidth}
% 		{\includegraphics[width=6cm] {Images/baselines_by_size.pdf}}
% 		\caption{Latency vs. Table size}
% 		\label{fig:baseline_size}
% 	\end{subfigure}
% 	\begin{subfigure}{0.33\linewidth}
% 		\centering
% 		{\includegraphics[width=6cm] {Images/baselines_by_views.pdf}}
% 		\caption{Latency vs. Num Views}
% 		\label{fig:baseline_views}
% 	\end{subfigure}
% 	\begin{subfigure}{0.33\linewidth}
% 		{\includegraphics[width=6cm] {Images/multi_agg.pdf}}
% 		\caption{Latency vs. number of aggregates}
% 		\label{fig:multi_agg}
% 	\end{subfigure}
% 	\vspace{-10pt}
% 	\caption{Baseline performance and Effect of Combining Aggregates }
% 	\vspace{-10pt}
% 	\label{fig:bank_perf}
% \end{figure*}


\begin{figure}[h]
	\centering
	\vspace*{-10pt}
	\begin{subfigure}{0.48\linewidth}
		\centering
		\includegraphics[width=4.4cm] {Images/baselines_by_size.pdf}
		\vspace{-15pt}
		\caption{Latency vs. Table size}
		\label{fig:baseline_size}
	\end{subfigure}
	\begin{subfigure}{0.48\linewidth}
		\centering
		\includegraphics[width=4.4cm] {Images/baselines_by_views.pdf}\
		\vspace{-15pt}
		\caption{Latency vs. Num Views}
		\label{fig:baseline_views}
	\end{subfigure}
	\vspace{-10pt}
	\caption{Baseline performance}
	\label{fig:bank_perf}
		\vspace{-10pt}
\end{figure}

\begin{figure}[h]
	\centering
	\vspace*{-10pt}
	\begin{subfigure}{0.48\linewidth}
		\centering
		\includegraphics[width=4.4cm] {Images/multi_agg.pdf}
		\vspace{-15pt}
		\caption{Latency vs. aggregates}
		\label{fig:multi_agg}
	\end{subfigure}
	\begin{subfigure}{0.48\linewidth}
		\centering
		\includegraphics[width=4.4cm] {Images/parallel_noop.pdf}\
		\vspace{-15pt}
		\caption{Effect of parallelism}
		\label{fig:parallelism}
	\end{subfigure}
	\vspace{-10pt}
	\caption{Effect of Group-by and Parallelism}
\end{figure}


\begin{figure}[h]
	\centering
	\vspace*{-10pt}
	\begin{subfigure}{0.48\linewidth}
		\centering
		\includegraphics[width=4.4cm] {Images/multi_gb_same.pdf}
		\vspace{-15pt}
		\caption{Latency vs. Num of Groups}
		\label{fig:multi_gb_same}
	\end{subfigure}
	\begin{subfigure}{0.48\linewidth}
		\centering
		\includegraphics[width=4.4cm] {Images/multi_gb.pdf}\
		\vspace{-15pt}
		\caption{Latency vs. Num Dimensions}
		\label{fig:multi_gb_bp}
	\end{subfigure}
	\vspace{-10pt}
	\caption{Effect of Groups and Dimensions}
	\vspace{-15pt}
\end{figure}



% \begin{figure*}[t]
% 	\centering
% 	\begin{subfigure}{0.33\linewidth}
% 		\centering
% 		{\includegraphics[width=6cm] {Images/parallel_noop.pdf}}
% 		\caption{Effect of parallelism}
% 		\label{fig:parallelism}
% 	\end{subfigure}
% 	\begin{subfigure}{0.33\linewidth}
% 		\centering
% 		{\includegraphics[width=6cm] {Images/multi_gb_same.pdf}}
% 		\caption{Latency vs. Num of Groups}
% 		\label{fig:multi_gb_same}
% 	\end{subfigure}
% 	\begin{subfigure}{0.33\linewidth}
% 		\centering
% 		{\includegraphics[width=6cm] {Images/multi_gb.pdf}}
% 		\caption{Latency vs. Num Dimensions}
% 		\label{fig:multi_gb_bp}
% 	\end{subfigure}
% 	\vspace{-10pt}
% 	\caption{Effect of Combining Group-by attributes and Parallel Query Execution}
% 	\label{fig:bank_perf}
% 	\vspace{-10pt}
% \end{figure*} 

\subsection{Basic {\large \SeeDB} Framework}
\label{sec:basic_framework_expts}

\noindent {\em \underline{Summary:} Applying no optimizations 
leads to latencies in the 100s of seconds for both ROW and COL;
latency increases linearly in the size of the dataset and  number
of views. 
%Column stores are superior to row stores,
%with $\frac{1}{5}$th the latency.
}
Without any optimizations, the basic \SeeDB framework
serially executes two SQL queries for each
possible view.
Figure \ref{fig:baseline_size} shows latency of \SeeDB\ vs. the number of rows (100K rows--1M rows) 
in the dataset, while Figure \ref{fig:baseline_views} shows latency as a function 
of the number of views (50--250).
These charts show results for the SYN dataset obtained by varying the size of the
table and number of attributes (SYN is comparable to the AIR dataset).
% and we created subsets of the dataset with
% varying numbers of rows and views, by varying the number of dimension attributes. 
First, notice that the basic framework with no optimizations has very 
poor performance: latency for ROW is between 50-500s, 
while it is between 10-100s for COL. 
This is because, depending on the dataset, both ROW and COL run between 50 to 250 SQL queries 
for each \SeeDB invocation.
%Clearly, latencies associated with the basic \SeeDB framework are not practical for interactive 
%applications. 
Second, COL runs about 5X faster than ROW. 
This is expected because most queries only select a few attributes,
benefitting from a column layout.
Third, as expected, the latency of the
basic framework is proportional to the number of rows as well as the 
number of views in the table.
Since the latencies for the basic framework are very high for interactive
applications, it is clear that aggressive optimization needs to be employed.

\subsection{Sharing Optimizations}
\label{sec:expts_dbms_execution_engine}

% Our primary metric for evaluating the DBMS-backed
% execution engine is {\em latency},
% i.e., .
% Specifically, we study the impact of the following properties on latency:
% (a) parameters of the dataset including size and number of views.
% (b) the optimizations detailed in Section~\ref{sec:dbms_optimizations}, and
% (c) 
% Our goal is to identify which data layout is a better
% fit for \SeeDB, and study the gains of each optimization
% for row-stores vs.~column stores. 
In this section, we study the performance of sharing optimizations described in Section \ref{sec:sharing_opt}.
Recall that the goal of these optimizations is to reduce the number of queries run against the DBMS and to share scans as much as possible between queries.
The following experiments report results on the synthetic datasets SYN and SYN* 
(Table~\ref{tab:datasets}).
We chose to test these optimizations on synthetic data since we can control all parameters of the data including size, number of attributes, and data distribution.
(Results on real datasets are shown in Figure \ref{fig:share_prune_row} and \ref{fig:share_prune_col}).

% \stitle{Result Highlights:}
% \begin{denselist}
% \item Our optimizations to the DBMS-based execution engine can reduce
% latency on large datasets from 500 seconds (ROW) and 100 seconds (COL)
% to $<$10 seconds for COL, and $<$20 seconds for ROW.
% % This is particularly noteworthy given that
% % \SeeDB is running 100s of queries on the DBMS for each \SeeDB query.

% \item Latency scales linearly 
% with dataset size and number of views.

% \item Column stores are superior to row stores 
% for our workload. Row stores, however, benefit more from the
% proposed optimizations. 
% % For instance, row stores benefit
% % significantly (with reductions of up to 2.5X on latency) from applying
% % bin-packing-based algorithms for aggregation, 
% % while column stores are not significantly affected. 

% \item Although our optimizations lead to a 8X-20X speedup, the total latency remains in the 10s of seconds.
%  % - a number unacceptable in interactive systems.
% In addition, each \SeeDB query translates to 50+ SQL queries. 
% This {\it query bloat} unnecessarily consumes DBMS resources including memory and CPU.
% Consequently, we see the need for a custom solution with  lower latencies and a
% smaller resource footprint.
% \end{denselist}


% As mentioned in Section \ref{sec:dbms_execution_engine}, our DBMS-based
% execution engine leverages the DBMS API to execute view queries directly on the
% database.
% While this approach has the advantages of reusing existing query procesing
% systems and being agnostic to the specific underlying DBMS, its limitations
% include the lack of fine grained control over sharing of table scans and lack of
% ability to prune low-utility views. 

% We start with an evaluation of the basic framework (without any optimizations) 
% and then study the effect of the proposed optimizations (Section~\ref{sec:dbms_optimizations}).
% Our goal is to characterize the effect of each optimization and find the optimal parameter 
% settings for the DBMS execution engine.
% , i.e., 
% how long do row and column stores take if each view is evaluated 
% in sequence, without any optimization. 
% We then study the effect of adding each optimization from Section~\ref{sec:dbms_optimizations} 
% in turn.

 

% \begin{figure}[h] 
% \centerline{
% \resizebox{4cm}{!} {\includegraphics {Images/baselines_by_size.pdf}}
% \resizebox{4cm}{!} {\includegraphics {Images/baselines_by_views.pdf}}
% }
% \end{figure}



% \begin{figure}[h]
% \centering
% \begin{subfigure}{0.49\linewidth}
% \centering
% {\includegraphics[width=4.2cm] {Images/baselines_by_size.pdf}}
% \caption{Latency vs. Table size}
% \label{fig:baseline_size}
% \end{subfigure}
% \begin{subfigure}{0.49\linewidth}
% \centering
% {\includegraphics[width=4.2cm] {Images/baselines_by_views.pdf}}
% \caption{Latency vs. Num Views}
% \label{fig:baseline_views}
% \end{subfigure}
% \label{fig:baselines}
% \caption{Latency of Basic Framework}
% \end{figure}

\stitle{Combining Multiple Aggregates:} 
{\em \underline{Summary:} Com\-bining 
view quer\-ies with the same group-by attribute
but different aggregates gives a 
3-4X speedup for both row and column stores.}
To study the limits of adding multiple aggregates to a single view query, we
varied the maximum number of aggregates in any \SeeDB-generated SQL query 
($n_{agg}$) between 1 and 20.
% We combine multiple view queries that have the same group-by (dimension) 
% attribute, but different aggregation (measure) attributes into one query.
% We ran these experiments on the SYN2 dataset 
% since it has a large number (20) of measure attributes.
% We varied the number of aggregate attributes 
% in a query between 1 and 20 (i.e., we grouped $n_{agg}$ view
% queries that share the same group-by attribute into one query
% that is issued to the DBMS).
The resulting latencies on the SYN dataset are shown in Figure \ref{fig:multi_agg} (log scale on the y-axis).
As we can see, latency reduces consistently with the number of aggregations performed 
per query.
However, the latency reduction is not linear in $n_{agg}$ because
larger $n_{agg}$ values require maintenance of more state and access more columns in a 
column store.
Overall, this optimization provides a 4X speedup for ROW and 3X for COL.

\stitle{Parallel Query Execution:} 
{\em \underline{Summary:} Running view queries in parallel can offer significant
performance gains.}
Executing \SeeDB-generated SQL queries in parallel can provide significant performance gains
because queries can share buffer pool pages.
However, a high degree of parallelism can degrade performance for a variety of reasons \cite{Postgres_wiki}. 
Figure \ref{fig:parallelism} shows how latency varies with the number of parallel SQL queries
issued by \SeeDB.
As expected, low levels of parallelism produce sizable performance gains but
high levels degrade performance.  The optimal number of queries to 
run in parallel is approximately $16$ (equal to the number of cores),
suggesting that choosing a degree of parallelism equal to the number of cores
% for these kinds of simple aggregation queries when data largely fits in memory,
is a reasonable policy. 
% \mpv{separate this from general study of parallel queries}

% \begin{figure}[h]
% \centering
% {\includegraphics[width=6cm] {Images/multi_agg.pdf}}
% \caption{Latency vs. number of aggregates}
% \label{fig:multi_agg}
% \end{figure} 

\stitle{Combining Multiple Group-bys:} 
{\em \underline{Summary:} Combining multiple view queries, each with a single group-by attribute into 
a single query with multiple group-by attributes improves performance by 2.5X in row stores.}
% We now study the effect of combining multiple queries each with a single group-by into one query
% with multiple grouping.
Unlike the multiple aggregates optimization, the impact of combining multiple group-bys attributes
into one query is unclear due to the (potentially) significantly larger memory utilization.
We claim in Section \ref{sec:sharing_opt} that grouping can benefit performance so long as 
the total memory utilization stays under a threshold.
To verify our claim, we ran an experiment with datasets SYN*-10 and SYN*-100.
For each dataset, we varied the number of group-by attributes in \SeeDB-generated SQL 
queries ($n_{gb}$) between 1 and 10.
Since each attribute in SYN*-10 has 10 distinct values and that in SYN*-100
has 100, a query with $n_{gb}=p$ will require memory proportional to $\max(10^p$,
num\_rows) for SYN*-10 and proportional to $\max(100^p$, num\_rows) for SYN*-100.
The results of the experiment are shown in Figure \ref{fig:multi_gb_same}.
We see that as the number of group-by attributes increases from 1 to 10, 
the latency of ROW (blue) decreases initially.
However, once the memory budget $\mathcal{S}_{ROW}$ (proxied by the number of distinct groups) exceeds 10000, 
latency increases significantly.
We see a similar trend for COL, but with a memory budget $\mathcal{S}_{COL}$ of 100.\footnote{\scriptsize The different memory
budgets can be explained based on the different internal parameteres and implementations of
the two systems.}
Thus, we find empirically that memory usage from grouping is, in fact, related to latency and that 
optimal groupings must respect the memory threshold.

% We see that for the row-store, latency does improve (and then gets worse) as the
% number of group by attributes increases.  
% The best latency is obtained for 2 group by attributs in SYN3-100 and 4 attributes in SYN3-10.  
% This suggests that it is the number of distinct groups that matters most, since these 
% minima occur at 10,000 distinct groups in both cases (shown on \ref{fig:multi_gb_same} as the 
% two ``10000'' labels on the ROW lines).
% Beyond 5 attributes, the performance degrades drastically.  
%  For the column-store, we see a
% relatively small improvement in latency for 2 groups on SYN3-10, with 1 group being best for
% SYN3-100.  Here again it appears that the total number of groups (100 in the case of the column store) determines
% the overall optimal performance.
% After $10^5$ groups, the performance also becomes much worse for COL.

% In Section~\ref{sec:dbms_optimizations}, we described
%  how the impact of this optimization was not
% clear since it increases the total number of groups significantly and therefore
% leads to higher costs of processing intermediate results.
% To evaluate this optimization, we use the SYN3-10 and SYN3-100 datasets.
% We chose these datasets over SYN1 and SYN2 since we wanted to control
% the number of distinct groups in every attribute and consequently the number of
% distinct groups in every combination of attributes.
% In SYN3-10 for example, all dimensions have 10 distinct values and each
% dimension is independently generated. 
% Therefore, the total number of distinct
% groups produced by a query with $p$ group-by attributes is $\max(10^p,
% num\_rows)$.
% SYN3-100 similarly has 100 distinct values per attribute and produces
% $\max(10^p, num\_rows)$ groups for $p$ attributes.
% Our goal in these experiments is to determine if (a) combining multiple
% group-by attributes improved performance, and (b) whether it was the number of
% group-by attributes or the number of distinct groups produced by a query that
% predicted performance.

% We ran \SeeDB\ on SYN3-10 and SYN3-100, and varied the number of
% group-by attributes in view queries ($n_{gb}$) between 1 and 10.


% These results show that 1) combining group by attributes can improve performance, up to a point, and 2) the optimal
% number of attributes is determined by the total number of distinct groups that will result.  Different systems have a different
% number of optimal groups, and this number is somewhat implementation dependent.  The performance degradation with
% large numbers groups likely result from cache misses or other (non)-locality effects
% as the memory required for the grouping grows, or as the system switches to external algorithms for these
% operations.

% \begin{figure}[h]
% \centering
% \begin{subfigure}{0.49\linewidth}
% \centering
% {\includegraphics[width=4.2cm] {Images/multi_gb_same.pdf}}
% \caption{Latency vs. Num of Groups}
% \label{fig:multi_gb_same}
% \end{subfigure}
% \begin{subfigure}{0.49\linewidth}
% \centering
% {\includegraphics[width=4.2cm] {Images/multi_gb.pdf}}
% \caption{Latency vs. Num Dimensions}
% \label{fig:multi_gb_bp}
% \end{subfigure}
% \label{fig:multi_gb}
% \caption{Effect of combining multiple groups}
% \end{figure}

% We can guard against the above performance degradation by ensuring that the
% number of distinct groups never goes beyond a pre-configured upper limit.
% From Figure \ref{fig:multi_gb_same}, we see that this limit is approximately
% $10^4$ for the row-store and $10^2$ for the column store.

To evaluate the gains offered by our bin-packing optimization, we evaluate two methods to
perform grouping, MAX\_GB and BP.
MAX\_GB simply sets a limit on the number of group-bys in each query ($n_{gb}$) whereas BP applies
our bin-packing strategy using the respective memory budgets.
Figure \ref{fig:multi_gb_bp} shows a comparison of the two methods on the SYN dataset.
To evaluate MAX\_GB, we varied $n_{gb}$ was varied between 1 and 20 (solid lines). 
Since SYN contains attributes with between 1 -- 1000 distinct values, memory utilization for a given
$n_{gb}$ can be variable.
For example, $n_{gb}$ = 3 can have anywhere between 1 and $10^9$ distinct groupings, thus breaking the
memory budget for some groups. 
Because groupings with MAX\_GB depends on order, results in Figure \ref{fig:multi_gb_bp} are averages
over 20 runs.
Dotted lines show the latency obtained by BP, our bin-packing scheme.
Unlike MAX\_GB, BP consistently keeps memory utilization under the memory budget.
Consequently, we observe that BP improves performance for both ROW and COL. 
We observe a significant, 2.5X improvement in ROW because the large memory budget, $\mathcal{S}_{ROW}$,
allows many queries to be combined.
COL, in contrast, shows a much less pronounced speedup since its smaller memory budget ($\mathcal{S}_{COL}$=100) 
biases optimal grouping to contain single attribute groups.

% Next, we use the space constraint derived in the previous experiment
% to perform optimal grouping via bin-packing (Section \ref{sec:sharing_opt}).
% With knowledge of this upper limit on the number of distinct groups, we can now
% apply our grouping technique based on bin packing (Section \ref{subsec:mult_gb}) to optimally
% group the dimension attributes.
% Bin-packing ensures that the number of distinct groups produced by any query is
% less than $10^4$ (for rows) or $10^2$ (for columns).
% Figure \ref{fig:multi_gb_bp} shows the results of multi-attribute group-bys on SYN where
% $n_{gb}$ was varied between 1 and 20 (solid lines).
% Note that SYN contains attributes with variable number of distinct values (between 1 and 1000). 
% Therefore, a query with
% $n_{gb}$=3 can have memory utilization anywhere from 1 unit to $10^9$ units, thus breaking the
% memory threshold.
% Since the latency results in this experiment are greatly impacted by how attributes are grouped,
% we randomize the grouping 20 times and report average latency.

% The dotted lines show the latency obtained by performing optimal grouping via bin-packing.
% In both cases, bin packing offers a significant performance
% advantage than naively grouping queries into batches with a certain number of groups-by attributes---this is because naive batching can often demand excessive resources.
% We observe a significant, 2.5X improvement in ROW latency due to bin-packing.
% This can be attributed to the fact that ROW's large threshold on space (10000) allows many queries
% to be combined.
% COL, in constrast, shows a much less pronounced speedup. This is expected since its memory
% threshold of 100 biases the optimal grouping to contain single attribute groupings.
% To obtain these metrics, we randomly grouped attributes into groups of size $n_{gb}$ and ran
% the experiment 20 times to get average latency.
% We also show the performance of \SeeDB\ when we use bin-packing to optimally group the dimension
% attributes.
% As seen in the chart, grouping based on bin-packing is always superior to grouping based
% on the number of attributes $n_{gb}$. 
% In fact, for the row-store, bin-packing reduces latency by a factor of 2.5X. 
% Although benefit is less pronounced, it is also noticeable for column-stores.


% of bin packing when the number of groups is set to $10^2$ (columns) or $10^4$ (rows) (dotted lines).
% In the former strategy, we randomly group attributes into groups of size
% $n_{gb}$. 
% Since the latency for this strategy will depend on the particular grouping of
% attributes, we repeated this experiment 20 times and report the average latency.
% As we can see from the chart, bin-packing is always superior to grouping
% based on attribute number, given the optimal bin-size for a particular system.
% We see that for the row-store, bin-packing reduces latency by a factor of 2.5X. 
% Although benefit is less pronounced, it is still noticeable for column-stores.
 
%For other systems, we recommend users that they set the number of parallel
%queries to the maximum number of parallel queries that can be run in
%their DBMS without performance degradation.
%If this number is not easily available, a simple experiment as shown in Figure
%\ref{fig:parallelism} can help approximate the right amount of parallelism. \\

% \noindent {\it Combining Target and Comparison Views}:
% The last optimization we evaluate is that of combining the target and comparison
% views and running a single SQL query per view as opposed to two.
% We expected this optimization to roughly halve the latency since each query
% takes one table scan instead of two.\\

\begin{figure}[h]
	\centering
	\vspace*{-10pt}
	\begin{subfigure}{0.48\linewidth}
		\centering
		\includegraphics[width=4.4cm] {Images/row_all_none_by_size.pdf}
		\vspace{-15pt}
		\caption{Row store latencies by size}
		\label{fig:row_all_none_size}
	\end{subfigure}
	% \begin{subfigure}{0.24\linewidth}
	% 	\centering
	% 	\includegraphics[width=4.6cm] {Images/row_all_none_by_views.pdf}
	% 	\caption{Row store latencies by views}
	% 	\label{fig:row_all_none_views}
	% \end{subfigure}
	% \begin{subfigure}{0.24\linewidth}
	% 	\centering
	% 	\includegraphics[width=4.6cm] {Images/col_all_none_by_size.pdf}
	% 	\caption{Column store latencies by size}
	% 	\label{fig:col_all_none_size}
	% \end{subfigure}
	\begin{subfigure}{0.48\linewidth}
		\centering
		\includegraphics[width=4.4cm] {Images/col_all_none_by_views.pdf}\
		\vspace{-15pt}
		\caption{Column store latencies by views}
		\label{fig:col_all_none_views}
	\end{subfigure}
	\vspace{-10pt}
	\caption{Effect of All Optimizations}
	\label{fig:all_opt}
	\vspace{-15pt}
\end{figure}


\stitle{All Sharing Optimizations:} 
{\em \underline{Summary:} Applying all of our sharing optimizations
leads to a speedup of up to 20X for row stores, and 8X for column stores;
column stores are still faster than row stores.}
Based on the optimal parameters identified from the previous experiments, we combined 
our optimizations to obtain the best performance. 
% Now that we have explored the proposed optimizations in detail, we pick the optimal
% parameters discovered above and combine our optimizations to get the maximum
% performance gain. 
For ROW, we set
%applied all the above optimizations with 
$n_{agg}$ equal to the total number of measure attributes in the table, 
memory threshold $\mathcal{S}_{ROW}$=$10^4$ and parallelism as $16$.
For COL, we set $n_{agg}$ and number of parallel queries similarly
but did not apply the group-by optimization because of low performance gains. 
Figure~\ref{fig:all_opt} shows the latency of \SeeDB\ on SYN when all
optimizations have been applied.
(Results on real datasets are presented in Section
\ref{sec:expt_summary}.)
We see that our sharing optimizations lead to a speed up of 20X for ROW 
(Figures~\ref{fig:all_opt}a) and a speedup of 10X in COL (Figures~\ref{fig:all_opt}b). 
Our optimizations are most effective for datasets with 
large sizes and many views, particularly for ROW where reduction in table scans has 
large benefits.
%In summary, sharing allows small and moderate sized datasets to be processed at
%interactive time scales (few seconds) and reduces latency on large datasets from 
%500 seconds (ROW) and 100 seconds (COL) to $<$10s for COL, and $<$20s for ROW.
% , with proportionally larger speedups as these numbers grow.

% \techreport{
% \stitle{Resource Utilization:}
% Finally, we examine the resource utilization for the DBMS-backed engine.
% With the DBMS-backed execution engine, each \SeeDB query translates into 50+ optimized SQL queries.
% We call this 50X increase in queries as {\it query bloat}. 
% Each SQL query scans the same data and stores individual state such as iterators, counters, buffers etc.
% In addition, we find that although parallel queries share buffer pool pages, running a large number of view queries in
% parallel can lead to thrashing due to timing issues.
% As a result, we find the DBMS-backed engine has large overheads of memory, CPU and query state due to the query bloat.}

% In summary, the above set of experiments shows that the application of well-designed optimizations
% and parallelism can reduce \SeeDB\ latency by 8-20X depending on the DBMS.
% We notice, however, that in spite of aggressive optimizations, the latencies with sharing optimizations are between 10-20s.
% Such latencies are unacceptable in any interactive system.
% Moreover, we observe that query bloat associated with each \SeeDB invocation unnecessarily consumes DBMS resources.
% In an environment where the DBMS serves multiple users with more than just visualization
% software, this approach is clearly wasteful. 
% As a result, we see need for a solution that can provide much lower latencies 
% and have a smaller resource footprint.
% In the next section, we evaluate the performance of our custom execution engine that promises these advantages.
%!TEX root=document.tex

\subsection{Pruning Optimizations}
\label{sec:custom_execution_engine_expts}
In Section \ref{sec:in_memory_execution_engine}, we developed a set of strategies that enable \SeeDB to eliminate aggregate views with low utility early on and identify top-$k$ aggregate views rapidly.
Through the next set of experiments, we evaluate 
the impact of these optimizations.
% Specifically, our evaluation studies the impact of
% (a) our pruning strategies, 
% (b) utility distributions for the dataset, and (c) parameters of our heuristics
% on latency and accuracy.
% Specifically, in addition to latency (i.e. time required for \SeeDB to produce recommendations),
% we evaluate whether the views chosen by \SeeDB actually have the highest utilities.

\stitle{Metrics:}
We evaluate performance of pruning optimizations along two dimensions, {\em latency}, as before, and
quality of results.
We measure the quality of results with two metrics: 
(1) {\em accuracy}: if $\{\mathcal{V}_T\}$
is the set of aggregate views with the highest utility and $\{\mathcal{V}_S\}$ is the set of 
aggregate views returned by
\SeeDB, the accuracy of \SeeDB is defined as $\frac{1}{\{|\mathcal{V}_T\}|} \times 
|\{\mathcal{V}_T\} \cap \{\mathcal{V}_S\}|$, i.e. the
fraction of true positives in the aggregate views returned by \SeeDB. 
(2) {\em utility distance}: since multiple
aggregate views can have similar utility values, we use utility distance as a measure of how {\it far} \SeeDB 
results are from the true top-$k$ aggregate views. 
We define utility distance as the difference between the average utility of $\{\mathcal{V}_T\}$ 
and the average utility of $\{\mathcal{V}_S\}$, i.e., $\frac{1}{n}(\sum_{i}U(\mathcal{V}_{T,i}) - 
\sum_{i}U(\mathcal{V}_{S,i}))$.

\noindent{\underline{Accuracy vs. Utility Distance.}} 
Since our pruning optimizations rely on utility estimates, the accuracy of pruning
depends on the differences between utilities of consecutive views.
Specifically, if $V_1 \ldots V_n$ is the list of aggregate views ordered by decreasing 
utility, then the accuracy of pruning is inversely proportional to 
the difference between the $k$-th highest utility and the $k+1$-st utility, 
i.e., $\Delta_k = U(V_k) - U(V_{k+1})$.
Views with large $\Delta_k$ values can be pruned accurately while whose with
small $\Delta_k$s can lead to lower absolute accuracy.
While this is true, notice that small $\Delta_k$s are, in fact, the result of views at 
the top-$k$ boundary having similar utility (and interesting-ness).
For instance, the utilities at the top-$5$ boundary for the DIAB dataset are 
$U(V_5)$ = 0.257, $U(V_6)$ = 0.254, and $U(V_7)$ = 0.252.
The small $\Delta_k$s lead to lower accuracy for $k$ = 5, but the 
very similar utility values indicate that $V_6$ and $V_7$ are (almost) equally interesting.
Therefore, even if $V_6$ or $V_7$ are incorrectly chosen to be in the top-$k$, 
the quality of results is essentially as high as when $V_5$ would have been chosen.
Our {\em utility distance} metric correctly captures this overall quality of results.
Utility distance indicates that, in the worst case, even when both $V_6$ or $V_7$ are 
incorrectly chosen, the overall utility of the top-$5$ differs only by 0.013 (~5\% error) 
units compared to the true top-$5$.
As a result, we jointly consider accuracy as well as utility distance when evaluating 
result quality.

% Therefore, when small $\Delta_k$ leads to lower accuracy, the same $\Delta_k$ guarantees that 
% the underlying small difference between utilities ensures that the views returned are {\it 
% just as interesting} as the true top-$k$.
% We refer to this as the {\it paradox of pruning}.

% Views with utilities that are spread apart can be pruned accurately while
% those with similar utilities can lead to pruning inaccuracies.
% The difference between utility values is particularly relevant at the top-$k$
% boundary.
 % to prune aggregate views, 
 % the accuracy of pruning depends on how accurately
 % our estimates can capture the relative order of utilities.
 % Aggregate views with utilities that are spread apart can be pruned accurately while
 % those with similar utilities can lead to pruning inaccuracies.
 % The difference between utility values is particularly relevant at the top-$k$
 % boundary.
 % Reusing notation from the MAB technique, if $V_1 \ldots V_n$ is the list of aggregate views 
 % ordered by decreasing utility, then the accuracy of pruning is inversely proportional to 
 % the difference between the $k$-th highest utility and the $k+1$-st utility, 
 % i.e., $\Delta_k = U(V_k) - U(V_{k+1})$.
 % If $\Delta_k$ is very small, the accuracy of our techniques is lower; however, the small 
 % $\Delta_k$ also implies that views around $k$ have almost equal utility.
 % For instance ...

\stitle{Techniques:}
In the following experiments, we evaluate four techniques for pruning low-utility views.
In addition to the two pruning strategies from Section~\ref{sec:pruning_opt}, 
namely the Hoeffding Confidence Intervals (CI) and the Multi-Armed Bandit (MAB),
we implement two baseline strategies.
First, the no pruning strategy processes the entire data and does not discard any views (NO\_PRU). 
It thus provides an upperbound on latency and accuracy, and lower bound on utility distance.
The other baseline strategy we evaluate is the random strategy (RANDOM) that returns a random 
set of $k$ aggregate views as the result.
This strategy gives a lowerbound on accuracy and upperbound on utility distance: for any 
technique to be useful, it must do significantly better than RANDOM.
Since absolute latencies of any pruning strategy depend closely on the exact DBMS execution techniques, 
in this section, we report relative improvements in latency, specifically, the percent improvement 
in latency with pruning compared to latency without pruning.
Absolute latency numbers for real datasets are discussed in Section \ref{sec:expt_summary}.

%we present latencies in this section 

\stitle{Datasets:}
Because pruning quality depends closely on the underlying data distribution, we evaluate
our pruning optimizations on the real-world datasets from Table \ref{tab:datasets}. 
In this section, we analyze the results for BANK and DIAB in detail; results for AIR and AIR10 are 
discussed in Section \ref{sec:expt_summary}. 
% \techreport{
% Specifically, we make use of the BANK and DIAB datasets listed in Table
% \ref{tab:datasets}. 
% Both datasets
% contain a mix of numerical and categorical attributes. 
% The diabetes dataset~\cite{diab} contains records of hospital visits by diabetes patients. Records include demographics,
% diagnoses, number of hospital days, and procedures performed. 
% The bank dataset~\cite{bank} contains records of customers who applied for a loan, including demographic information about 
% the customers, information about the bank's previous contact with the customer, and the ultimate loan decision.
% }





% For the custom execution engine, on the other hand, we are concerned with both, {\em latency} 
% as well as {\em accuracy}, i.e., whether \SeeDB actually returns the top-$k$ views or not.

% \techreport {
% \stitle{Result Highlights:}
% \begin{denselist}
% \item Our pruning strategies
% reduce \SeeDB latency from 20 seconds (without pruning in custom engine) to less than 2 seconds
% to return the first view, representing a {\em 10X reduction in latency}.

% \item Our latency improvements do not impact accuracy significantly;
% the accuracy of our results is $> 80\%$ with an almost zero utility distance.
% % the utility diof the returned views are close to the utilities
% % of the actual top-$k$ views.

% \item The distribution of (true) view utilities impacts accuracy of pruning strategies.
% Particularly, accuracy is inversely proportional to $\Delta_k$ the difference in
% utility between the $k$-th highest utility and the two neighboring utilities. 

% % \item Our technique of making a single pass through the data along with pruning reduces the overall
% % resource utilization of \SeeDB compared to the DBMS engine \mpv{quantify}.

% % \item Our  indicates that pruning significantly improves \SeeDB performance.
% % In addition, our requirement of only a single pass through the data reduces resource utilization.
% % Most importantly, 

% \item In all, the latency reduction of 10X (for top few views) -- 2X (overall) along with low
% utility distance enables \SeeDB to return high-quality recommendations {\bf at interactive time scales}.
% This makes the custom execution engine a better alternative to a DBMS-backed engine
% in terms of latency, accuracy as well as resource utilitzation (more in Section \ref{sec:comp_of_engines}).
% \end{denselist}
% }

% In addition to {\em latency}, we measure {\em accuracy},
% the number of true top-$k$ views present among the top-$k$ returned by the algorithm.
% In some cases, we will also measure {\em utility distance}, i.e., the 
% difference between the mean utility of the returned top-$k$ views
% and actual top-$k$ views.
% Unlike accuracy, which is 0-1, {\em utility distance}
% allows us to assess the benefit of strategies that return views with very high 
% (but not top) utilities.


% that have gone into We believe that comparing such
% systems offers little value:
% on one hand, our custom implementation lacks features such as logging or
% concurrency control that can slow down more complete systems; on the other hand,
% it also lacks optimizations for exploiting multiple cores, compression,
% vectorization, and other optimizations that scan-optimized column-stores employ.
% Rather, our goal is to highlight the relative performance benefits that can be
% obtained by performing pruning and shared scans.

\begin{figure}[h]
	\centering
	\begin{subfigure}{1\linewidth}
		\centering
		\includegraphics[width=6.5cm]
		{Images/bank_utility_distribution.pdf}
		\vspace{-5pt}
		\caption{Bank dataset: utility distribution}
		\label{fig:bank_utility_distribution}
	\end{subfigure}
	
	\begin{subfigure}{1\linewidth}
		\centering
		\includegraphics[width=6.5cm]
		{Images/diabetes_utility_distribution.pdf}
		\vspace{-5pt}
		\caption{Diabetes dataset: utility distribution}
		\label{fig:diabetes_utility_distribution}
	\end{subfigure}

\vspace{-10pt}
\label{fig:utility_distribution}
\caption{Distribution of Utilities}
\vspace{-15pt}
\end{figure}

In each of our experiments, we vary $k$ --- the number of visualizations to recommend --- between
1 and 25 (a realistic upper limit on the number of aggregates views displayed on a screen)
and measure the latency, accuracy, and utility distance for each of our
strategies. 
We pay special attentio to $k$ = 5 and 10 because empirically these $k$ values are used most commonly.
Since the accuracy and utility distance of our techniques are influenced by the
ordering of data, we repeat each experiment 20
times and randomize data between runs. We report average
metrics over 20 runs.
% We note upfront that our goal is not to directly compare the latency of our custom
% execution engine to that of the DBMS-backed execution
% engine; comparisons between a commercial DBMS-backed system and a proof-of-concept
% system are futile.
% Instead, our goal is to evaluate the performance improvements that can be
% obtained by pruning \techreport{and sharing table scans }in our proof-of-concept 
% implementation.\footnote{We note however that for our experimental datasets, 
% the proof-of-concept prototype provides performance comparable to the DBMS-backed
% system.} We start with an evaluation of the quality of results produced by our custom execution
% engine.

% For example, the DBMS-backed engines (especially the column store) benefits from 
% many man-years of optimizations, including optimizations for scan-intensive workloads, 
% vectorization, compression, the ability to exploit multiple cores and so on.  







% \begin{compactenum}[(a)]
%  \item Hoeffding Confidence Intervals (CI): we use Hoeffding-Serfling
%  confidence intervals with overall $\delta = 0.05$; 
%  % \item 95\% Confidence Intervals (95\_CI): this pruning strategy uses normal 95\% confidence intervals; and 
% \item Multi-armed Bandit (MAB): this pruning strategy uses the multi-armed bandit algorithm.
% \item No Pruning (NO\_PRU): this strategy returns the top-$k$ views with highest utility,
% with no intermediate pruning (to study the impact of pruning on latency);
% \item Random (RANDOM): this strategy returns randomly selected $k$ views (to study the impact of pruning on accuracy).
% \end{compactenum}

\begin{figure*}[t]
	\centering
	\begin{subfigure}{0.33\linewidth}
		\centering
		{\includegraphics[width=6cm] {Images/in_memory_bank_accuracy.pdf}}
		\caption{Accuracy}
		\label{fig:bank_accuracy}
	\end{subfigure}
	\begin{subfigure}{0.33\linewidth}
		\centering
		{\includegraphics[width=6cm] {Images/in_memory_bank_utility_dist.pdf}}
		\caption{Utility Distance}
		\label{fig:bank_utility_dist}
	\end{subfigure}
	\begin{subfigure}{0.33\linewidth}
		\centering
		{\includegraphics[width=6cm] {Images/in_memory_bank_latency.pdf}}
		\caption{Latency}
		\label{fig:bank_latency}
	\end{subfigure}
	\vspace{-10pt}
	\caption{Performance of strategies for Bank dataset}
	\label{fig:bank_perf}
	\vspace{-10pt}
\end{figure*}

\begin{figure*}[t]
	\centering
	\begin{subfigure}{0.33\linewidth}
		\centering
		{\includegraphics[width=6cm] {Images/in_memory_dia_accuracy.pdf}}
		\caption{Accuracy}
		\label{fig:dia_accuracy}
	\end{subfigure}
	\begin{subfigure}{0.33\linewidth}
		\centering
		{\includegraphics[width=6cm] {Images/in_memory_dia_utility_dist.pdf}}
		\caption{Utility Distance}
		\label{fig:dia_utility_dist}
	\end{subfigure}
	\begin{subfigure}{0.33\linewidth}
		\centering
		{\includegraphics[width=6cm] {Images/in_memory_dia_latency.pdf}}
		\caption{Latency}
		\label{fig:diabetes_latency}
	\end{subfigure}
	\vspace{-10pt}
	\caption{Performance of strategies for Diabetes dataset}
	\label{fig:diabetes_perf}
	\vspace{-10pt}
\end{figure*}

 % \stitle{Utility Distributions and Quality of Results}:
 
 
 % Specifically, if a large number of views around the (true) top-$k$ boundary have similar utilities,
 % then our pruning strategies may choose views incorrectly at this boundary.
 % This results in lower accuracy.
 % On the other hand, since these views have utility similar to the true top-$k$ views, the
 % resulting utility distance is almost zero.
 %since utilities that are close together have very similar running
%estimates of utility and hence are difficult to tease apart and prune.
 % For each of the test datasets, we first review the utility distributions and then
 % analyze the performance of our pruning strategies given the utility distribution.

\stitle{Accuracy and Utility Distance:}
{\em \underline{Summary:} The MAB and CI strategy both produce results with 
accuracy $>$75\% and near-zero utility distance for a variety of $k$ values.
MAB does slightly better than CI when utlity values are closely spaced.
In general, smaller $\Delta_k$ values lead to lower accuracy, but this is offset by
lower utility distance that is a consequence of the smaller $\Delta_k$s. 
}

%  \em \underline{Summary:} The MAB strategy dominates the CI
% strategy when it comes to both accuracy and utility distance,
% with accuracy $>$75\%  for $k = 1$ and even larger for larger
% values of $k$, and a near-zero utility distance. 
% Smaller $\Delta_k$ values lead to lower accuracy, but this is offset by
% lower utility distance.
%  }

\noindent {\it \underline {BANK dataset}}:
The distribution of utilities for all aggregate views of the bank dataset is
shown in Figure~\ref{fig:bank_utility_distribution}. 
In this chart, vertical lines denote the cutoffs for utilities of the top-$k$ views
where $k$=\{1,\ldots,10,15,20,25\}.
The highest utility for this dataset corresponds to the {\it right-most} line
in this chart while the 25-th highest utility corresponds to the {\it left-most}
line. 
We observe that the highest and second highest utility are spread well apart 
from the rest ($\Delta_k$=0.0125). 
The top 3rd--9th utilities are similar ($\Delta_k$<0.002) while the 10th highest 
utility is well separated from neighboring utilities ($\Delta_{10}$=0.0125).
The remaining aggregate views once again have similar utilities ($\Delta_k$<0.001).
We see the effect of utility distribution in the performance of our pruning 
strategies.
Figure~\ref{fig:bank_accuracy} and Figure~\ref{fig:bank_utility_dist} respectively show
the {\em average} accuracy and utility distance of our strategies over 20 runs.
We find that MAB consistently produces 75\% or better accuracy for all values of $k$ and
CI produces 85\% or better accuracy for $k$$>$10.
For $k$=1 and 2, the accuracy is 75\% for both pruning strategies (due to large 
$\Delta_k$ values).
The corresponding utility distance is almost zero for MAB and about 0.015 for CI (note that these
are averages).
Between $k$=3\ldots9, the accuracy for all strategies suffers due to small $\Delta_k$s ($< 0.002$).
In spite of lower accuracies, note that utility distance is consistently small ($<$ 0.02).
After $k$=10, the performance of all our strategies improves once again and tends to 100\% accuracy and
0 utility distance.
We note that NO\_PRU necessarily has perfect performance, while RANDOM has extremely poor accuracy (<0.25) 
and utility distance ($>$5X that of CI and MAB). 

% As with accuracy, the utility distance tends to zero with large $k$s.
% All our strategies produce views with 0 or almost 0 utility distance for most $k$. 
% Thus, even if \SeeDB picks a few incorrect views, there is effectively no difference in the 
% utilities of these views and the true top-$k$ views.
% So even when a top-$k$ strategy picks a few incorrect views, the selected views
% have utility very close to the real top-$k$ views, i.e., are views are
% of high quality.
% This implies that even if our top-$k$ views are
% approximate, they are of high quality.
% Another way to analyze mistakes in the top-$k$ views is by examining if the an
% incorrectly returned view for the top-$k$ views also appears in the top-$2k$,
% top-$3k$ or top-$4k$.
% Figure \ref{} shows the results for the banking dataset.
% We see that XXX,



% \begin{figure}[h]
% \centering
% \begin{subfigure}{0.49\linewidth}
% \centering
% {\includegraphics[width=4.2cm] {Images/bank_in_memory_accuracy.pdf}}
% \caption{Accuracy of strategy for bank dataset}
% \label{fig:bank_accuracy}
% \end{subfigure}
% \begin{subfigure}{0.49\linewidth}
% \centering
% {\includegraphics[width=4.2cm] {Images/dia_in_memory_accuracy.pdf}}
% \caption{Accuracy of strategies for diabetes dataset}
% \label{fig:dia_accuracy}
% \end{subfigure}
% \label{fig:accuracy}
% \caption{Accuracy of strategies}
% \end{figure}


% \begin{figure}[h]
% \centering
% \begin{subfigure}{0.49\linewidth}
% \centering
% {\includegraphics[width=4.2cm] {Images/bank_in_memory_utility_dist.pdf}}
% \caption{Utility Distance of strategy for bank dataset}
% \label{fig:bank_utility_dist}
% \end{subfigure}
% \begin{subfigure}{0.49\linewidth}
% \centering
% {\includegraphics[width=4.2cm] {Images/dia_in_memory_utility_dist.pdf}}
% \caption{Utility Distance of strategies for diabetes dataset}
% \label{fig:dia_utility_dist}
% \end{subfigure}
% \label{fig:accuracy}
% \caption{Utility Distance for strategies}
% \end{figure}

% \begin{figure}[h]
% \centering
% {\includegraphics[trim = 0mm 50mm 0mm 50mm, clip, width=6cm]
% {Images/bank_utility_distribution.pdf}}
% \caption{Bank dataset: utility distribution}
% \label{fig:bank_utility_distribution}
% \end{figure}
% \begin{figure}[h]
% \centering
% {\includegraphics[trim = 0mm 50mm 0mm 50mm, clip, width=6cm]
% {Images/diabetes_utility_distribution.pdf}}
% \caption{Diabetes dataset: utility distribution}
% \label{fig:diabetes_utility_distribution}
% \end{figure}
 
\noindent {\it \underline{DIAB dataset:}} 
Next, we briefly review results for the diabetes dataset.
The distribution of true utilities for all aggregate views in this dataset are shown in
Figure~\ref{fig:diabetes_utility_distribution}.
We observe that utilities for the top 10 aggregate views are very closely clustered ($\Delta_k<$0.002) while
they are sparse for larger $k$s.
Therefore, we expect lower pruning accuracy for $k<10$ but high accuracy for
large $k$'s.
We see this behavior in Figure~\ref{fig:dia_accuracy} where the accuracy of
pruning is quite low ($<60\%$) for $k$=1 but improves consistently to 68\% (CI) 
and 86\% (MAB) for $k$=5 and is $>$80\% for $k$$\geq$10.
In the companion figure, Figure \ref{fig:dia_utility_dist}, we see that although
accuracy is relatively low $k$$<$5, utility distance is small (0.013 for CI, 0.002
for MAB) indicating that the results are high quality.
Both CI and MAB produce 40X smaller utility distances compared to RANDOM.

% We also observe an important property of our strategies: the accuracy of both
% of our pruning strategies, MAB and CI, is comparable; MAB appears to
% perform better for small number of $k$s but all three produce similar results
% for $k>10$. (NO\_PRU is guaranteed to have perfect accuracy).
% This suggests that since all strategies perform similarly on accuracy, we can
% choose the strategy with the minimum latency.\\

% 95\_CI does the best among all our strategies for the whole range of $k$ values.
% MAB and HOEFF produce similar accuracy values with MAB being slightly better
% than HOEFF.
% There are a few reasons why 95\_CI performs better: the MAB strategy is tied to
% either accepting or discarding a view at the end of each phase; therefore, even
% if MAB is not very confidence in the action of accepting or discarding, it must
% reduce one view in each phase. HOEFF on the other hand is less accurate because
% XXX.
% All our strategies however seem to have low accuracy for $k<10$. 

\stitle{Latency:}
{\em \underline{Summary:} Both pruning strategies provide a reduction in latency of 50\% or more
relative to NO\_PRU. For smaller $k$, reductions can be even higher, closer to 90\%; this can be
especially useful when we want to identify and quickly display the first one or two top views.}
Figures~\ref{fig:bank_latency} and \ref{fig:diabetes_latency} show the latency
of our strategies for the banking and diabetes dataset.
First off, we observe that the use of either of CI or MAB produces a 50\% reduction in latency
throughout.
In fact, for CI, we obtain almost a 90\% reduction in latency for small $k$. 
For $k$=5, MAB produces betwen 50 - 60\% reducation while CI produces a reduction of 60 - 80\%.
Early stopping, i.e. returning approximate results once the top-$k$ views have been identified, 
can produce even better latency reduction (results in Section \ref{??}).
As expected, as $k$ increases, latency also increases because we can prune fewer aggregate views.
% These reductions bring latency down from multiple tens of seconds to {\bf single digit latencies}, i.e.,
% \SeeDB can operate on interactive time scales.
% Since these latency numbers come from our prototype implementation, a well-tuned system could
% produce results in few seconds, i.e., {\it in interactive time scales}. 
% Clearly, we give up some accuracy when we obtain this reduction in latency, however, as demonstrated
% experimentally, our strategies consistently provide high utility views with low utility distance.op-$k$.
% This latency-accuracy tradeoff is particularly important in a real-time system where we want to quickly 
% provide recommendations for the analyst to browse.

\stitle{CI vs. MAB}.
In our evaluation, we compared two competing pruning strategies, CI and MAB. 
From Figures \ref{fig:bank_perf} and \ref{fig:diabetes_perf}, we observe that MAB, 
on average, has higher accuracy and lower utility distance compared to
CI, i.e., overall, it produces higher quality results.
However, we find that CI performs much better than MAB on latency.
Since CI can prune views more aggressively than MAB (MAB only discards one view at a time),
it can rapidly prune the space of views, but this comes at the cost of result quality.
Depending on the tradeoff between latency and quality of results, we can choose the best
pruning strategy from CI and MAB.

% \techreport{
% \stitle{Resource Utilization:}
% % \mpv{Not sure about this section. Someone needs to review}
% % We contrast the resource utilization of our custom execution engine to the DBMS-backed engine. 
% % Specifically:
% Compared to the resource utilization in the DBMS-backed engine, our custom engine requires much fewer resources.
% Specifically, since the custom engine makes a single pass through the data, the CPU utilization is expected to be lower compared to the repeated scans of the DBMS-backed engine.
% Another consequence of the single scan is that there is
% no thrashing introduced by multiple parallel scan queries.
% Lastly, there is no {\it query bloat} associated with
% the custom engine, saving overheads of per-query state such as iterators, buffers etc.
% }
% means that the custom engine is expected to
% have lower CPU as compared to the multiple scans of data resulting from the DBMS-backed engine (assuming that the
% additional state maintenance is not significant).

% Each \SeeDB query translates to only a single query in the custom execution engine.
% The DBMS-backed engine, in contrast, issues $~$ 50 queries for each \SeeDB query. 
% Consequently, additional resources must be wasted by the DBMS in keeping state for each query such as 
% iterators, buffers etc.

% \techreport{
% \subsection{Comparison of Execution Engines}
% \label{sec:comp_of_engines}

% The previous sections discussed the performance of the DBMS-backed and Custom execution engines of \SeeDB.
% We found that with a set of clever optimizations, we could reduce the latency of the DBMS-backed engine by
% 20X.
% However, the resulting latency of few tens of seconds was too large for interactive visualization.
% In addition, we found that the query bloat led to large memory footprint, repeated scans of data, and
% higher resource utilization.
% In contrast, the custom engine provided a means to perform a single scan of the data and identify top-$k$ views
% with high accuracy. \mpv{numbers?}
% Moreover, the custom engine produced latencies of a few seconds -- {\bf it enabled \SeeDB to respond at interactive
% time scales}.
% These performance results together indicate that the custom execution engine and its pruning strategies are a superior 
% alternative to a DBMS-backed execution engine.
% Since the goal of \SeeDB is to provide recommendations in real time, we can pay a small penalty in accuracy 
% and instead provide almost instantaneous results.

% Clearly, the ideal solution would be to integrate the custom execution engine functionality into a vanilla database.
% This would enable the DBMS to use existing structures to efficiently scan a dataset while maintaining and pruning
% views on the fly. 
% The SQL GROUPING SETS\footnote{} functionality, i.e. multiple independent group-bys in a single query is a first step in this direction.
% However, GROUPING SETS needs to be supplemented with much finer control and scalability to support \SeeDB-like  functionality.}

% In summary, the performance results of this section indicate that due to the lower latency, high accuracy,
% and overall lower resource utilization, the custom engine is a superior alternative to a DBMS-backed
% execution engine.
% It enables \SeeDB to make a single pass through the data, avoids unnecessary {\it query bloat} leading to
% lower resource utilization and lowers latency by upto 10X enabling \SeeDB to work in a real 
% interactive visualization system.

% so all of these strategies are promising alternatives to use 
% in a production system --- especially when we want to quickly identify and provide a few 
% views for the analyst to browse.
% If latency is paramount, then CI may be used,
% and if utility is paramount, then MAB may be used. \mpv {counter-intuitive}
% We observe that the latency of CI increases almost linearly
% with $k$. This trend arises because as $k$
% increases, we throw out fewer views and therefore perform more
% computation per record.
% This exact trend is not seen in MAB because MAB's view pruning is agnostic
% to the number of views that must be selected.

% \subsection{Combining Sharing \& Pruning}
% \label{sec:sharing_and_pruning}
% As discussed in the previous sections, sharing optimizations can provide performance gains upto 10X for COL and 40X for ROW while pruning optimizations can reduce latency by 2X -- 5X.
% In this section, we evaluate the performance gains that can be obtained by combining both types of optimizations. 
% In the ideal case, we expect the gains to be {\it multiplicative}, i.e., we would expect gains between 20X -- 200X.



% Figure \ref{fig:share_prune_col} shows the relative performance gains that can be obtained by applying various combinations of optimizations in the COL store. 
% As seen in figure, 

% The corresponding data for ROW stores was presented in Figure~\ref{fig:share_prune_row} above.
% We find that the combination of optimizations produce gains of upto 50X (COMB) -- 150X (COMB\_EARLY) for ROW, with pruning-based
% optimizations providing a 4X speedup over the sharing optimizations.


% We see similar results for ROW where pruning provides a similar 4X speedup.
% However, we make a few observations. First, pruning doesn't benefit small datasets (e.g. BANK, DIAB). In fact, due to the overhead of multiple phases (and the associated query costs), the combined optimization (COMB) does worse than SHARING alone. Since the SHARING latencies for small datasets are under a few seconds, we do not find the need to perform sharing.
% Second, as noted before, due to different data layout, COL stores have significantly lower latencies for the \SeeDB workload compared to ROW.
% However, the data layout for ROW benefits more from the sharing of scans and pruning since the cost of a table scan is much higher for ROW.
% T
% COL stores in general are a better fit for the \SeeDB workload.

% Compared to the performance of ROW stores in Figure \ref{}, we find that COL stores benefit less from optimizations.
% As before, this is because column stores can selectively load attributes that are relevant for the particular visualization. \mpv{verify}

% % We also find a few differences in the performance gains obtained by pruning in Section \ref{} () and those obtained by the combined optimizations ().

% The reasons for this difference are two-fold: (1) the simple pruning implementation described in Section \ref{} is based on shared scans and does not incur any overhead for each phase of the algorithm; in contrast, when we implement pruning in \SeeDB, each phase involves \SeeDB issuing a large number of queries to the DBMS and thus incuring overheads such as query dispatch, maintaining query state etc. (2) due to the specific implementations of ROW and COL stores, there is threshold on table size below which the time to scan tables is essentially the same.
% As a consequence of these two factors, we find that the combination of pruning and sharing degrades performance for small datasets (e.g. BANK, DIAB) due to phase overhead.
% The combination of optimizations is well suited to moderate and large datasets such as AIR and AIR10 and produces a CCC speedup.



% The next set of experiments vary the parameters for each strategy to study
% the accuracy vs. latency tradeoff.

% \begin{figure}[h]
% \centering
% \begin{subfigure}{0.49\linewidth}
% \centering
% {\includegraphics[width=4.2cm] {Images/bank_in_memory_latency.pdf}}
% \caption{Bank dataset: latency}
% \label{fig:bank_latency}
% \end{subfigure}
% \begin{subfigure}{0.49\linewidth}
% \centering
% {\includegraphics[width=4.2cm] {Images/dia_in_memory_latency.pdf}}
% \caption{Diabetes dataset: latency}
% \label{fig:diabetes_latency}
% \end{subfigure}
% \label{fig:accuracy}
% \caption{Latency for strategies}
% \end{figure}

% \stitle{Accuracy vs.~Latency:}
% {\em \underline{Summary:} Tuning the knobs in the pruning strategies
% gives us further reduction in latency for some losses in accuracy.}
% All of our strategies have ``knobs'' we can use to study the
% trade-off between accuracy and latency: for MAB, this corresponds to the
% number of phases used during processing while for the CI pruning, it 
% corresponds to the size of confidence intervals.
% As expected, if we set these respective parameters to favor greater accuracy 
% (i.e. fewer pruning phases in MAB or larger confidence intervals in CI pruning),
% it also leads to larger latency since fewer views can be pruned at any step.

% Here, we study the impact of these knobs on MAB and 95\_CI, which represent
% two extremes in our set of pruning strategies.
% For MAB, the knob is the number of phases involved in
% processing file; since MAB reduces the number of 
% views by 1 after each phase, the number of
% phases is proportional to the pruning power of our algorithm.
% A large number of phases means that MAB will prune more views and will prune
% them more often.
% Figure \ref{fig:latency_vs_accuracy_mab} shows how latency and accuracy both
% reduce as we increase the number of phases in MAB ($k$=15).
% Each point on the chart corresponds to a different setting for the number of
% phases uses in that implementation of the MAB strategy.
% For 95\_CI, we can vary the $z$-score used
% to determine the size of our confidence intervals.
% That is, we can decide to take a 50\% confidence interval or a 80\% interval or
% a 95\% interval.
% If we take a smaller confidence interval, we will have higher pruning and
% therefore lower latency.
% However, a smaller confidence interval also leads to lower latency since we
% prune views with lower confidence.
% Figure \ref{fig:latency_vs_accuracy_ci} shows that as the $z$-score of the
% confidence interval increases, the accuracy of our strategies increases, but so
% does its latency ($k$=15).
% Every point corresponds to a different size of the confidence intervals.

% \begin{figure}[h] 
% \centering
% \vspace{-10pt}
% \begin{subfigure}{0.49\linewidth}
% \centering
% {\includegraphics[width=4.2cm] {Images/latency_vs_accuracy_ci.pdf}}
% \caption{95\_CI: Values depict CI \%}
% \label{fig:latency_vs_accuracy_ci}
% \end{subfigure}
% \begin{subfigure}{0.49\linewidth}
% \centering
% {\includegraphics[width=4.2cm] {Images/latency_vs_accuracy_mab.pdf}}
% \caption{MAB: Values depict no. of phases}
% \label{fig:latency_vs_accuracy_mab}
% \end{subfigure}
% \label{fig:accuracy}
% \vspace{-10pt}
% \caption{Latency vs. Accuracy for different strategies}
% \vspace{-20pt}
% \end{figure}











\section{conclusion}
\label{sec:related_work}

In the paper, we discussed \SeeDB\ : a system that provides analysts with visualizations highlighting interesting aspects of the query result. We defined several concrete problems in \SeeDB\ and their solutions, ranging from multi-query optimization and approximation to multi-criteria optimization. 











\end{document}