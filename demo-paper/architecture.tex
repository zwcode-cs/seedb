%!TEX root=demo-paper.tex

\section{{\large \VizRecDB\ } Design}
\label{sec:system_architecture}

In this section, we present the \VizRecDB\ architecture, starting with an
overview followed by a detailed discussion of its components.

\subsection{{\large \VizRecDB} architecture overview}
\label{subsec:overview}

Our \VizRecDB\ prototype is designed as a layer on top of a traditional
relational database system.
While optimization opportunities are restricted by virtue of being outside the
database, our design permits \VizRecDB\ to be used in conjunction with a variety of
existing database systems. 
\VizRecDB\ is comprised of two parts: a frontend and a backend. 
The frontend is a ``thin client'' that
is used to issue queries and display visualizations. The backend, in
contrast, performs all the computation required to generate and select views
to be recommended. Figure \ref{fig:sys-arch}
depicts the architecture of our system.

\begin{figure}[htb]
\vspace{-10pt}
\centerline{
\hbox{\resizebox{9cm}{!}{\includegraphics[trim=10mm 50mm 10mm 50mm,
clip=true]{Images/VizRecDB-architecture.pdf}}}}
\caption{VizRecDB Architecture}
\label{fig:sys-arch}
\vspace{-15pt}
\end{figure} 

An analyst uses the frontend to issue queries to \VizRecDB. We provide three
mechanisms for the analyst to issue queries (further discussion in
Section \ref{subsec:VizRecDB_frontend}).
Once the analyst issues a query via the frontend, the backend takes over.
First, the Metadata Collector module queries metadata tables (a combination of
database-provided and \VizRecDB\ specific tables) for information such as table
sizes, column types, data distribution, and table access patterns.
The resulting metadata along with the analyst's query is then passed to the
Query Generator module. The purpose of the Query Generator is two-fold:
first, it uses metadata to prune the space of candidate views to only retain the
most promising ones; and second, it generates target and comparison views for
each view that has not been pruned.
The SQL queries corresponding to the target and comparison views are then passed
to the Optimizer module. We refer to these queries collectively as {\it view
queries}.
Next, the Optimizer module determines the best way to
combine view queries intelligently so that the total execution time is
minimized.
(We discuss optimizations performed by \VizRecDB\ in Section
\ref{subsec:VizRecDB_backend}.) Once the Optimizer module has generated the
optimized queries, \VizRecDB\ runs them on the underlying DBMS.
Results of the optimized queries are processed by the View Processor in a
streaming fashion to produce results for individual views. Individual view
results are then normalized and the utility of each view is computed.
Finally \VizRecDB\ selects the top $k$ views with the highest utility and returns
them to the \VizRecDB\ frontend. The frontend generates 
and displays visualizations for each of these view. We now discuss
\VizRecDB\ modules in detail.

\subsection{The Frontend}
\label{subsec:VizRecDB_frontend}

The \VizRecDB\ frontend, designed as a thin client, performs two main functions: it
allows the analyst to issue a query to \VizRecDB, 
and it visualizes the results (views) produced by the \VizRecDB\
backend.
To provide the analyst maximum flexibility in issuing queries, \VizRecDB\
provides the analyst with three
mechanisms for specifying an input query: 
(a) directly filling in SQL into a text box, 
(b) using a query builder tool that allows analysts
unfamiliar with SQL to formulate queries through a form-based interface, and (c)
using pre-defined query templates which encode commonly performed operations,
e.g., selecting outliers in a particular column. 
%We find that pre-defined query
%templates are particularly useful since analysts are often interested in
%anomalous data points.

Once the analyst issues a query via the \VizRecDB\ frontend, the backend
evaluates various views and delivers the most interesting ones (based on
utility) to the frontend.
For each view delivered by the backend, the frontend creates a visualization
based on parameters such as the data
type (e.g. ordinal, numeric), number of distinct values, and semantics (e.g.
geography vs. time series).
The resulting set of visualizations is displayed to the analyst who can then
easily examine these ``most interesting'' views at a glance, explore specific views in
detail via drill-downs, 
%by hovering and clicking on various portions of the view, 
and study metadata for each view (e.g. size of result, sample data, value with
maximum change and other statistics). 
%The analyst can also slice-and-dice views further by performing drill-downs on
%specific attributes in the view. 
Figure~\ref{fig:frontend1} shows a screenshot of the \VizRecDB\ frontend (showing
the query builder) in action.
 
\begin{figure}[htb]
\vspace{-10pt}
\centerline{
\hbox{\resizebox{5cm}{!}{\includegraphics[trim=15mm 0mm 120mm 0mm,
clip=true]{Images/sql_builder.pdf}}}
\hbox{\resizebox{!}{6cm}{\includegraphics[trim=50mm 30mm 150mm 52mm,
clip=true]{Images/viz_panel.pdf}}}}
\caption{VizRecDB Frontend: Query Builder (left) and Example Visualizations
(right)}
\label{fig:frontend1}
\vspace{-15pt}
\end{figure} 

%This action automatically
%modifies the selection query and displays views for the subset of data
% selected. The user can of course revert back to the original views and continue exploring the data.
\vspace{-5mm}
\subsection{The Backend}
\label{subsec:VizRecDB_backend}

The \VizRecDB\ backend is responsible for all the computations for 
generating and selecting views. 
%\agp{next line can be deleted if needing space, repetitive.}
% As shown in Figure~\ref{fig:sys-arch}, the \VizRecDB\ backend is composed of four
% modules that are respectively responsible for collecting metadata (Metadata Collector), pruning
% the view space and generating view queries (Query Generator), optimizing view
% queries (Optimizer), and processing query results to identify the top-$k$
% interesting views (View Processor). 
To achieve its goal of finding the most
interesting views accurately and efficiently, the \VizRecDB\ backend must not only accurately
estimate the accuracy of a large number of views but also design ways in which
the total processing time will be minimized.
We first describe the basic \VizRecDB\ backend framework and then briefly discuss our optimizations.

% One of the chief challenges in \VizRecDB\ is producing the most interesting views
% of the query result in the least possible time. For achieve the above
% performance goal, \VizRecDB\ must perform optimizations at two stages: first, using
% prior knowledge such as statistics to prune out uninteresting views without examining table data; and second, minimizing the
% execution time for queries that are issued to the database. 

%\subsubsection{Basic Framework}
%\label{subsubsec:basic_framework}
\stitle{Basic Framework:}
Given a user query $Q$, the basic approach computes all
possible two-column views obtained by adding a single-attribute aggregate and group-by clause to $Q$. 
The target and comparison views corresponding to each view are then
computed and each view query is executed independently on the DBMS. The query
results for each view are normalized, and utility is computed as the
distance between these two distributions (Section \ref{sec:problem_statement}).
Finally, the top-$k$ views with the largest utility are chosen to be displayed. 
The basic approach is clearly inefficient
since it examines every possible view 
and executes each view query independently.
We next discuss how our optimizations fix these problems.

%\subsubsection{View Space Pruning}
%\label{subsubsec:view_space_pruning}

\stitle{View Space Pruning:}
In practice, most views for any query $Q$ have low utility since the target view
distribution is very similar to the comparison view distribution. 
\VizRecDB\ uses this property to aggressively prune 
view queries that are unlikely to have high utility. 
This pruning is based on metadata about the table including data
distributions and access patterns. Our techniques include:
\begin{denselist}
\item {\it Variance-based pruning}: Dimension attributes with low variance are
likely to produce views having low utility (e.g. consider the extreme case where
an attribute only takes a single value); \VizRecDB\ therefore prunes views
with grouping attributes with low variance.
\item {\it Correlated attributes}: If two dimension attributes $a_i$ and $a_j$ have
a high degree of correlation (e.g. full name of airport and abbreviated name of
airport), the views generated by grouping the table on $a_i$ and $a_j$ will be
very similar (and have almost equal utility). We can therefore generate and
evaluate a single view representing both $a_i$ and $a_j$. \VizRecDB\ clusters
attributes based on correlation and evaluates a representative view per
cluster.
\item {\it Access frequency-based pruning}: In tables with a large number of
attributes, only a small subset of attributes are relevant to the analyst and
are therefore frequently accessed for data analysis. \VizRecDB\ tracks access patterns
for each table to identify the most frequently accessed columns and combinations of
columns. While creating views, \VizRecDB\ uses this information to prune attributes
that are rarely accessed and are thus likely to be unimportant.
\end{denselist}

%\subsubsection{View Query Optimizations}
%\label{subsubsec:optimizations}
\stitle{View Query Optimizations:}
The second set of optimizations used by \VizRecDB\ minimizes the execution time for
view queries that haven't been pruned using the techniques described above.
Since view queries tend to be very similar in structure (they differ in the aggregation
attribute, grouping attribute or subset of data queried), \VizRecDB\ uses multiple
techniques to intelligently combine view queries.
The ultimate goal is to minimize scans of the underlying dataset by sharing as
many table scans as possible. Our strategies include:

\begin{denselist}
  \item {\it Combine target and comparison view query}: Since the target view
  and comparison views only differ in the subset of data that the query is
  executed on, we can easily rewrite these two view queries as one.
  This simple optimization halves the time required to compute the results for
  a single view.
  \item {\it Combine Multiple Aggregates}: A large number of view
  queries have the same group-by attribute but different aggregation attributes.
  Therefore, \VizRecDB\ combines all view queries with the same group-by attribute
  into a single query. This rewriting provides a speed up linear in the
  number of aggregate attributes.
  \item {\it Combine Multiple Group-bys}: 
  Since \VizRecDB\ computes a large number of group-bys, one significant
  optimization is to combine queries with different
  group-by attributes into a single query with multiple group-bys attributes.
  For instance, instead of executing queries for views $(a_1$, $m_1$, $f_1)$,
  $(a_2$, $m_1$, $f_1)$ \ldots $(a_n$, $m_1$, $f_1)$ independently, 
  we can combine the $n$ views into a single view represented by
  $(\{a_1, a_2\ldots a_n\}$, $m_1$, $f_1)$ and post-process results at the
  backend. Alternatively, if the SQL GROUPING SETS functionality is available in
  the underlying DBMS, \VizRecDB\ can leverage that as well. 
  While this optimization has the potential to significantly reduce query
  execution time, the number of views that can be combined depends
  on the correlation between values of grouping attributes and system parameters like the
  working memory. Given a set of candidate views, we model the
  problem of finding the optimal combinations of views as a variant of bin-packing and apply ILP techniques to obtain the best solution. 
%   (We discuss our model and
%   algorithm in our full paper~\ref{}).\agp{if full paper is not available
%   by the time of the demo, we can't cite it unfortunately..}
%   A variation of this approach also implemented
%   on \VizRecDB\ is to send the results of the multiple group-by query to the front
%   end and ask the \VizRecDB\ frontend to compute utility and select views. The
%   advantage of this approach is that it allows for more efficient interactive
%   exploration of the views.
  \item {\it Sampling}: For datasets of large size, an optimization that
  affects performance significantly is employing sampling: we  
  construct a sample of the dataset
  that can fit in memory and run all view queries against the sample. However, as
  expected, the sampling technique and size of the sample both affect
  view accuracy. 
  \item {\it Parallel Query Execution}: The final optimization that
  \VizRecDB\ employs is taking advantage of parallel query execution 
  at the DBMS to reduce total latency.
  We observe that as the number of queries executed in parallel
  increases, the total latency decreases at the cost of
  increased per query execution time.
\end{denselist}
%\begin{figure}[htb]
%\centerline{
%\hbox{\resizebox{9cm}{!}{\includegraphics[trim=10mm 50mm 10mm 50mm,
%clip=true]{Images/VizRecDB-frontend.pdf}}}}
%\caption{VizRecDB Frontend}
%\label{fig:frontend}
%\end{figure} 
