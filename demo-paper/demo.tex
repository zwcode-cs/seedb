%!TEX root = demo-paper.tex

\section{Demo Walkthrough}
\label{demo-walkthrough}

We propose to demonstrate the following aspects of \SeeDB\ as part our
scenarios: (i) the utility and versatility of \SeeDB, (ii) the latency
of \SeeDB\ operating on various dataset sizes, along with the relative
impact of optimizations within \SeeDB.

\stitle{Demonstrating Utility and Versatility:} We will allow
conference attendees to interact with three diverse datasets,
capturing three potential ways \SeeDB\ may be used. Attendees will be
both (a) able to pose queries, view results, and interact further with
these datasets in real time, and (b) be able to browse a selection of
queries that we have pre-selected to be those that give unexpected or
surprising results. The datasets that we will allow the conference
attendees to browse include the following:

\squishlist
  \item {\bf Store Orders dataset}~\cite{superstore}: This dataset is
    often used by Tableau~\cite{tableau} as a canonical dataset for
    business intelligence applications. It consists of information
    about orders placed in a store including products, prices, ship
    dates, geographical information, profits, and so on. This dataset
    is well-studied by users learning how to use Tableau, with several
    web-pages dedicated to discovering interesting trends hidden in
    this dataset~\cite{website}. Attendees using \SeeDB\ will be able
    to identify very quickly the same insights and trends that Tableau
    users have discovered over many years. This dataset will also
    enable us to demonstrate how \SeeDB can correctly deal with
    numeric, categorical, time series and geographic data.
  \item {\bf Election Contribution dataset}~\cite{}: This dataset is
    an example of a real-world dataset that is typically analyzed by
    non-expert data analysts, such as journalists or historians. This
    dataset will enable us to demonstrate to the attendees how
    non-experts can quickly arrive at interesting visualizations via
    the intuitive user interface.
  \item {\bf Medical dataset~\cite{}:} This dataset is an example of a
    real-world dataset that a researcher (here, a clinical researcher)
    might use over the course of his/her work. This data has a schema
    that is more complex than the the election or store one, and is of
    larger size too.  \squishend

For each dataset, the interaction workflow will be the following: the conference
attendee can either opt for a pre-selected query, or formulate a query of their
own. After submitting the query, \SeeDB\ will search the space of possible views
to return the top-$k$ views (tunable value, default set to $10$.). For the
purposes of the demonstration, we will also allow attendees to view the
bottom-$k$ visualizations, enabling them to see why \SeeDB\ actually displayed
the chosen set of visualizations over other ones.


\stitle{Demonstrating Speed and Optimizations:} This demonstration
scenario will use an enhanced user interface, with three additional
``knobs'' or ``dials'' that attendees can fix, described below. For
this demonstration scenario, we will focus on the Medical dataset,
which is a large dataset with a large schema.

\squishlist
\item Fraction of dataset: The first knob allows attendees to change
  the fraction of the medical dataset selected. Clearly, as the
  dataset size increases, the latency of \SeeDB\ increases, so
  attendees will be able to see how much of an impact the size of the
  dataset has on the latency of \SeeDB.

\item Fraction of schema: The second knob allows attendees to change
  the number of columns in the medical dataset selected. As before, as
  the number of columns increases, the latency of \SeeDB\ increases,
  so attendees will be able to see how much of an impact the number of
  columns has on the latency of \SeeDB.

\item Optimizations selected: The last knob allows attendees to change
  the optimizations (from Section~\ref{sec:optimzations}) that are
  being used within \SeeDB, allowing attendees to see which
  optimizations actually have an impact.  \squishend

\noindent This demonstration scenario, in addition to having a
different user interface, will also have a different result interface,
with additional statistics reported when the visualizations are
generated, including (but not limited to):

\squishlist
\item Time to completion
\item Time to first visualization
\item \agp{more here...}
\squishend



\eat{
We propose to demonstrate the functionality of the SeeDB system by means of
analyzing three diverse datasets of practical importance. Users will be able to
explore each of these datasets in real-time by using SeeDB to formulate
queries and find interesting trends in the underlying dataset. Specifically, we
will use the following datasets for demonstration purposes:

\begin{itemize}
  \item {\bf Store Orders dataset}: This is a canonical dataset used in business
  intelligence applications. It consists of information about orders placed in a
  store including products, prices, ship dates, geographical information,
  profits etc. The dataset is well known for its interesting trends and
  richness of various data types. It will show off SeeDB capabilities to
  correctly identify diverse trends in the data and the ability to deal with
  numeric, categorical, time series and geographic data.
  \item {\bf Election Contribution dataset}: This dataset is a great example of
  the kind of data and analysis that must be done by potential users like
  journalists who are not data analysts by trade but often need to find
  interesting trends in datasets. As a result, this use case will help the
  audience guage the intuitiveness of the user interface, ease of use and fast
  response times.
  \item {\bf Medical dataset:} This dataset is an example of a dataset that a
  researcher (here, a clinical researcher) might use over the course of his/her
  work. This data has a schema that is more complex than the the election
  or store one, and is of larger size too. Since this data is usually analyzed
  by experts, in addition to fast provision of insights, the user also cares about
  flexiblity and ``expert'' operations on this data such as statistical
  information, accuracy of visualizations, drill-downs etc.
\end{itemize}

We envision the demonstration workflow to be as follows: The user selects one of
the three dataset from above for analysis. He/she formulates a selection query
using the SeeDB query builder or by using ready-made queries (e.g. selecting
outliers in a column etc.) and submits the query to SeeDB. SeeDB then searches
through the entire space of possible views using techniques and heuristics
described in Section \ref{optimizations} and returns the top-{\it k} views it
considers most interesting. The SeeDB frontend then visualizes the top-{it k} views and
presents them to the user. The user can interact with each of these views and
perform further exploration through operations such as drill-downs (graphically
selecting subsets of data), comparisons of multiple views etc. The user will
also be able to experiment with the effect of choosing different utility metrics
and optimization strategies described in Section \ref{}.
}
